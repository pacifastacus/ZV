\def\PAPER{a4paper} % papítméret
\def\FONTSIZE{10pt} % betűméret
\def\OPTIONS{twoside} % oldalbeállítások
\def\TARSSZERZO{} % Ha tettél hozzá érdemi munkát, és közreadnád, írd fel magad

\documentclass[magyar,\PAPER,\OPTIONS]{article}
\usepackage{imakeidx}    %Tárgymutatóhoz (index)
\usepackage[utf8]{inputenc}
\usepackage[T1]{fontenc}
\usepackage{mathtools}
\usepackage{units}
\usepackage{graphicx}
\usepackage{subcaption}
\usepackage{textgreek}
\usepackage{lipsum}
\usepackage{amsthm}
\usepackage{multicol}
\usepackage{enumitem}
\usepackage[magyar]{babel}
\usepackage[sharp]{easylist}
\usepackage[
top=2cm,
bottom=2cm,
left=3cm,
right=1cm
]{geometry}
\usepackage[]{hyperref}
%\usepackage{showframe}
%\usepackage{titlesec}

%\titleformat{\section}[display]{\bfseries}{\thesection. tétel:}{0pt}{}[]
\let\OldEasylist\easylist
\let\OldEndEasylist\endeasylist
\renewenvironment{easylist}{
	\OldEasylist
	\ListProperties(Numbers2=l,FinalMark2={)},Hide2=1,Progressive*=3ex, Start1=1)
}{
	\OldEndEasylist
}
\newcounter{descriptcount}
\newcounter{prevdescriptcount}
\newlist{enumdescript}{description}{2}
\setlist[enumdescript,1]{
	before={\setcounter{descriptcount}{0}
		\renewcommand*\thedescriptcount{\arabic{descriptcount}}}
	,font=\bfseries\stepcounter{descriptcount}\thedescriptcount.~
}
\setlist[enumdescript,2]{
	before={\setcounter{prevdescriptcount}{\value{descriptcount}}
		\setcounter{descriptcount}{0}
		\renewcommand*\thedescriptcount{\alph{descriptcount}}}
	,font=\bfseries\stepcounter{descriptcount}\thedescriptcount.~
	,after={\setcounter{descriptcount}{\value{prevdescriptcount}}}
}

% oldal tetejére és aljára került "elárvult" sorok letiltása
\widowpenalty=30000 
\clubpenalty=30000

\newtheorem*{definition}{Definíció}
\newtheorem*{theorem}{Tétel}
\newtheorem*{note}{Megjegyzés}

\makeindex

\title{Záróvizsga tételsor mérnökinformatikus hallgatóknak\\
	{\large A Debreceni Egyetem mérnökinformatikus alapszakához}}
\author{Palkovics Dénes\thanks{Az egyes tételek kidolgozásai nem tőlem származnak. Az esetenként előforduló hibákért és pongyola fogalmazásért felelősséget nem vállalok.} \TARSSZERZO}
\date{2019}


\setlength{\parskip}{6pt plus 3pt minus 3pt} % Bekezdések térközölése
\setcounter{tocdepth}{3} % Tartalomjegyzék mélysége. Alapegyég 3

\begin{document}
\maketitle
%\cleardoublepage
%\begin{titlepage}
\begin{center}
\EGYETEM \\
\KAR \\
%\TANSZEK
\end{center}

\vfill

\begin{center}
\LARGE
\textbf{\CIM}
\normalsize
\end{center}

\vfill

\begin{minipage}[rt]{\linewidth}
\centering
\textit{Készítette:}\\
\textbf{\SZERZO}\\
\SZAK
\end{minipage}

\vfill

\begin{center}
Debrecen, \VEDESEVE
\end{center}

\end{titlepage}

\textbf{ A záróvizsga tematikája és tartalma}\\
A záróvizsgán kettő kérdésre kell válaszolni, egyre az általános kérdések közül, egyre pedig a specializációnak megfelelő kérdések közül
\tableofcontents
\cleardoublepage
\section{Általános kérdések}
%-------------------------------------------------------------------------------
\subsection{Az informatika logikai alapjai}
%-------------------------------------------------------------------------------
\def\InterpretOnNu{^{\langle U, \rho \rangle}_{\nu}}
\subsubsection{Az elsőrendű matematikai logikai nyelv.}
\begin{definition}[Elsőrendű nyelv]
	Klasszikus elsőrendű nyelven az $$ L^{(1)} = \langle LC,Var,Con,Term,Form\rangle $$ rendezett ötöst értjük, ahol\\
	\begin{easylist}
	# $LC = \{\neg,\supset,\land,\lor,\equiv,=,\forall,\exists,(,)\}$ a nyelv logikai konstansainak halmaza\footnote{A logikai konstansok olyan nyelvi eszközök, amelyek jelentését a szemantikai szabályok (logikai kalkulusok esetén az axiómák) rögzítik. Egy adott logikai rendszer esetén a logikai konstansok rögzített jelentéssel (rögzített szemantikai értékkel)rendelkeznek, jelentésük (szemantikai értékük) minden interpretációban megegyezik. Egy adott logikai rendszer esetén a logikai konstansokat általában az adott logikai rendszer nyelvének $LC$	halmaza tartalmazza.}
	# $Var = \{x_{n}| n = 0,1,2,\dots\}$ a nyelv változóinak megszámlálhatóan végtelen halmaza\footnote{A köznyelvi mondatokban nevek helyett néha névmásokkal utalunk egyes individuumokra (objektumokra). A tudományos nyelvben gyakran kívánatos analóg kifejezési formák megadása. A szabatosság, az egyértelműség és a tömörség érdekében ilyenkor mesterséges névmásokat vezetnek be, amelyeket változóknak neveznek.}
	# $Con = \bigcup_{n=0}^\infty(\mathcal{F}(n)\cup\mathcal{P}(n))$ a nyelv nemlogikai konstansainak legfeljebb megszámlálhatóan végtelen halmaza\footnote{A nemlogikai konstansok, más néven paraméterek olyan nyelvi eszközök, amelyek jelentését az interpretáció rögzíti. Egy adott logikai rendszer esetén a nemlogikai konstansok (a paraméterek) nem rendelkeznek rögzített jelentéssel (rögzített szemantikai értékkel), jelentésük (szemantikai értékük) interpretációról interpretációra változhat. Egy adott logikai rendszer esetén a nemlogikai konstansokat általában az adott logikai rendszer nyelvének $Con$ halmaza tartalmazza.}
	## $\mathcal{F}(0)$ a névparaméterek (névkonstansok),
	## $\mathcal{F}(n)$ az $ n $ argumentumú függvényjelek (műveleti jelek),
	## $\mathcal{P}(0)$ a állításparaméterek (állításkonstansok),
	## $\mathcal{P}(n)$ az $ n $ argumentumú predikátumparaméterek (predikátumkonstansok) halmaza.
	# Az $LC,Var,\mathcal{F}(n),\mathcal{P}(n)$ halmazok ($n = 0,1,2,\dots$) páronként diszjunktak.
	# A nyelv terminusainak a halmazát, azaz a $Term$ halmazt az alábbi induktív definíció adja:
	## $Var \cup \mathcal{F}(0)\subseteq Term$
	## Ha $f\in\mathcal{F}(n), (n=1,2,\dots)$,és $t_1,t_2,\dots,t_n \in Term$, akkor $f(t_1,t_2,\dots,t_n)\in Term$
	# \label{itm:induction}A nyelv formuláinak halmazát, azaz a $Form$ halmazt az alábbi induktív definíció adja meg:
	## \label{itm:rule1} $\mathcal{P}\subseteq Form$
	## \label{itm:rule2} Ha $t_1,t_2 \in Term$, akkor $(t_1 = t_2) \in Form$
	## \label{itm:rule3} Ha $P \in \mathcal{P}, (n=1,2,\dots)$, és $t_1,t_2,\dots,t_n\in Term$, akkor $P(t_1,t_2,\dots,t_n)\in Form$
	## Ha $A \in Form$, akkor $\neg A \in Form$
	## Ha $A,b \in Form$, akkor $(A \supset B),(A \land B),(A\lor B),(A\equiv B) \in Form$
	## Ha $x\in Var, A\in Form$, akkor $\forall x A, \exists x A \in Form$
	\end{easylist}
\end{definition}
\begin{note}
	Azokat a formulákat, amelyek a \ref{itm:induction} \ref{itm:rule1}, \ref{itm:rule2}, \ref{itm:rule3} szabályok által jönnek létre, atomi formuláknak\index{atomi formula} vagy prímformuláknak\index{prímforumla} nevezzük.
\end{note}

\subsubsection{A nyelv interpretációja, formulák igazságértéke az interpretációban adott változókiértékelés mellett.}
\begin{definition}[interpretáció (elsőrendű)]
	Az $\langle U, \rho\rangle$ párt az $L^{(1)}$ nyelv egy interpretációjának nevezzük, ha\\
	\begin{easylist}
		# $U \neq \emptyset$ azaz $U$ nemüres halmaz
		# $Dom(\rho) = Con$ azaz a $\rho$ a $Con$ halmazon értelmezett függvény, amelyre teljesülnek a következők:
		## Ha $a \in F(0)$, akkor $\rho(a) \in U$
		## Ha $f \in \mathcal{F}(n)$ ahol $n\neq 0$, akkor $\rho(f)$ az $U^{(n)}$ halmazon értelmezett az $U$ halmazba képező függvény ($\rho(f) : U^{(n)} \rightarrow U $)
		## Ha $p \in \mathcal{P}(0)$, akkor $\rho(p) \in {0, 1}$
		## Ha $P \in \mathcal{P}(n)$ ahol $n \neq 0$, akkor $\rho(P) \subseteq U^{(n)}$
	\end{easylist}
\end{definition}
\begin{definition}[értékelés (elsőrendű)]
	Legyen $L^{(1)} = \langle LC, Var, Con, Term, Form\rangle$ egy elsőrendű nyelv, $\langle U, \rho\rangle$ pedig a nyelv egy interpretációja. Az $\langle U, \rho\rangle$ interpretációra támaszkodó $\nu$ értékelésen egy olyan függvényt értünk, amely teljesíti a következőket:
	\begin{itemize}
		\item  $ Dom(\nu) = Var $
		\item  $ Ha x \in Var$, akkor $\nu(x) \in U $
	\end{itemize}
\end{definition}
\begin{definition}[értékelés (elsőrendű)]
	Legyen $L^(1) = (LC, Var, Con, Term, Form)$ egy elsőrendű nyelv, $\langle U, \rho \rangle$ pedig a nyelv egy interpretációja, $\nu$ pedig az $\langle U, \rho \rangle$ interpretációra támaszkodó értékelés.\\
	\begin{easylist}
		# Ha $a \in F(0)$, akkor 
				$|a|\InterpretOnNu = \rho(a)$
		# Ha $x \in Var$, akkor 
				$|x|^{\langle U,\rho \rangle}_{\nu} = \nu(x)$
		# Ha $f \in F(n)$,$(n = 1,2,\dots)$ és $t_1,t_2,\dots,t_n \in Term$, akkor 
				$$|f(t_1,t_2,\dots,t_n)|\InterpretOnNu = \rho(f)(|t_1|\InterpretOnNu,|t_2|\InterpretOnNu,\dots,|t_n|\InterpretOnNu)$$
		# Ha $p \in P(0)$, akkor
				 $|p|_\nu^{\langle U,\rho\rangle} = \rho(p)$
		# Ha $t_1, t_2 \in Term$, akkor
		\begin{equation}
			|(t_1 = t_2)|\InterpretOnNu =
			\begin{cases}
				1, & \text{ha $|t_1|\InterpretOnNu = |t_2|\InterpretOnNu$}\\
				0, & \text{egyébként.}
			\end{cases}
		\end{equation}
		# Ha $P \in P(n) ahol n = 0, t1 , \dots , tn \in Term$, akkor
		\begin{equation}
			|P(t_1,t_2,\dots,t_n)|\InterpretOnNu =
			\begin{cases}
				1, & \text{ha $\big(|t_1|\InterpretOnNu,|t_2|\InterpretOnNu,\dots,|t_n|\InterpretOnNu\big)\in \rho(P)$}\\
				0, & \text{egyébként.}
			\end{cases}
		\end{equation}
		# Ha $A \in Form$, akkor
			$|\neg A|\InterpretOnNu = 1 - |A|\InterpretOnNu$.
		# Ha $A,B \in Form$, akkor
			\begin{equation}
				|(A\supset B)|\InterpretOnNu
				\begin{cases}
				0, & \text{ha $|A|\InterpretOnNu = 1$, és $|B|\InterpretOnNu = 0$}\\
				1, & \text{egyébként.}
				\end{cases}
			\end{equation}
			
			\begin{equation}
				|(A\land B)|\InterpretOnNu
				\begin{cases}
				1, & \text{ha $|A|\InterpretOnNu = 1$, és $|B|\InterpretOnNu = 1$}\\
				0, & \text{egyébként.}
				\end{cases}
			\end{equation}
			
			\begin{equation}
				|(A\lor B)|\InterpretOnNu
				\begin{cases}
				0, & \text{ha $|A|\InterpretOnNu = 0$, és $|B|\InterpretOnNu = 0$}\\
				1, & \text{egyébként.}
				\end{cases}
			\end{equation}
			
			\begin{equation}
				|(A\equiv B)|\InterpretOnNu
				\begin{cases}
				1, & \text{ha $|A|\InterpretOnNu = |B|\InterpretOnNu$}\\
				0, & \text{egyébként.}
				\end{cases}
			\end{equation}
		# Ha $A \in Form, x \in Var$, akkor
			\begin{equation}
				|(\forall_x A)|\InterpretOnNu =
				\begin{cases}
				0, & \text{ha van olyan $u \in U$, hogy$|A|^{\langle U, \rho \rangle}_{\nu [x:u]} = 0$}\\
				1, & \text{egyébként.}
				\end{cases}
			\end{equation}
			
			\begin{equation}
				|(\exists_x A)|\InterpretOnNu =
				\begin{cases}
				1, & \text{ha van olyan $u \in U$, hogy$|A|^{\langle U, \rho \rangle}_{\nu [x:u]} = 1$}\\
				0, & \text{egyébként.}
				\end{cases}
			\end{equation}
	\end{easylist}
\end{definition}

\subsubsection{Logikai törvény, logikai következmény.}
\begin{definition}[modell]
	Legyen $L^{(1)} = (LC, Var, Con, Term, Form)$ egy elsőrendű nyelv és $\Gamma \subseteq Form$ egy tetszőleges formulahalmaz. Az $(U, \rho, \nu)$ rendezett hármas elsőrendű modellje a $\Gamma$ formulahalmaznak, ha 
	\begin{itemize}
		\item  $(U, \rho)$ egy interpretációja az $L^{(1)}$ nyelvnek; 
		\item  $\nu$ egy $(U, \rho)$ interpretációra támaszkodó értékelés; 
		\item  minden  $A \in \Gamma$ esetén $|A|\InterpretOnNu = 1$.
	\end{itemize}
	
\end{definition}
\begin{definition}
	Legyen $L^{(1)} = (LC, Var, Con, Term, Form)$ egy elsőrendű nyelv és $\Gamma \subseteq Form$ egy tetszőleges formulahalmaz, $A,B \in Form$ egy tetszőleges formulák.
	\begin{itemize}
		\item Egy $\Gamma$ formulahalmaz \emph{kielégíthető}, ha van (elsőrendű) modellje;
		\item Egy $\Gamma$ formulahalmaz \emph{kielégíthetetlen}, ha nem kielégíthető, azaz nincs modellje;
		\item Az $A$ formula \emph{modellje} az $\{A\}$ egyelemű formulahalmaz modelljét értjük;
		\item Az $A$ formula \emph{kielégíthető}, ha $\{A\}$ formulahalmaz kielégíthető;
		\item Az $A$ formula \emph{kielégíthetetlen}, ha $\{A\}$ formulahalmaz kielégíthetetlen;
		\item A  $\Gamma$ formulahalmaznak \underline{logikai következménye} az $A$ formula, ha a $\Gamma \cup \{\neq A\}$ formulahalmaz kielégíthetetlen. Jelölés: $\Gamma \models A$ 
		\item Az $A$ formulának \underline{logikai következménye} a $B$ formula, ha a $\{A\} \models B$. Jelölés: $A \models B$ 
		\item Az $A$ formula \emph{érvényes} (\underline{logikai törvény})\label{def:logikai törvény}, ha $\emptyset \models A$, azaz ha az $A$ formula \underline{logikai következménye} az üres halmaznak. Másképpen, ha minden $ \langle U, \rho \rangle$ interpretációjában, minden $\nu$ értékelés szempontjából $|A|\InterpretOnNu = 1$ Jelölés: $\models A$
		\item Az $A$ és a $B$ formula \emph{logikailag ekvivalens}, ha $A \models B$ és $B \models A$. Jelölés: $A \Leftrightarrow B$ 
	\end{itemize}
\end{definition}

\subsubsection{Logikai ekvivalencia, normálformák.}
\begin{definition}[Logikai ekvivalencia]
	lásd:a \ref{def:logikai törvény} fejezet definíciója.
\end{definition}
\begin{definition}[elemi konjunkció]
	Legyen $L^{(0)} = (LC, Con, Form)$ egy nulladrendű nyelv. Ha az $A \in Form$ formula literál vagy különböző alapú literálok konjunkciója, akkor $A$-t elemi konjunkciónak nevezzük. 
\end{definition}
\begin{definition}[elemi diszjunkció]
	Legyen $L^{(0)} = (LC, Con, Form)$ egy nulladrendű nyelv. Ha az $A \in Form$ formula literál vagy különböző alapú literálok diszjunkciója, akkor $A$-t elemi diszjunkciónak nevezzük. 
\end{definition}
\begin{definition}[diszjunktív normálforma]
	Egy elemi konjunkciót vagy elemi konjunkciók diszjunkcióját diszjunktív normálformának nevezzük. 
\end{definition}
\begin{definition}[konjunktív normálforma]
	Egy elemi diszjunkciót vagy elemi diszjunkciók konjunkcióját konjunktív normálformának nevezzük.
\end{definition}
\begin{definition}
	Legyen $L^{(0)} = (LC, Con, Form)$ egy nulladrendű nyelv és $A \in Form$ egy formula. Ekkor létezik olyan $B \in Form$, hogy
	\begin{itemize}
		\item $A\Leftrightarrow B$
		\item $B$ diszjunktív vagy konjunktív normálformájú. 
	\end{itemize}

\end{definition}

\subsubsection{Kalkulusok (Gentzen-kalkulus).}

\paragraph{Logikai kalkulus} 
Logikai kalkuluson olyan adott nyelv formuláihoz tartozó formális rendszert, szabályrendszert értünk, amely pusztán szintaktikailag, szemantika nélkül ad meg egy következményrelációt. A logikai kalkulus tehát egy axiómarendszer, amely magában a logikai tautológiákat állítja elő, adott formulákat ideiglenesen hozzávéve (premissza) pedig más formulákra (konklúzió) lehet jutni (következtetni) vele.

\paragraph{Gentzen-féle szekvenciakalkulus}
Ebben a kalkulusban nem formulákra vonatkoznak a szabályok és nem is formulák alkotják az axiómákat, hanem a formulák eddigi szerepét az ún. szekvencia töltik be. Szekvenciának nevezzük a
$$\Gamma \vdash \Delta$$
alakú jelsorozatokat, ahol $\Gamma$ és $\Delta$ olyan rendezett jelsorozatok, amelyeknek minden tagja egy formula.
\begin{definition}[axiómasémák]
	Legyen $L^{(0)} = (LC, Con, Form)$ egy nulladrendű nyelv (a klasszikus állításlogika nyelve). A nulladrendű kalkulus (klasszikus állításkalkulus) axiómasémái (alapsémái):
	\begin{easylist}
		# $A \supset (B \supset A)$
		# $(A \supset (B \supset C)) \supset ((A \supset B) \supset (A \supset C))$
		# $(\neg A \supset \neg B) \supset (B \supset A)$
	\end{easylist}
\end{definition}

Az axiómaséma szabályos behelyettesítésén olyan formulát értünk, amely az axiómasémából a benne szereplő betűk tetszőleges formulával való helyettesítése útján jön létre. A nulladrendű kalkulus (klasszikus állításkalkulus) axiómái az axiómasémák szabályos behelyettesítései. 

\begin{definition}[szintaktikai következmény]
	Legyen $L^{(0)} = (LC, Con, Form)$ egy nulladrendű nyelv, $\Gamma \subseteq Form$ egy tetszőleges formulahalmaz. A $\Gamma$ formulahalmaz szintaktikai következményeinek induktív definíciója:

	Bázis:
	\begin{itemize}
		\item Ha $A \in \Gamma$, akkor $\Gamma \vdash A$ 
		\item Ha $A$ axióma, akkor $\Gamma \vdash A$. 
	\end{itemize}
	
	Szabály (leválasztási szabály): 
	\begin{itemize}
		\item Ha $\Gamma \vdash B$, és $\Gamma \vdash (B \subset A)$, akkor $\Gamma \vdash A$. 
	\end{itemize}
\end{definition}

\begin{definition}[szintaktikai következmény]
Legyen $L^{(0)} = (LC, Con, Form)$ egy nulladrendű nyelv és $A, B \in Form$ két tetszőleges formula. Az $A$ formulának szintaktikai következménye a $B$ formula, ha $\{A\} \vdash B$. Jelölés: $A \vdash B$ 
\end{definition}

\begin{definition}[szekvencia]
Legyen $L^{(0)} = (LC, Con, Form)$ egy nulladrendű nyelv, $\Gamma \subseteq Form$ egy formulahalmaz és $A \in Form$ egy formula. Ha az $A$ formula szintaktikai következménye a $\Gamma$ formulahalmaznak, akkor a $\Gamma \vdash A$ jelsorozatot szekvenciának nevezzük. 
\end{definition}

\begin{definition}[levezethetőség]
Legyen $L^{(0)} = (LC, Con, Form)$ egy nulladrendű nyelv és $A \in Form$ egy tetszőleges formula. Az $A$ formula levezethető, ha $\emptyset \vdash A$, azaz ha az $A$ formula szintaktikai következménye az üres halmaznak. Jelölés: $\vdash A$ 
\end{definition}

\begin{definition}[természetes levezetés szabályai]
Legyen $L^{(0)} = (LC, Con, Form)$ egy nulladrendű nyelv  $\Gamma, \Delta \subseteq Form$ és $A, B, C \in Form$. A természetes levezetés által az $L^{(0)}$ nyelvben bizonyítható következményrelációk alábbiak:

Bázis:
\begin{equation}
	\frac{\omega}{\Gamma,A \vdash A}
\end{equation}
Szabályok:
\begin{itemize}
	\item Struktúrális szabályok:
	\begin{itemize}
		\item Bővítés $\frac{\Gamma \vdash A}{\Gamma, B \vdash A} $
		\item Felcserélés $\frac{\Gamma,B,C,\Delta\vdash A} {\Gamma,C,B,\Delta\vdash A} $
		\item Szűkítés $\frac{\Gamma,B,B,\Delta\vdash A}
		{\Gamma,B,\Delta\vdash A} $
		\item Metszet $\frac{\Gamma\vdash A \Delta,A\vdash B}{\Gamma,\Delta\vdash B} $
	\end{itemize}
	\item Logikai szabályok:
	\begin{itemize}
		\item Implikáció szabályai:
		\begin{itemize}
			\item bevezető: $\frac{\Gamma,A\vdash B}{\Gamma\vdash A \supset B} $
			\item alkalmazó: $\frac{\Gamma\vdash A \Gamma\vdash A \supset B}{\Gamma \vdash B} $
		\end{itemize}
		\item Negáció szabályai:
		\begin{itemize}
			\item bevezető:$\frac{\Gamma,A\vdash B \Gamma,A\vdash \neg B}{\Gamma\vdash \neg A} $
			\item alkalmazó:$\frac{\Gamma\vdash \neg\neg A}{\Gamma \vdash A}$
		\end{itemize}
		\item Konjunkció szabályai:
		\begin{itemize}
			\item bevezető:$\frac{\Gamma\vdash A \Gamma\vdash B}{\Gamma\vdash A\land B}$
			\item alkalmazó:$\frac{\Gamma,A,B\vdash C}{\Gamma,A\land B,\vdash C}$
		\end{itemize}
		\item Diszjunkció szabályai:
		\begin{itemize}
			\item bevezető:$\frac{\Gamma\vdash A}{\frac{\Gamma\vdash A \lor B \Gamma\vdash B}{\Gamma\vdash A \lor B}}$
			\item alkalmazó:$\frac{\Gamma,A\vdash C \Gamma,B\vdash C}{\Gamma,A\lor B\vdash C}$
		\end{itemize}
		\item (Materiális) ekvivalencia szabályai:
		\begin{itemize}
			\item bevezető:$\frac{\Gamma,A\vdash B \Gamma,B\vdash A}{\Gamma\vdash A\equiv B}$
			\item alkalmazó:$\frac{\Gamma\vdash A \Gamma\vdash A\equiv B}{\frac{\Gamma\vdash B \Gamma\vdash B \Gamma\vdash A \equiv B}{\Gamma\vdash A}}$
		\end{itemize}
	\end{itemize}
\end{itemize}
\end{definition}

%-------------------------------------------------------------------------------
%-------------------------------------------------------------------------------
\section{Operációs rendszerek}
{\footnotesize Operációs rendszerek fogalma, felépítése, osztályozásuk. Az operációs rendszerek jellemzése 	(komponensei és funkciói). A rendszeradminisztráció, fejlesztői és alkalmazói támogatás eszközei.}
%-------------------------------------------------------------------------------
\subsection{Operációs rendszerek fogalma, felépítése, osztályozásuk.}
\paragraph{Operációs rendszerek fogalma}
Egy program, amely közvetítő szerepet játszik a számítógép felhasználója
és a számítógéphardver között.
Az operációs rendszer feladata, hogy a felhasználónak egy olyan egyenértékű kiterjesztett
vagy virtuális gépet nyújtson, amelyiket egyszerűbb programozni, mint a mögöttes hardvert
\paragraph{Operációs rendszerek felépítése}
Az operációs rendszerek alapvetően három részre bonthatók:
	\begin{itemize}[nosep]
	\item a felhasználói felület (a shell, amely lehet egy grafikus felület, vagy egy szöveges)
	\item alacsony szintű segédprogramok
	\item kernel (mag), amely közvetlenül a hardverrel áll kapcsolatban.
	\end{itemize}
\paragraph{Operációs rendszerek osztályozása}
	\begin{enumerate}[nosep]
	\item Az operációs rendszer alatti hardver "mérete" szerint:
		\begin{itemize}[nosep]
		\item mikroszámítógépek operációs rendszerei
		\item kisszámítógépek, esetleg munkaállomások operációs rendszerei
		\item nagygépek (Main Frame Computers, Super Computers) operációs rendszerei
		\end{itemize}
	\item A kapcsolattartás típusa szerint:
		\begin{itemize}[nosep]
		\item kötegelt feldolgozású operációs rendszerek vezérlőkártyás kapcsolattartással
		\item interaktív operációs rendszerek.
		\end{itemize}
	\item cél szerint: általános felhasználású vagy céloperációs rendszer
	\item a processzkezelés: single-tasking, multi-tasking
	\item a felhasználók száma szerint: single, multi
	\item CPU-idő kiosztása szerint: szekvenciális, megszakítás vezérelt, event-polling, time-sharing
	\item a memóriakezelés megoldása szerint: valós és virtuális címzésű
	\end{enumerate}

\subsection{Az operációs rendszerek jellemzése (komponensei és funkciói).}
\paragraph{Operációs rendszerek komponensei:}
\begin{description}[nosep]
	\item[Eszközkezelők (Device Driver)] Felhasználók elől el fedik a perifériák különbségeit, egységes kezelői felületet kell biztosítani.
	\item[Megszakítás kezelés (Interrupt Handling)] Alkalmas perifériák felől érkező kiszolgálási igények fogadására, megfelelő ellátására.
	\item[Rendszerhívás, válasz (System Call, Reply)] az operációs rendszer magjának ki kell szolgálnia a felhasználói alkalmazások (programok) erőforrások iránti igényeit úgy, hogy azok lehetőleg észre se vegyék azt, hogy nem közvetlenül használják a perifériákat$\leftarrow$ programok által kiadott rendszerhívások, melyekre rendszermag válaszokat küldhet.
	\item[Erőforrás kezelés (Resource Management)] Az egyes eszközök közös használatából származó konfliktusokat meg kell előznie, vagy bekövetkezésük esetén fel kell oldania.
	\item[Processzor ütemezés (CPU Scheduling)] Az operációs rendszerek ütemező funkciójának a várakozó munkák között valamilyen stratégia alapján el kell osztani a processzor idejét, illetve vezérelnie kell a munkák közötti átkapcsolási folyamatot.
	\item[Memóriakezelés (Memory Management)] Gazdálkodnia kell a memóriával, fel kell osztania azt a munkák között úgy, hogy azok egymást se zavarhassák, és az operációs renszerben se tegyenek kárt.
	\item[Állomány- és lemezkezelés (File and Disk Management)] Rendet kell tartania a hosszabb távra megőrzendő állományok között.
	\item[Felhasználói felület (User Interface)] A parancsnyelveket feldolgozó monito utódja, 	fejlettebb változata, melynek segítségével a felhasználó közölni tudja a rendszermaggal kívánságait, illetve annk állapotáról információt szerezhet.
\end{description}

\paragraph{Operációs rendszerek funkciói:}
\begin{description}[nosep]
\item[Folyamatkezelés]
A folyamat egy végrehajtás alatt álló program. Hogy feladatát ellássa erőforrásokra van szüksége (processzor idő, memória, állományok I/O berendezések).
Az operációs rendszer feladata:
\begin{itemize}[nosep]
	\item Folyamatok létrehozása és törlése
	\item Folyamatok felfüggesztése és újraindítása
	\item Eszközök biztosítása a folyamatok kommunikációjához és szinkronizációjához.
\end{itemize}
\item[Memória (főtár) kezelés]
Bájtokból álló tömbnek tekinthető, amelyet a CPU és az I/O közösen használ. Tartalma törlődik rendszerkikapcsoláskor és rendszerhibáknál.
Az operációs rendszer feladata:
\begin{itemize}[nosep]
	\item Nyilvántartani, hogy az operatív memória melyik részét ki (mi) használja.
	\item Eldönteni melyik folyamatot kell betölteni, ha memória felszabadul.
	\item Szükség szerint allokálni és felszabadítani a memória területeket a szükségleteknek megfelelően.
\end{itemize}
\item[Másodlagos tárkezelés]
Nem törlődik, és elég nagy hogy minden programot tároljon. A merevlemez a legelterjedtebb
formája. Az operációs rendszer feladata:
\begin{itemize}[nosep]
	\item Szabadhely kezelés.
	\item Tárhozzárendelés.
	\item Lemez elosztás.
\end{itemize}
\item[I/O rendszerkezelés] ~
\begin{itemize}[nosep]
	\item Puffer rendszer.
	\item Általános készülék meghajtó (device driver) interface.
	\item Speciális készülék meghajtó programok.
\end{itemize}
\item[Fájlkezelés]
Egy fájl kapcsolódó információk együttese, amelyet a létrehozója definiál. Általában program és adatfájlokról beszélünk.
Az operációs rendszer feladata:
\begin{itemize}[nosep]
	\item Fájlok és könyvtárak létrehozás és törlése.
	\item Fájlokkal és könyvtárakkal történő alapmanipuláció.
	\item Fájlok leképezése a másodlagos tárra, valamilyen nem törlődő, stabil adathordozóra.
\end{itemize}
\item[Védelmi rendszer]
Olyan mechanizmus, mely az erőforrásokhoz való hozzá férést felügyeli. Az operációs rendszer feladata:
\begin{itemize}[nosep]
	\item Különbséget tenni jogos (authorizált) és jogtalan használat között.
	\item Specifikálni az alkalmazandó kontrolt.
	\item Korlátozó eszközöket szolgáltatni.
\end{itemize}
\item[Hálózat elérés támogatása]
Az elosztott rendszer processzorok adat és vezérlő vonallal összekapcsolt együttese, ahol a memória és az óra nem közös. Adat- és vezérlővonal segítségével történik a kommunikáció. Az elosztott rendszer a felhasználóknak különböző osztott erőforrások elérését teszi lehetővé, mely lehetővé teszi:
\begin{itemize}[nosep]
	\item a számítások felgyorsítását,
	\item a jobb adatelérhetőséget,
	\item a nagyobb megbízhatóságot.
\end{itemize}
\item[Parancs interpreter alrendszer]
Az operációs rendszernek sok parancsot vezérlő utasítás formájában lehet megadni. Vezérlő utasítások minden területhez tartoznak (folyamatok, I/O kezelés...). Az operációs rendszernek azt a programját, amelyik a vezérlő utasítást beolvassa és interpretálja a rendszertől függően más és más módon nevezhetik:
\begin{itemize}[nosep]
	\item Vezérlő kártya interpreter.
	\item Parancs sor interpreter (command line).
	\item Héj (burok, shell)
\end{itemize}
\end{description}

\subsection{A rendszeradminisztráció, fejlesztői és alkalmazói támogatás eszközei.}
\paragraph{Rendszeradminisztráció}
Magának az operációs rendszernek a működtetésével kapcsolatos funkciók. Ezek közvetlenül semmire sem használhatók, csak a hardverlehetőségek kibővítését célozzák, illetve a hardver kezelését teszik kényelmesebbé. A rendszeradminisztráción belül a következő \emph{összetett funkciókat} jelölhetjük ki:
\begin{enumdescript}[nosep]
	\item[processzorütemezés:] a CPU-idő szétosztása a rendszer- és a felhasználói feladatok
	(taszkok, folyamatok) között;
	\item[megszakításkezelés:] a hardver-szoftver megszakításkérések elemzése, állapotmentés,
	a kezelőprogram hívása;
	\item[szinkronizálás:] az események és az erőforrásigények várakozási sorokba állítása;
	\item[folyamatvezérlés:] a programok indítása és a programok közötti kapcsolatok
	szervezése;
	\item[tárkezelés:] a főtár, -- mint kiemelten kezelt erőforrás, -- elosztása;
	\item[perifériakezelés:] a bemeneti/kimeneti (B/K ill. I/O) igények sorba állítása és
	kielégítése;
	\item[adatkezelés:] az adatállományokon végzett műveletek segítése (létrehozás, nyitás,
	zárás, írás, olvasás stb.);
	\item[működés-nyilvántartás:] a hardver hibastatisztika vezetése és a számlaadatok
	feljegyzése;
	\item[operátori interfész:] a kapcsolattartás az üzemeltetővel.
\end{enumdescript}
A konkrét operációs rendszerek a funkciókat másképpen oszthatják fel. Így például az IBM OS operációs rendszerek változataiban négy fő funkciót szoktak megkülönböztetni:
\begin{enumerate}[nosep]
	\item a munkakezelést,
	\item a taszkkezelést,
	\item az adatkezelést és
	\item a rendszerstatisztikát.
\end{enumerate}
A rendszeradminisztrációs funkciókat a \textbf{rendszermag} valósítja meg, amelynek a szolgáltatásait a már említett rendszerhívásokkal érhetjük el.

\paragraph{Programfejlesztési támogatás} fő funkciói:
\begin{enumdescript}[nosep]
	\item[rendszerhívások:] a programokból alacsony szintű operációsrendszeri funkciók
	aktivizálására,
	\item[szövegszerkesztők:] a programok és dokumentációk írására,
	\item[programnyelvi eszközök:] fordítóprogramok és interpreterek (értelmezők) a nyelvek
	fordítására vagy értelmezésére,
	\item[szerkesztő- és betöltő-programok:] a programmodulok összefűzésére illetve tárba
	töltésére (végcímzés),
	\item[programkönyvtári funkciók:] a különböző programkönyvtárak használatára,
	\item[nyomkövetési rendszer:] a programok belövésére.
\end{enumdescript}

\paragraph{Alkalmazói támogatás}
Az alkalmazói támogatás funkciói a számítógépes rendszer több szintjén valósulnak meg, és az alábbi fő funkciókra bonthatók:
\begin{enumdescript}[nosep]
	\item[operátori parancsnyelvi rendszer:] a számítógép géptermi üzemvitelének
	támogatására;
	\item[munkavezérlő parancsnyelvi rendszer:] a számítógép alkalmazói szintű
	igénybevételének megfogalmazására;
	\item[rendszerszolgáltatások:] az operációs rendszer magjával közvetlenül meg nem oldható
	rendszerfeladatokra;
	\item[segéd-programkészlet:] rutinfeladatok megoldására;
	\item[alkalmazói programkészlet:] az alkalmazásfüggő feladatok megoldására
\end{enumdescript}

%----------------------------------------------------------------------------
\section{Magas szintű Programozási nyelvek}
{\footnotesize Adattípus, konstans, változó, kifejezés. Paraméterkiértékelés, paraméterátadás. Hatáskör, névterek, élettartam. Fordítási egységek, kivételkezelés.}
%----------------------------------------------------------------------------
\subsection{Adattípus, konstans, változó, kifejezés.}
Az adatabsztrakció első megjelenési formája az adattípus\index{adattípus} a programozási nyelvekben. Az adattípus maga egy absztrakt programozási eszköz, amely mindig más, konkrét programozási eszköz egy komponenseként jelenik meg. Az adattípusnak neve van, ami egy azonosító. A programozási nyelvek
egy része ismeri ezt az eszközt, más része nem. Ennek megfelelően beszélünk típusos és nem típusos nyelvekről. Az eljárásorientált nyelvek típusosak. Egy adattípust három dolog határoz meg, ezek:
\begin{enumerate}[noitemsep]
	\item tartomány
	\item műveletek
	\item reprezentáció
\end{enumerate}
Az adattípusok tartománya azokat az elemeket tartalmazza, amelyeket az adott típusú konkrét programozási eszköz fölvehet értékként. Bizonyos típusok esetén a tartomány elemei jelenhetnek meg a programban literálként. Az adattípushoz hozzátartoznak azok a műveletek, amelyeket a tartomány elemein végre tudunk hajtani. Minden adattípus mögött van egy megfelelő belső ábrázolási mód. A reprezentáció az egyes típusok tartományába tartozó értékek tárban való megjelenését határozza meg, tehát azt, hogy az egyes elemek hány bájtra és milyen bitkombinációra képződnek le. Minden típusos nyelv rendelkezik beépített (standard) típusokkal. Egyes nyelvek lehetővé teszik azt, hogy a
programozó is definiálhasson típusokat. A saját típus definiálási lehetőség az adatabsztrakciónak egy magasabb szintjét jelenti, segítségével a valós világ egyedeinek tulajdonságait jobban tudjuk modellezni.
A saját típus definiálása általában szorosan kötődik az absztrakt adatszerkezetekhez. Saját típust úgy tudunk létrehozni, hogy megadjuk a tartományát, a műveleteit és a reprezentációját. Szokásos, hogy saját típust a beépített és a már korábban definiált saját típusok segítségével adjuk meg. Általános, hogy a reprezentáció megadásánál így járunk el. Csak nagyon kevés nyelvben lehet saját reprezentációt megadni (ilyen az Ada). Kérdés, hogy egy nyelvben lehet-e a saját típushoz saját műveleteket és saját operátorokat megadni. Van, ahol igen, de az is lehetséges, hogy a műveleteket alprogramok (l. 5.1. alfejezet) realizálják. %TODO alfejezet?
A tartomány megadásánál is alkalmazható a visszavezetés technikája, de van olyan lehetőség is, hogy explicit módon adjuk meg az elemeket. Az egyes adattípusok, mint programozási eszközök önállóak, egymástól különböznek. Van azonban egy speciális eset, amikor egy típusból (ez az alaptípus) úgy tudok származtatni egy másik típust (ez lesz az altípus), hogy leszőkítem annak tartományát, változatlanul hagyva műveleteit és reprezentációját. Az alaptípus és az altípus tehát nem különböző típusok.
Az \emph{adattípusoknak} két nagy csoportjuk van:
\begin{description}
	\item[skalár vagy egyszerű adattípus] tartománya atomi értékeket tartalmaz, minden érték egyedi, közvetlenül nyelvi eszközökkel tovább nem bontható. A skalár típusok tartományaiból vett értékek jelenhetnek meg literálként a program szövegében.
	\item[strukturált vagy összetett adattípus] tartományának elemei maguk is valamilyen típussal rendelkeznek. Az elemek egy-egy értékcsoportot képviselnek, nem atomiak, az értékcsoport elemeihez külön-külön is hozzáférhetünk. Általában valamilyen absztrakt adatszerkezet programnyelvi megfelelői. Literálként általában nem jelenhetnek meg, egy konkrét értékcsoportot explicit módon kell megadni.
\end{description}

\subsubsection{Egyszerű típusok}
Minden nyelvben létezik az egész típus\index{egész típus}, sőt általában egész típusok. Ezek belső ábrázolása fixpontos. Az egyes egész típusok az ábrázoláshoz szükséges bájtok számában térnek el és nyilván ez határozza meg a tartományukat is. Néhány nyelv ismeri az előjel nélküli egész típust, ennek belső ábrázolása előjel nélküli (direkt).

Alapvetőek a valós típusok\index{valós típus}, belső ábrázolásuk lebegőpontos. A tartomány itt is az alkalmazott ábrázolás függvénye, ez viszont általában implementációfüggő.
Az egész és valós típusokra közös néven, mint numerikus típusokra\index{numerikus típus} hivatkozunk. A numerikus típusok értékein a numerikus és hasonlító műveletek hajthatók végre.
A karakteres típus tartományának elemei karakterek, a karakterlánc vagy sztring típuséi pedig karaktersorozatok. Ábrázolásuk karakteres (karakterenként egy vagy két bájt, az alkalmazott kódtáblától függően), műveleteik a szöveges és hasonlító műveletek.

Egyes nyelvek ismerik a logikai típust. Ennek tartománya a hamis és igaz értékekből áll, műveletei a logikai és hasonlító műveletek, belső ábrázolása logikai.

Speciális egyszerű típus a felsorolásos típus. A felsorolásos típust\index{felsorolásos típus} saját típusként kell létrehozni. A típus definiálása úgy történik, hogy megadjuk a tartomány elemeit. Ezek azonosítók lehetnek. Az elemekre alkalmazhatók a hasonlító műveletek.

Egyes nyelvek értelmezik az egyszerű típusok egy speciális csoportját, a sorszámozott típust\index{sorszámozott típus}. Ebbe a csoportba tartoznak általában az egész, karakteres, logikai és felsorolásos típusok. A sorszámozott típus tartományának elemei listát (mint absztrakt adatszerkezetet) alkotnak, azaz van első és utolsó elem, minden elemnek van megelőzője (kivéve az elsőt) és minden elemnek van rákövetkezője (kivéve az utolsót). Tehát az elemek között egyértelmű sorrend értelmezett. A tartomány elemeihez kölcsönösen egyértelműen hozzá vannak rendelve a 0,~1,~2,~\dots sorszámok. Ez alól kivételt képeznek az egész típusok, ahol a tartomány minden eleméhez önmaga, mint sorszám van hozzárendelve. Egy sorszámozott típus esetén mindig értelmezhetők a következő műveletek:
\begin{itemize}[noitemsep]
	\item ha adott egy érték, meg kell tudni mondani a sorszámát, és viszont
	\item bármely értékhez meg kell tudni mondani a megelőzőét és a rákövetkezőjét
\end{itemize}
A sorszámozott típus az egész típus egyfajta általánosításának tekinthető. Egy sorszámozott típus altípusaként lehet származtatni az intervallum típust.

\paragraph{Mutató típus} Lényegében egyszerű típus, specialitását az adja, hogy tartományának elemei tárcímek. A mutató típus segítségével valósítható meg a programnyelvekben az indirekt címzés. A mutató típusú programozási eszköz értéke tehát egy tárbeli cím, így azt mondhatjuk, hogy az adott eszköz a tár adott területét címzi, az adott tárterületre „mutat”. A mutató típus egyik legfontosabb művelete a megcímzett tárterületen elhelyezkedő érték elérése. A mutató típus tartományának van egy speciális eleme, amely nem valódi tárcím. Tehát ezzel az értékkel rendelkező mutató típusú programozási eszköz „nem mutat sehova”. A nyelvek ezt az értéket általában beépített nevesített konstanssal kezelik. A mutató típus alapvető szerepet játszik az absztrakt adatszerkezetek szétszórt reprezentációját kezelő implementációknál.

\subsubsection{Összetett típusok}
Az eljárás orientált nyelvek két legfontosabb összetett típusa a tömb (melyet minden nyelv ismer) és a rekord (egyes nyelvek, pl. a FORTRAN nem ismerik).
\paragraph{A tömb típus} absztrakt adatszerkezet megjelenése típus szinten. A tömb statikus és homogén összetett típus, vagyis tartományának elemei olyan értékcsoportok, amelyekben az elemek száma azonos, és az elemek azonos típusúak. A tömböt, mint típust meghatározza:
\begin{itemize}[noitemsep]
	\item dimenzióinak száma
	\item indexkészletének típusa és tartománya
	\item elemeinek a típusa
\end{itemize}
Egyes nyelvek (pl. a C) nem ismerik a többdimenziós tömböket. Ezek a nyelvek a többdimenziós tömböket úgy képzelik el, mint olyan egydimenziós tömbök, amelyek elemei egydimenziós tömbök. Többdimenziós tömbök reprezentációja lehet sor- vagy oszlop-folytonos. Ez általában implementációfüggő, a sorfolytonos a gyakoribb. Ha van egy tömb típusú programozási eszközünk, akkor a nevével az összes elemre együtt, mint egy értékcsoportra tudunk hivatkozni (az elemek sorrendjét a reprezentáció határozza meg). Az értékcsoport egyes elemeire a programozási eszköz neve után megadott indexek segítségével hivatkozunk. Az indexek a nyelvek egy részében szögletes, másik részében kerek zárójelek között állnak. Egyes nyelvek (pl. COBOL, PL/Ő) megengedik azt is, hogy a tömb egy adott dimenziójának összes elemét (pl. egy kétdimenziós tömb egy sorát) együtt hivatkozhassuk.

A nyelveknek a tömb típussal kapcsolatban a következő kérdéseket kell megválaszolniuk:
\begin{itemize}[noitemsep]
	\item Milyen típusúak lehetnek az elemek?
	\item Milyen típusú lehet az index?
	\item Amikor egy tömb típust definiálunk, hogyan kell megadni az indextartományt?
	\item Hogyan lehet megadni az alsó és a felső határt, illetve a darabszámot?
\end{itemize}
A tömb típus alapvető szerepet játszik az absztrakt adatszerkezetek folytonos ábrázolását megvalósító
implementációknál.
\paragraph{A rekord típus} absztrakt adatszerkezet megjelenése típus szinten. A rekord típus minden esetben heterogén, a tartományának elemei olyan értékcsoportok, amelyeknek elemei különböző típusúak lehetnek. Az értékcsoporton belül az egyes elemeket mezőnek nevezzük. Minden mezőnek saját, önálló neve (ami egy azonosító) és saját típusa van. A különböző rekord típusok mezőinek neve megegyezhet.

A nyelvek egy részében (pl. C) a rekord típus statikus, tehát a mezők száma minden értékcsoportban azonos. Más nyelvek esetén (pl. Ada) van egy olyan mezőegyüttes, amely minden értékcsoportban szerepel (a rekord fix része), és van egy olyan mezőegyüttes, amelynek mezői közül az értékcsoportokban csak bizonyosak szerepelnek (a rekord változó része). Egy külön nyelvi eszköz (a diszkriminátor) szolgál annak megadására, hogy az adott konkrét esetben a változó rész mezői közül melyik jelenjen meg.Az ősnyelvek (pl. PL/Ő, COBOL) többszintű rekord típussal dogoznak. Ez azt jelenti, hogy egy mező felosztható újabb mezőkre, tetszőleges mélységig, és típus csak a legalsó szintű mezőkhöz rendelhető, de az csak egyszerű típus lehet. A későbbi nyelvek (pl. Pascal, C, Ada) rekord típusa egyszintű, azaz nincsenek almezők, viszont a mezők típusa összetett is lehet.

Egy rekord típusú programozási eszköz esetén az eszköz nevével az értékcsoport összes mezőjére hivatkozunk egyszerre (a megadás sorrendjében). Az egyes mezőkre külön minősített névvel tudunk hivatkozni, ennek alakja:

{
\centering
\verb|eszköznév.mezőnév|\\
}
Az eszköz nevével történő minősítésre azért van szükség, mert a mezők nevei nem szükségszerűen egyediek. A rekord típus alapvető szerepet játszik az input-outputnál.
\subsubsection{Literálok vagy konstansok}
A literál olyan programozási eszköz, amelynek segítségével fix, explicit értékek építhetők be a program szövegébe. A literáloknak két komponensük van: típus és érték. A literál mindig önmagát definiálja. A literál felírási módja (mint speciális karaktersorozat) meghatározza mind a típust, mind az értéket. A nyelveknek saját literál rendszerük van.
\paragraph{Nevesített konstans}
Ez már konkrét eszköz, melynek három komponense van: név, típus, érték. Jelentősége: programozás technikai eszköz. A programozás szövegében a nevével jelenik meg, de az értéket jelenti. Azon programozási eszközökhöz, melyeknek van neve (név komponense) kötődik a deklaráció (a nyelvekben speciális utasítások állnak rendelkezésre). Mindhárom komponense a deklarációnál dől el (ott kell megadni, csak ott lehet megadni). Vannak olyan literálok, melyeknek nincs szemantikai értékük, ezeket, ha nevesítjük, beszélő névvel láthatjuk el. Technikai problémákat egyszerűsít, ha a program szövegében meg akarjuk változtatni ezt a névvel ellátott értéket, akkor nem kell annak valamennyi előfordulását megkeresni és átírni, hanem elegendő egy helyen, a deklarációs utasításban végrehajtani a módosítást.

\subsubsection{Változó}
A változó olyan programozási eszköz, amelynek négy komponense van:
\begin{enumerate}[noitemsep]
	\item név
	\item attribútumok
	\item cím
	\item érték
\end{enumerate}
A \emph{név} egy azonosító. A program szövegében a változó mindig a nevével jelenik meg, az viszont bármely komponenst jelentheti. Szemlélhetjük úgy a dolgokat, hogy a másik három komponenst a névhez rendeljük hozzá.\\
Az \emph{attribútumok} olyan jellemzők, amelyek a változó futás közbeni viselkedését határozzák meg. Az eljárás-orientált nyelvekben (általában a típusos nyelvekben) a legfőbb attribútum a típus, amely a változó által felvehető értékek körét határolja be. Változóhoz attribútumok deklaráció segítségével rendelődnek. A deklarációnak különböző fajtáit ismerjük.
\begin{description}
	\item[Explicit deklaráció] A programozó végzi explicit deklarációs utasítás segítségével. A változó teljes nevéhez kell az attribútumokat megadni. A nyelvek általában megengedik, hogy egyszerre több változónévhez ugyanazokat az attribútumokat rendeljük hozzá.
	\item[Implicit deklaráció] A programozó végzi, betűkhöz rendel attribútumokat egy külön deklarációs utasításban. Ha egy változó neve nem szerepel explicit deklarációs utasításban, akkor a változó a nevének kezdőbetűjéhez rendelt attribútumokkal fog rendelkezni, tehát az azonos kezdőbetűjű változók ugyanolyan attribútumúak lesznek.
	\item[Automatikus deklaráció] A fordítóprogram rendel attribútumot azokhoz a változókhoz, amelyek nincsenek explicit módon deklarálva, és kezdőbetűjükhöz nincs attribútum rendelve egy implicit deklarációs utasításban. Az attribútum hozzárendelése a név valamelyik karaktere (gyakran az első) alapján történik:
\end{description}
Az eljárás-orientált nyelvek mindegyike ismeri az explicit deklarációt, és egyesek csak azt ismerik. Az utóbbiak általánosságban azt mondják, hogy minden névvel rendelkező programozói eszközt explicit módon deklarálni kell. A változó címkomponense a tárnak azt a részét határozza meg, ahol a változó értéke elhelyezkedik. A futási idő azon részét, amikor egy változó rendelkezik címkomponenssel, a változó élettartamának hívjuk. Egy változóhoz cím rendelhető az alábbi módokon:
\begin{description}
	\item[Statikus tárkiosztás] A futás előtt eldől a változó címe, és a futás alatt az nem változik. Amikor a program betöltődik a tárba, a statikus tárkiosztású változók fix tárhelyre kerülnek.
	\item[Dinamikus tárkiosztás] A cím hozzárendelését a futtató rendszer végzi. A változó akkor kap címkomponenst, amikor aktivizálódik az a programegység, amelynek ő lokális változója, és a címkomponens megszűnik, ha az adott programegység befejezi a működését. A címkomponens a futás során változhat, sőt vannak olyan időintervallumok, amikor a változónak nincs is címkomponense. \item[A programozó által vezérelt tárkiosztás] A változóhoz a programozó rendel címkomponenst futási időben. A címkomponens változhat, és az is elképzelhető, hogy bizonyos időintervallumokban nincs is címkomponens. Három alapesete van:
	\begin{enumerate}[noitemsep]
		\item A programozó abszolút címet rendel a változóhoz, konkrétan megadja, hogy hol helyezkedjen el.
		\item Egy már korábban a tárban elhelyezett programozási eszköz címéhez képest mondja meg, hogy hol legyen a változó elhelyezve, vagyis relatív címet ad meg. Lehet, hogy a programozó az abszolút címet nem is ismeri.
		\item A programozó csak azt adja meg, hogy mely időpillanattól kezdve legyen az adott változónak   címkomponense, az elhelyezést a futtató rendszer végzi. A programozó nem ismeri az abszolút címet.
	\end{enumerate}
	Mindhárom esetben lennie kell olyan eszköznek, amivel a programozó megszüntetheti a címkomponenst.
\end{description}
A programozási nyelvek általában többféle címhozzárendelést ismernek, az eljárás-orientált nyelveknél általános a dinamikus tárkiosztás. A változók címkomponensével kapcsolatos a többszörös tárhivatkozás esete. Erről akkor beszélünk, amikor két különböző névvel, esetleg különböző attribútumokkal rendelkező változónak a futási idő egy adott pillanatában azonos a címkomponense ésígy értelemszerűen az értékkomponense is. Így ha az egyik  változó értékét módosítjuk, akkor a másiké is megváltozik. A korai nyelvekben (pl. FORTRAN, PL/Ő) erre explicit nyelvi eszközök álltak rendelkezésre, mert bizonyos problémák megoldása csak így volt lehetséges. A szituáció viszont előidézhető (akár véletlenül is) más nyelvekben is, és ez nem biztonságos kódhoz vezethet.

A változó értékkomponense mindig a címen elhelyezett bitkombinációként jelenik meg. A bitkombináció felépítését a típus által meghatározott reprezentáció dönti el.

Egy változó értékkomponensének meghatározására a következő lehetőségek állnak rendelkezésünkre:
\begin{description}
	\item[Értékadó utasítás] Az eljárás-orientált nyelvek leggyakoribb utasítása, az algoritmusok kódolásánál alapvető.
\end{description}

\subsubsection{Kifejezés}
A kifejezések szintaktikai eszközök. Arra valók, hogy a program egy adott pontján ott már ismert értékekből új értéket határozzunk meg. Két komponensük van, érték és típus. Egy kifejezés formálisan a következő összetevőkből áll:
\begin{description}
\item[Operandusok] az operandus literál, nevesített konstans, változó vagy függvényhívás lehet. Az értéket képviseli.
\item[Operátorok] Műveleti jelek. Az értékekkel végrehajtandó műveleteket határozzák meg.
\item[Kerek zárójelek] A műveletek végrehajtási sorrendjét befolyásolják. Minden nyelv megengedi a redundáns zárójelek alkalmazását.
\end{description}
Attól függően, hogy egy operátor hány operandussal végzi a műveletet, beszélünk \emph{egyoperandusú} (unáris), \emph{kétoperandusú} (bináris), vagy \emph{háromoperandusú} (ternáris) operátorokról. A kifejezésnek három alakja lehet attól függően, hogy kétoperandusú operátorok esetén az operandusok és az operátor sorrendje milyen. A lehetséges esetek:
\begin{description}
\item[prefix] az operátor az operandusok előtt áll (* 3 5)
\item[infix] az operátor az operandusok között áll (3 * 5)
\item[postfix] az operátor az operandusok mögött áll (3 5 *)
\end{description}
Az egyoperandusú operátorok általában az operandus előtt, ritkán mögötte állnak. A háromoperandusú operátorok általában infixek.
\paragraph{Kifejezés kiértékelése} Azt a folyamatot, amikor a kifejezés értéke és típusa meghatározódik, a kifejezés kiértékelésének nevezzük. A kiértékelés során adott sorrendben elvégezzük a műveleteket, előáll az érték, és hozzárendelődik a típus. A műveletek végrehajtási sorrendje a következő lehet:
\begin{itemize}[noitemsep]
	\item A műveletek felírási sorrendje, azaz balról-jobbra.
	\item A felírási sorrenddel ellentétesen, azaz jobbról-balra.
	\item Balról-jobbra a precedencia táblázat figyelembevételével.
\end{itemize}
Az infix alak nem egyértelmű. Az ilyen alakot használó nyelvekben az operátorok nem azonos erősségűek. Az ilyen nyelvek operátoraikat egy precedencia táblázatban adják meg. A precedencia táblázat sorokból áll, az egy sorban megadott operátorok azonos erősségűek (prioritásúak,precedenciájúak), az előrébb szereplők erősebbek. Minden sorban meg van adva még a kötési irány,
amely megmondja, hogy az adott sorban szereplő operátorokat milyen sorrendben kell kiértékelni, ha azok egymás mellett állnak egy kifejezésben. A kötési irány lehet balról jobbra, vagy jobbról balra.

A kifejezés típusának meghatározásánál kétféle elvet követnek a nyelvek. Vannak a típus-egyenértékűséget és vannak a típuskényszerítést vallók. A típus-egyenértékűséget valló nyelvek azt mondják, hogy egy kifejezésben egy kétoperandusú vagy háromoperandusú operátornak csak azonos típusú operandusai lehetnek. Ilyenkor nincs konverzió, az eredmény típusa vagy az operandusok közös
típusa, vagy azt az operátor dönti el (például hasonlító műveletek esetén az eredmény logikai típusú lesz). A különböző nyelvek szerint két programozási eszköz típusa azonos, ha azoknál fönnáll a:
\begin{description}
\item[deklaráció egyenértékűség] az adott eszközöket azonos deklarációs utasításban, együtt, azonos típusnévvel deklaráltuk.
\item[név egyenértékűség] az adott eszközöket azonos típusnévvel deklaráltuk
\item[struktúra egyenértékűség] a két eszköz összetett típusú és a két típus szerkezete megegyezik.
\end{description}
A típuskényszerítés elvét valló nyelvek esetén különböző típusú operandusai lehetnek az operátornak. A műveletek viszont csak az azonos belső ábrázolású operandusok között végezhetők el, tehát különböző típusú operandusok esetén konverzió van. Ilyen esetben a nyelv definiálja, hogy egy adott operátor esetén egyrészt milyen típuskombinációk megengedettek, másrészt, hogy mi lesz a művelet
eredményének a típusa. A kifejezés kiértékelésénél minden művelet elvégzése után eldől az adott részkifejezés típusa és az utoljára végrehajtott műveletnél pedig a kifejezés típusa. Egyes nyelvek (pl. Pascal, C) a numerikus típusoknál megengedik a típuskényszerítés egy speciális fajtáját még akkor is, ha egyébként a típus-egyenértékűséget vallják. Ezeknél a nyelveknél beszélünk a bővítés és szűkítés esetéről. A bővítés olyan típuskényszerítés, amikor a konvertálandó típus tartományának minden eleme egyben eleme a céltípus tartományának is (pl. egész $\rightarrow$ valós). Ekkor a konverzió minden további nélkül, értékvesztés nélkül végrehajtható. A szűkítés ennek a fordítottja (pl. valós $\rightarrow$ egész), ekkor a konverziónál értékcsonkítás, esetleg kerekítés történik. A nyelvek közül az ADA-ban semmiféle típuskeveredés nem lehet, a PL/I viszont a teljes konverzió híve.
A \emph{konstans kifejezés}\index{konstans kifejezés} olyan kifejezés, melynek értéke fordítási időben eldől, kiértékelését a fordító végzi. Operandusai literálok és nevesített konstansok lehetnek.

\subsection{Paraméterkiértékelés, paraméterátadás.}

\subsubsection{Paraméterkiértékelés}
alatt értjük azt a folyamatot, amikor egy alprogram hívásánál egymáshoz
rendelődnek a formális- és aktuális paraméterek, és meghatározódnak azok az információk, amelyek a paraméterátadásnál a kommunikációt szolgáltatják. A paraméterkiértékelésnél mindig a formális paraméter lista az elsődleges, ezt az alprogram specifikációja tartalmazza, egy darab van belőle. Aktuális paraméter lista viszont annyi lehet, ahányszor meghívjuk az alprogramot, ezeket rendeljük a formális paraméterlistához.

A formális és aktuális paraméterek egymáshoz rendelése történhet \emph{sorrendi kötés} vagy \emph{név szerinti kötés} szerint. \textbf{Sorrendi kötés}\index{Sorrendi kötés} esetén a formális paraméterekhez a felsorolás sorrendjében rendelődnek hozzá az aktuális paraméterek. Ezt minden nyelv ismeri, általában ez az alapértelmezés. \textbf{Név szerinti kötés}\index{Név szerinti kötés} esetén az aktuális paraméter listában határozhatjuk meg az egymáshoz rendelést, a formális paraméter nevét és mellette valamilyen szintaktikával az aktuális paramétert megadva. Ilyenkor lényegtelen a formális paraméterek sorrendje. Néhány nyelv ismeri. Alkalmazható a sorrendi és név szerinti kötés kombinációja együtt is, az aktuális paraméter lista elején sorrendi kötés, utána név szerinti kötés van.

Ha a formális paraméterek száma fix, a formális paraméter lista adott számú paramétert tartalmaz. Ekkor az aktuális paraméterek számának meg kell egyeznie a formális paraméterek számával, vagy lehet kevesebb, mint a formális paraméterek száma. Ez csak érték szerinti paraméterátadási mód esetén lehetséges. Azon formális paraméterekhez, amelyekhez nem tartozik aktuális paraméter, a formális paraméter listában alapértelmezett módon rendelődik érték. Ha a formális paraméterek száma tetszőleges, az aktuális paraméterek száma is tetszőleges. Létezik olyan megoldás is, hogy a paraméterek számára van alsó korlát.

A nyelvek egyik része a \emph{típusegyenértékűséget}\index{típusegyenértékűséget} vallja, ekkor az aktuális paraméter típusának azonosnak kell lennie a formális paraméter típusával. A nyelvek másik része a \emph{típuskényszerítés}\index{típuskényszerítés} alapján azt mondja, hogy az aktuális paraméter típusának konvertálhatónak kell lennie a formális paraméter típusára.

\subsubsection{Paraméterátadás}
A paraméterátadás az alprogramok és más programegységek közötti kommunikáció egy formája. A paraméterátadásnál mindig van egy hívó, ez tetszőleges programegység és egy hívott, amelyik mindig alprogram. Kérdés, hogy melyik irányban és milyen információ mozog. A nyelvek érték szerinti, cím szerinti, eredmény szerinti, érték-eredmény szerinti, név szerinti és szöveg szerinti paraméterátadási módokat ismernek.

\paragraph{Érték szerinti paraméterátadás} esetén a formális paramétereknek van címkomponensük a hívott alprogram területén. Az aktuális paraméternek rendelkeznie kell értékkomponenssel a hívó oldalon. Ez az érték meghatározódik a paraméterkiértékelés folyamán, majd átkerül a hívott alprogram területén lefoglalt címkomponensre. A formális paraméter kap egy kezdőértéket, és az alprogram ezzel az értékkel dolgozik a saját területén. Az információáramlás egyirányú, a hívótól a hívott felé irányul. A hívott alprogram semmit sem tud a hívóról, a saját területén dolgozik. Mindig van egy értékmásolás, és ez az érték tetszőleges bonyolultságú lehet. Ha egy egész adatcsoportot kell átmásolni, az hosszadalmas. Lényeges, hogy a két programegység egymástól függetlenül működik, és egymás működését az érték meghatározáson túl nem befolyásolják. Az aktuális paraméter kifejezés lehet.

\paragraph{Cím szerinti paraméterátadás} esetén a formális paramétereknek nincs címkomponensük a hívott alprogram területén. Az aktuális paraméternek viszont rendelkeznie kell címkomponenssel a hívó területén. Paraméterkiértékeléskor meghatározódik az aktuális paraméter címe és átadódik a hívott alprogramnak, ez lesz a formális paraméter címkomponense. Tehát a meghívott alprogram a hívó területén dolgozik. Az információátadás kétirányú, az alprogram a hívó területéről átvehet értéket, és írhat is oda, átnyúl a hívó területre. Időben gyors, mert nincs értékmásolás, de veszélyes lehet, mert a hívott hozzáfér a hívó területén lévő információkhoz. Az aktuális paraméter változó lehet.

\paragraph{Eredmény szerinti paraméterátadás} a formális paraméternek van címkomponense a hívott alprogram területén, az aktuális paraméternek pedig lennie kell címkomponensének. Paraméterkiértékeléskor meghatározódik az aktuális paraméter címe és átadódik a hívott alprogramnak, azonban az alprogram a saját területén dolgozik, és csak működésének befejeztekor másolja át a formális paraméter értékét erre a címre. A kommunikáció egyirányú, a hívottól a hívó felé irányul. Van értékmásolás. Az aktuális paraméter változó lehet.

\paragraph{Érték-eredmény szerinti paraméterátadás} esetén a formális paraméternek van címkomponense a hívott területén és az aktuális paraméternek rendelkeznie kell érték- és címkomponenssel. A paraméterkiértékelésnél meghatározódik az aktuális paraméter értéke és címe és mindkettő átkerül a hívotthoz. Az alprogram a kapott értékkel, mint kezdőértékkel kezd el dolgozni a saját területén és a címet nem használja. Miután viszont befejeződik, a formális paraméter értéke átmásolódik az aktuális paraméter címére. A kommunikáció kétirányú, kétszer van értékmásolás. Az aktuális paraméter változó lehet.

\paragraph{Név szerinti paraméterátadás} esetén az aktuális paraméter egy az adott szövegkörnyezetben értelmezhető tetszőleges szimbólumsorozat lehet. A paraméterkiértékelésnél rögzítődik az alprogram szövegkörnyezete, itt értelmezésre kerül az aktuális paraméter, majd a szimbólumsorozat a formális paraméter nevének minden előfordulását felülírja az alprogram szövegében és ezután fut le az. Az információáramlás iránya az aktuális paraméter adott szövegkörnyezetbeli értelmezésétől függ. 

\paragraph{Szöveg szerinti paraméterátadás} a név szerintinek egy változata, annyiban különbözik tőle, hogy a hívás után az alprogram elkezd működni, az aktuális paraméter értelmező szövegkörnyezetének rögzítése, a formális paraméter csak akkor íródik felül, amikor a formális paraméter neve először fordul elő az alprogram szövegében a végrehajtás folyamán.

Alprogramok esetén típust paraméterként átadni nem lehet. Egy adott esetben a paraméterátadás módját az alábbiak döntik el: a nyelv csak egyetlen paraméterátadási módot ismer (pl. C, Java), a formális paraméter listában explicit módon meg kell adni a paraméterátadási módot (pl. Ada), az aktuális és formális paraméter típusa együttesen dönti el, a formális paraméter típusa dönti el.
Az alprogramok formális paramétereit három csoportra oszthatjuk:
\begin{enumdescript}[noitemsep]
	\item[Input paraméterek] ezekkel az alprogram kap információt a hívótól (pl. érték szerinti).
	\item[Output paraméterek] a hívott alprogram ad információt a hívónak (pl. eredmény szerinti).
	\item[Input-output paraméterek] az információ mindkét irányba mozog (pl. érték-eredmény).
\end{enumdescript}

\subsection{Hatáskör, névterek, élettartam. }
A hatáskör a nevekhez kapcsolódó fogalom. Egy név hatásköre alatt értjük a program szövegének azon részét, ahol az adott név ugyanazt a programozási eszközt hivatkozza, tehát jelentése, felhasználási módja, jellemzői azonosak. A hatáskör szinonimája a láthatóság. A név hatásköre az eljárásorientált programnyelvekben a programegységekhez illetve a fordítási egységekhez kapcsolódik. Egy programegységben deklarált név a programegység lokális neve. A nem a programegységben deklarált, de ott hivatkozott név a szabad név. Azt a tevékenységet, mikor egy név hatáskörét megállapítjuk, hatáskörkezelésnek hívjuk. Kétféle hatáskörkezelést ismerünk, a statikus és a dinamikus hatáskörkezelést.

\subsubsection{Statikus hatáskörkezelés}
A statikus hatáskörkezelés fordítási időben történik, a fordítóprogram végzi. Alapja a programszöveg programegység szerkezete. Ha a fordító egy programegységben talál egy szabad nevet, akkor kilép a tartalmazó programegységbe, és megnézi, hogy a név ott lokális-e. Ha igen vége a folyamatnak, ha nem, akkor tovább lépked kifelé, amíg meg nem találja lokális névként, vagy el nem jut a legkülső szintre. Ha kiért a legkülső szintre, akkor vagy a mivel a programozónak kellett volna deklarálnia a nevet, ez fordítási hiba, vagy mivel ismeri az automatikus deklarációt a nyelv, a fordító hozzárendeli a névhez az attribútumokat. A név ilyenkor a legkülső szint lokális neveként értelmeződik.

Statikus hatáskörkezelés esetén egy lokális név hatásköre az a programegység, amelyben deklaráltuk és minden olyan programegység, amelyet ez az adott programegység tartalmaz, hacsak a tartalmazott programegységekben a nevet nem deklaráltuk újra. A hatáskör befelé terjed, kifelé soha. Egy programegység a lokális neveit bezárja a külvilág elől. Azt a nevet, amely egy adott programegységben nem lokális név, de onnan látható, globális névnek hívjuk. A globális név, lokálisnév relatív fogalmak. Ugyanaz a név az egyik programegység szempontjából lokális, egy másikban globális, egy harmadikban pedig nem is látszik.

\subsubsection{Dinamikus hatáskörkezelés}
A dinamikus hatáskörkezelés futási idejű tevékenység, a futtató rendszer végzi. Alapja a hívási lánc. Ha a futtató rendszer egy programegységben talál egy szabad nevet, akkor a hívási láncon keresztül kezd el visszalépkedni mindaddig, amíg meg nem találja lokális névként, vagy a hívási lánc elejére nem ér. Ez utóbbi esetben vagy futási hiba keletkezik, vagy automatikus deklaráció következik be. Dinamikus hatáskörkezelésnél egy név hatásköre az a programegység, amelyben deklaráltuk, és minden olyan programegység, amely ezen programegységből induló hívási láncban helyezkedik el, hacsak ott nem deklaráltuk újra a nevet. Újradeklarálás esetén a hívási lánc további elemeiben az újradeklarált eszköz látszik, nincs „lyuk a hatáskörben” szituáció.

Statikus hatáskörkezelés esetén a programban szereplő összes név hatásköre a forrásszöveg alapján egyértelműen megállapítható. Dinamikus hatáskörkezelésnél viszont a hatáskör futási időben változhat és más-más futásnál más-más lehet.

Az eljárásorientált nyelvek statikus hatáskörkezelést valósítanak meg. Általánosságban elmondható, hogy az alprogramok formális paraméterei az alprogram lokális eszközei, így neveik az alprogram lokális nevei. Viszont a programegységek neve a programegység számára globális. A kulcsszavak, mint nevek a program bármely pontjáról láthatók. A standard azonosítók, mint nevek azon programegységekből láthatók, ahol nem deklaráltuk újra őket. A globális változók az eljárásorientált nyelvekben a programegységek közötti kommunikációt szolgálják.

\paragraph{Névtér} tulajdonképpen egy csoport azon azonosítóknak (változók, konstansok (Math.PI, Math.E), függvény nevek (Math.Abs) stb) amik létezhetnek már egy másik fájlban, dokumentumban. Namespacek segítségével egyértelműen be tudjuk azonosítani, hogy melyik dokumentumban definiált azonosítót szeretnénk használni.

\subsection{Fordítási egységek, kivételkezelés.}
Az eljárásorientált nyelvekben a program közvetlenül fordítási egységekből épül föl. Ezek olyan forrásszöveg-részek, melyek önállóan, a program többi részétől fizikailag különválasztva fordíthatók le. Az egyes nyelvekben a fordítási egységek felépítése igen eltérő lehet. A fordítási egységek általában hatásköri és gyakran élettartam definiáló egységek is. A C\# fordítási egysége a névtér, ami a C forrásállományának felel meg, és hatásköri egység is.

\subsubsection{Kivételkezelés}
A kivételkezelési eszközrendszer azt teszi lehetővé, hogy az operációs rendszertől átvegyük a megszakítások kezelését, felhozzuk azt a program szintjére. A kivételek olyan események, amelyek megszakítást okoznak. A kivételkezelés az a tevékenység, amelyet a program végez, ha egy kivétel következik be. Kivételkezelő alatt egy olyan programrészt fogunk érteni, amely működésbe lép egy adott kivétel bekövetkezte után, reagálva az eseményre. A kivételkezelés az eseményvezérlés lehetőségét teszi lehetővé a programozásban. Operációs rendszer szinten lehetőség van bizonyos megszakítások maszkolására, ennek mintájára egyes kivételek figyelése letiltható vagy engedélyezhető. Egy kivétel figyelésének letiltása a legegyszerűbb kivételkezelés. Ekkor az esemény hatására a megszakítás bekövetkezik, feljön programszintre, kiváltódik a kivétel, de a program nem vesz róla tudomást, fut tovább. Természetesen nem tudjuk, hogy ennek milyen hatása lesz a program további működésére.

A kivételeknek általában van neve (egy kapcsolódó sztring, amely gyakran az eseményhez kapcsolódó üzenet szerepét játssza) és kódja (ami általában egy egész szám). A kivételkezelés a PL/I-ben jelenik meg és az Ada is rendelkezik vele. A két nyelv kétfajta kivételkezelési filozófiát vall. A PL/I azt mondja, hogy ha egy program futása folyamán bekövetkezik egy kivétel, akkor az azért van, mert a program által realizált algoritmust nem készítettük föl az adott esemény kezelésére, olyan szituáció következett be, amelyre speciális módon kell reagálni. Ekkor keressük meg az esemény bekövetkeztének az okát, szüntessük meg a speciális szituációt és térjünk vissza a program normál működéséhez, folytassuk a programot ott, ahol a kivétel kiváltódott. Az Ada szerint viszont, ha bekövetkezik a speciális szituáció, akkor hagyjuk ott az eredeti tevékenységet, végezzünk olyan tevékenységet, ami adekvát a bekövetkezett eseménnyel és ne térjünk vissza oda, ahol a kivétel kiváltódott. A kivételkezelési eszközrendszerrel kapcsolatban felmerülnek kérdések:
\begin{itemize}[noitemsep]
	\item Milyen beépített kivételek vannak?
	\item Definiálhatunk-e saját kivételt?
	\item Mik a kivételkezelés hatásköri szabályai?
	\item Hogyan folytatódik a futás a kivételkezelés után?
	\item Mi történik a kivételkezelőben történt kivétel esetén? 
	\item Van-e beépített kivételkezelő, illetve általános és parametrizált kivételkezelő?
\end{itemize}

Sem a PL/I-ben, sem az Adában nincs parametrizált és beépített kivételkezelő. Az Ada beépített kivételei általában eseménycsoportot neveznek meg. Alaphelyzetben minden kivétel figyelése engedélyezett, de egyes események figyelése (bizonyos ellenőrzések) letiltható. Saját kivétel az EXCEPTION attribútummal deklarálható. Kivételkezelő minden programegység törzsének végén, közvetlenül a záró END előtt helyezhető el, ebben WHEN-ágból tetszőleges számú megadható, de legalább egy kötelező. WHEN OTHERS ág viszont legfeljebb egyszer szerepelhet, és utolsóként kell megadni. Ez a nem nevesített kivételek kezelésére való (általános kivételkezelés). A kivételkezelő a teljes programegységben, továbbá az abból meghívott programegységekben látszik, ha azokban nem szerepel saját kivételkezelő. Tehát a kivételkezelő hatásköre az Adában dinamikus, hívási láncon öröklődik. Bármely kivételt explicit módon kiváltani a RAISE kivételnév; utasítással lehet. Programozói kivétel kiváltása csak így lehetséges.

Ha egy programegységben kiváltódik egy kivétel, akkor a futtató rendszer megvizsgálja, hogy az adott kivétel figyelése le van-e tiltva. Ha igen, akkor a program fut tovább, különben a programegység befejezi működését. Ezek után a futtató rendszer megnézi, hogy az adott programegységen belül van-e kivételkezelő. Ha van, akkor megnézi, hogy annak van-e olyan WHEN-ága, amelyben szerepel az adott kivétel neve. Ha van ilyen ág, akkor végrehajtja az ott megadott utasításokat. Ha ezen utasítások között szerepel a GOTO-utasítás, akkor a megadott címkén folytatódik a program. Ha nincs GOTO, akkor úgy folytatódik a program futása, mintha a programegység szabályosan fejeződött volna be. Ha a kivétel nincs nevesítve, megnézi, hogy van-e WHEN OTHERS ág. Ha van, akkor az ott megadott utasítások hajtódnak végre és a program ugyanúgy folytatódik mint az előbb. Ha nincs nevesítve a kivétel egyetlen ágban sem és nincs WHEN OTHERS ág, vagy egyáltalán nincs kivételkezelő, akkor az adott programegység továbbadja a kivételt. Ez azt jelenti, hogy a kivétel kiváltódik a hívás helyén, és a fenti folyamat ott kezdődik elölről. Tehát a hívási láncon visszafelé lépkedve keres megfelelő kivételkezelőt. Ha a hívási lánc elejére ér és ott sem talál kivételkezelőt, akkor a program a kivételt nem kezelte és a vezérlés átadódik az operációs rendszernek. Kivételkezelőben kiváltott kivétel azonnal továbbadódik. Csak a kivételkezelőben alkalmazható a RAISE; utasítás, amely újra kiváltja azt a kivételt, amely aktivizálta a kivételkezelőt. Ez viszont az adott kivétel azonnali továbbadását eredményezi. Deklarációs utasításban kiváltódott kivétel azonnal továbbadódik. Csomagban bárhol bekövetkezett és ott nem kezelt kivétel beágyazott csomag esetén továbbadódik a tartalmazó programegységnek, fordítási egység szintű csomagnál viszont a főprogram félbeszakad.

Az Ada fordító nem tudja ellenőrizni a kivételkezelők működését. Az Adában a saját kivételeknek alapvető szerepük van a programírásban, egyfajta kommunikációt tesznek lehetővé a programegységek között az eseményvezérlés révén.
%----------------------------------------------------------------------------
\section{Magas Sintű programozási nyelvek 2}
%----------------------------------------------------------------------------
\subsection{Speciális programnyelvi eszközök.}
Nem érdemel külön fejezetet

\subsection{Az objektumorientált programozás eszközei és jelentősége.}
Az objektumorientált (OO) paradigma középpontjában a programozási nyelvek absztrakciós szintjének növelése
áll. Ezáltal egyszerűbbé, könnyebbé válik a modellezés, a valós világ jobban leírható, a valós problémák
hatékonyabban oldhatók meg. Az OO szemlélet szerint az adatmodell és a funkcionális modell egymástól
elválaszthatatlan, külön nem kezelhető. A valós világot egyetlen modellel kell leírni és ebben kell kezelni a
statikus (adat) és a dinamikus (viselkedési) jellemzőket. Ez az egységbezárás elve.

\subsection{Funkcionális és logikai programozás.}

%----------------------------------------------------------------------------
\section{Adatszerkezetek és algoritmusok}
%----------------------------------------------------------------------------
\subsection{Adatszerkezetek reprezentációja.}

\subsection{Műveletek adatszerkezetekkel.}

\subsection{Adatszerkezetek osztályozása és jellemzésük. }

\subsection{Szekvenciális adatszerkezetek: sor, verem, lista, sztring.}

\subsection{Egyszerű és összetett állományszerkezetek.}

%----------------------------------------------------------------------------
\section{Adatbázisrendszerek}
%----------------------------------------------------------------------------
\subsection{Relációs, ER és objektumorientált modellek jellemzése.}

\subsection{Adatbázisrendszer.}

\subsection{Funkcionális függés.}

\subsection{Relációalgebra és relációkalkulus. }

\subsection{Az SQL.}

%----------------------------------------------------------------------------
\subsection{Hálózati architektúrák}
%----------------------------------------------------------------------------
\subsubsection{Az ISO OSI hivatkozási modell.}

\subsubsection{Ethernet szabványok.}

\subsubsection{A hálózati réteg forgalomirányító mechanizmusai. }

\subsubsection{Az internet hálózati protokollok, legfontosabb szabványok és szolgáltatások.}

%----------------------------------------------------------------------------
\section{Fizika 1}
%----------------------------------------------------------------------------
\subsection{Fizikai fogalmak, mennyiségek.}
\paragraph{Alapfogalmak}
\begin{description}[nosep]
	\item[Mérőszám] Megmutatja, hogy a mértékegységet hányszor lehet a mérendő mennyiségbe belefoglalni.
	\item[Mértékegység] Egy mérőszám típusát és nagyságrendjét meghatározó jelző.
	\item[Fizikai mennyiség] Adott fizikai jellemzőt leíró mérőszám és annak mértékegysége. Valamely jelenség, folyamat minőségileg megkülönböztethető, és mennyiségileg meghatározható tulajdonsága.
	\begin{enumdescript}[nosep]
		\item[Skalármennyiség] Olyan mennyiség, melynek nincs iránya, tehát teljes mértékben leír a nagysága (pl. hőmérséklet, tömeg, térfogat).
		\item[Vektormennyiség] Olyan mennyiség, melynek nagyságán kívül iránya is van (pl. erő, sebesség, térerősség).
	\end{enumdescript}
\end{description}

\paragraph{Erő}
\emph{Olyan hatás, ami egy tömeggel rendelkező testet gyorsulásra késztet.} Az \emph{eredő erő} a testre ható \emph{összes erő vektoriális összege}. $1N$ erő olyan hatás, amely egy $1kg$$  $ tömegű testet $1\frac{m}{s^2}$ mértékű gyorsulásra késztet. \underline{Vektormennyiség}. Iránya megegyezik a gyorsulás irányával. Néhány alapvető erő: elektromágneses erő, gravitációs erő, nukleáris erő.\\
Jele: $F$ (force) \quad $\vec{F} = m \cdot \vec{a}$\\
Mértékegysége: $N$ (Newton) \quad $ N = \nicefrac{kg\cdot m}{s^2} $

\paragraph{Tömeg}
\emph{A fizikai testek tulajdonsága, amely a tehetetlenségüket méri.} A súlytól eltérően a tömeg mindig ugyanaz marad, akárhová kerül is a hordozója. \underline{Skalármennyiség}.\\
Jele: $m$ (mass)\\
Mértékegysége: $kg$ (kilogramm)

Szigorúan véve három különböző dolgot neveznek tömegnek:
\begin{description}
	\item[Tehetetlen tömeg] a test tehetetlenségének mértéke: a rá ható erő mozgásállapot változtató hatásával szembeni ellenállás. A kis tehetetlen tömegű test sokkal gyorsabban változtatja mozgásállapotát, mint a nagy tehetetlen tömegű.
	\item[Passzív gravitáló tömeg] a test és a gravitációs tér kölcsönhatásának mértéke. Azonos gravitációs térben a kisebb passzív gravitáló tömegű testre kisebb erő hat, mint a nagyobbra.
	\item[Aktív gravitáló tömeg] a test által létrehozott gravitációs tér erősségének a mértéke. Például a Hold gyengébb gravitációs teret hoz létre, mint a Föld, mert a Holdnak kisebb az aktív gravitáló tömege.
\end{description}

\paragraph{Súly}
Az az \underline{erő}, amellyel a test az \textbf{alátámasztást nyomja} vagy a \textbf{felfüggesztést húzza}, tehát \emph{a test környezetére gyakorolt erőinek az eredője}. Azonos tömegű testek különböző erősségű gravitációs terekben különböző súlyúak, mert a testre ható gravitációs erőt a test közvetíti a környezetének.\\
Jele: $G$ \quad $\vec{G} = m \cdot \vec{g}$\\
Mértékegysége: $N$ (Newton)

\paragraph{Távolság}
Két pont közötti távolság az a legrövidebb út, melyen eljutunk egyik pontból a másikba, tehát a helyváltozás mértéke. A legrövidebb út általában egyenes mentén van. \underline{Skalármennyiség}.\\
Jele: $d$ (distance) vagy $l$ (length)\\
Mértékegysége: $m$ (méter)

\paragraph{Pálya}
Azt a vonalat, amin a test mozog, pályának nevezzük.

\paragraph{Út}
A mozgó test által befutott pályaszakasz hossza a megtett út.

\paragraph{Elmozdulás}
A kezdőpontból a végpontba mutató \underline{vektort} elmozdulásnak nevezzük.
Az út hossza nem lehet kisebb az elmozdulás nagyságánál, hiszen két pont között az egyenes szakasznak a legkisebb a hossza.

\paragraph{Idő}
A folyamatokban bekövetkező \emph{események sorrendiségének kifejezésére való skalármennyiség}. Az idő SI-alapegysége az SI-másodperc(?). Az ebből származó nagyobb időegységek, mint perc, óra és nap, nem SI-egységek, mert nem tízes számrendszerűek, és szükség van időnként szökőmásodpercre.\\
Jele: $t$ (time)\\
Mértékegysége: $s$ (szekundum vagy másodperc)

\paragraph{Nyomás}
Az egységnyi felületre ható erőhatást adja meg.\\
Jele: $p$ (pressure) \quad $p = \nicefrac{F}{A}$\\
Mértékegysége: $Pa$ (Pascal) \quad $Pa = \nicefrac{N}{m^2}$

\paragraph{Munka}
Amikor egy \emph{testre kifejtett erő hatására a test elmozdul, mechanikai munkavégzés történik}, ami \underline{arányos} a kifejtett \underline{erő nagyságával} és a \underline{megtett úttal}. A munka az erő és az elmozdulás \underline{skaláris szorzata}. Egy joule munkát végez az egy newton nagyságú erő a vele egyirányú egy méter hosszúságú elmozdulás közben. Ugyancsak egy joule az egy watt teljesítménnyel egy másodpercig végzett munka.\\
Jele: $W$ (work) \quad $W=F\cdot s \cdot \cos\alpha$\\
Mértékegysége: $J$ (Joule)	\quad $J = N \cdot m = kg \cdot \frac{m^2}{s^2}$

\paragraph{Teljesítmény}
A munkavégzés vagy energiaátvitel sebessége, más szóval az \emph{egységnyi idő alatt végzett munka}.\\
Jele: $P$ (power) \quad $P = \nicefrac{W}{t}$\\
Mértékegysége: $W$ (Watt) \quad $W = \nicefrac{J}{s}$

\paragraph{Sebesség}
Egy pontszerű test kitüntetett ponthoz viszonyított mozgásának jellemzésére szolgáló fizikai mennyiség. \textbf{Az út idő szerinti deriváltja}, tehát az \emph{időegység alatt bekövetkezett helyváltozás mértéke}. \underline{Vektormennyiség}.\\
Jele: $v$ (velocity) \quad $v = \nicefrac{s}{t}$\\
Mértékegysége: $\nicefrac{m}{s}$

\paragraph{Gyorsulás}
\textbf{A sebességvektor idő szerinti deriváltja}, tehát az \emph{időegység alatt bekövetkezett sebességváltozás mértéke}. \underline{Vektormennyiség}.\\
Jele: $a$ (acceleration) 
$$\overline{a}=\frac{\Delta v}{\Delta t} = \frac{v-u}{t}\quad \text{ahol v a végsebesség, u a kezdeti sebesség}$$
$$a=\lim\limits_{\Delta t \to 0} \frac{\Delta v}{\Delta t} = \frac{dv}{dt}$$
Mértékegysége: $\nicefrac{m}{s^2}$

\subsection{Impulzus, impulzusmomentum.}
\paragraph{Lendület (impulzus)}
Egy test mozgását leíró vektormennyiség. Nagysága arányos a tömeggel és a sebességgel. Lendületmegmaradás törvénye: zárt rendszer összes lendülete állandó. Nyugalomban lévő testnek nincs lendülete, a lendületet csakis külső erő változtathatja meg.\\
Jele: $I$ (impulzus)	$$\vec{I}=m\cdot \vec{v}$$
Mértékegysége: $kg\cdot \nicefrac{m}{s} = N \cdot s$

\paragraph{Perdület (impulzusmomentum)} Forgómozgásban lévő test lendülete által létrehozott nyomatékot jellemző vektormennyiség. Az erőkar és a lendület szorzata. Zárt rendszerben a lendületmegmaradás következtében a perdület állandó. Rögzített tengely körüli forgásnál a perdületet a tehetetlenségi nyomaték és a szögsebesség szorzatából számíthatjuk ki.
Jele: $L$	$$L=\Theta \cdot \omega$$
Mértékegysége: $kg\cdot \nicefrac{m^2}{s} = N \cdot m \cdot s$

\subsection{Newton törvényei.}
\begin{description}
	\item[I. Tehetetlenség törvénye] Minden inerciarendszerben vizsgált test nyugalomban marad vagy egyenes vonalú egyenletes mozgást végez mindaddig, míg ezt az állapotot egy másik test vagy erő hatása meg nem változtatja egy kölcsönhatás során.

	\item[II. Dinamika alaptörvénye] Egy pontszerű test gyorsulása azonos irányú a rá ható erővel, nagysága egyenesen arányos az erő nagyságával, és fordítottan arányos a test tömegével. $\vec{F}=m \cdot \vec{a}$

	\item[III. Hatás-ellenhatás törvénye] Két test kölcsönhatása során mindkét testre azonos nagyságú, azonos hatásvonalú és egymással ellentétes irányú erő hat.

	\item[IV. Szuperpozíció elve] Ha egy testre egy időpillanatban több erő hat, akkor ezek együttes hatása megegyezik a vektori eredőjük (vektoriális összegük) hatásának vonalával.
\end{description}

\subsection{Munkatétel.}
\begin{theorem}
	Egy test mozgási energiájának változásának mértéke megegyezik a testre ható összes erő munka előjeles összegével.
	$$\Delta E_m = \sum_{i=1}^{n}W_i$$
\end{theorem}

\subsection{A termodinamika I. és II. főtétel.}
\begin{theorem}[I. főtétel]
	A testek belső energiájának megváltozása egyenlő a testtel közölt hőmennyiség és a testen végzett munka előjeles összegével. Ez az energiamegmaradás törvénye, mert azt mondja ki, hogy külső beavatkozás nélkül nincs energiaváltozás.
	$$ \Delta E_b = Q+W $$
\end{theorem}
\begin{theorem}[II. főtétel]
	Termikus kölcsönhatással járó természetes folyamatoknál csak a nagyobb hőmérsékletű test képes a hőátadásra. Tehát egy elszigetelt rendszer állapota időben termikus egyensúly felé halad.
\end{theorem}

\subsection{A kinetikus gázmodell.}
A gázok tömeggel rendelkező részecskékből állnak, melyek energiaveszteség nélkül ütköznek egymással és a környezetükben lévő testekkel. Mozgási energiájuk csak a rendszer hőmérsékletétől függ. Állandóan mozgásban vannak, és az ütközések között egyenes vonalú egyenletes mozgást végeznek. Össztérfogatuk mindig jóval kisebb, mint a gázt tartalmazó tároló térfogata.
Alapegyenlet: $$p \cdot V = \frac{2}{3}N \cdot \frac{1}{2}mv^2$$ ahol p az ütközésekkel keltett nyomás, V a térfogat, N a molekulaszám, az $\nicefrac{1}{2}mv^2$ pedig egy molekula átlagos mozgási energiája.
%----------------------------------------------------------------------------
\subsection{Fizika 2}
%----------------------------------------------------------------------------
\subsubsection{Elektromos alapfogalmak és alapjelenségek.}

\subsubsection{Ohm-törvény.}

\subsubsection{A mágneses tér tulajdonságai. }

\subsubsection{Elektromágneses hullámok. }

\subsubsection{A Bohr-féle atommodell.}

\subsubsection{A radioaktív sugárzás alapvető tulajdonságai.}

%----------------------------------------------------------------------------
\section{Elektronika 1, 2}
{\footnotesize Passzív áramköri elemek tulajdonságai, RC és RLC hálózatok. Diszkrét félvezető eszközök, aktív áramköri elemek, alapkapcsolások. Integrált műveleti erősítők. Tápegységek. Mérőműszerek.}
%----------------------------------------------------------------------------
\subsection{Passzív áramköri elemek tulajdonságai, RC és RLC hálózatok.}
\paragraph{Ellenállás}
Jele: $R$\\
Kiszámítása: $R = \nicefrac{U}{I}$\\
Az ellenállás kapcsolata a teljesítménnyel: $P=\nicefrac{U^2}{R}\quad P=U \cdot I$\\
\begin{itemize}[nosep]
	\item Állandó értékű ellenállások
	\begin{itemize}[nosep]
		\item Felépítés: szigetelő hordozó, vezető réteg, fém kivezetések
		\item Főbb típusok: huzalellenállás, rétegellenállás, tömbellenállás
		\item Beszerelés: furatba, felületre (SMD)
		\item Értékét Ohm-ban [$\Omega$] adják meg  $\rightarrow R = \rho \cdot \nicefrac{l}{A}$ alapján
		\item Névleges érték, tűrés $\rightarrow$ nem tudják pontosan gyártani őket, ezért van egy tűréshatár \%-ban
		\item Terhelhetőség: Watt-ban, maximális teljesítménye; az ellenállás melegszik $\rightarrow$ hődisszipáció
		\item Ellenálláskódok $\rightarrow$ ellenálláson színkódok és számkódok
	\end{itemize}
	\item Változtatható ellenállások (potenciométerek)
	\begin{itemize}[nosep]
		\item Típusok: huzalpotenciométer, rétegpotenciométer
		\item Szabályozási jellemző: lineáris, nem lineáris (logaritmikus, fordított logaritmikus, S alakú)
		\item Terhelhetőség: a teljes névleges ellenállásra vonatkozik, az ebből számított áramot a csúszka egyik állásában sem haladhatja meg a potenciométer árama $$I_{\max} = \sqrt{\frac{P}{R_\text{névleges}}}$$
	\end{itemize}
	\item Speciális ellenállások (PTK, NTK, VDR)
\end{itemize}

\paragraph{Kondenzátor}
Jele: $C$\\
Kiszámítása: $Q = C \cdot U$\\
Mértékegysége: $\nicefrac{As}{V} =$ Farad
\begin{itemize}[nosep]
	\item Állandó kapacitású kondenzátorok
	\begin{itemize}[nosep]
		\item Felépítés: fém fegyverzetek, fém kivezetések, dielektrikum
		\item Főbb típusok: sík, hengeres, tekercselt, többrétegű
	\end{itemize}
	\item Változtatható kondenzátorok
	\begin{itemize}[nosep]
		\item Felépítés: mozgatható fegyverzetek, légrés (a fegyverzetek alakja határozza meg a szabályozási jelleget)
	\end{itemize}
\end{itemize}

\paragraph{Tekercs}
Jele: L\\
Kiszámítása: $B = \nicefrac{\Phi}{A}$\\
\begin{description}[nosep]
	\item[Indukció] a tekercsben feszültség jön létre, ha a tekercsen átmenő fluxus megváltozik.
	\item[Önindukció] feszültség indukálódik a tekercsben akkor is, ha a fluxus változását áramának megváltoztatásával saját maga idézte elő.
\end{description}
\textbf{A feszültség azért jön létre, mert megváltozik az áram folyásának iránya, mert ilyenkor a fluxus is megváltozik.}

\paragraph{Transzformátor}
Magyar találmány: Bláthy-Zipernowsky-Déry. Zárt vasmag, két oldal: primer és szekunder tekercs. $$ \frac{U_1}{U_2} = \frac{N_1}{N_2}$$
ahol $N$ a tekercs menetszáma.\\
Felhasználás: igény szerinti feszültség előállítás a 230V-os hálózati feszültségből és a villamos energia gazdaságos szállítása.

\subsubsection{RC és RLC hálózatok}
\paragraph{Soros RC hálózat} esetén az \emph{áramerősség} a közös mennyiség. A feszültségvektorok diagramjából impedancia háromszöget kapunk, melyből Z kiszámíthatjuk a $Z$ impedanciát:
$$Z^2 = R^2 + X_c^2 \rightarrow Z = \sqrt{R^2 + X_c^2}$$
$X_c$-vel a kondenzátor reaktanciája.
A kapcsolás kis frekvencián $X_c$ miatt szakadásként, nagy frekvencián vezetőként, ohmos ellenállásként viselkedik. Határfrekvenciának ($f_h$) nevezzük az áramkörben folyó váltakozó áram azon frekvenciáját, melynél $ R = X_c$.
\begin{figure}[h]
	\centering
	\begin{subfigure}[b]{0.45\textwidth}
		\includegraphics[width=\linewidth]{fig/10-serialRC_Schematic}
		\caption{Soros R-C kapcsolás és vektor diagramja}
		\label{fig:10-serialrcschematic}
	\end{subfigure}
	\begin{subfigure}[b]{0.45\textwidth}
		\includegraphics[width=\linewidth]{fig/10-serialRC_Plot}
		\caption{Soros R-C áramkör impedanciájának és fázisszögének változása}
		\label{fig:10-serialrcplot}
	\end{subfigure}
\end{figure}


\paragraph{Párhuzamos RC} hálózatnál a \emph{feszültség} a közös mennyiség. Az áramok vektorainak diagramjából admittancia (Y) háromszöget kapunk:
$$Y^2 = G^2 + B_c^2 \rightarrow Z = \sqrt{G^2 + B_c^2}$$
ahol $B_c = \nicefrac{I_c}{U}$ és $G=\nicefrac{I_R}{U}$.
A kapcsolás kis frekvencián $X_c$ miatt ohmos ellenállásként, nagy frekvencián rövidzárként ($0\Omega$ os ellenállásként) viselkedik. A határfrekvenciát ugyan úgy határozhatjuk meg, mint a soros RC hálózatok esetében.
\begin{figure}[h]
	\centering
	\begin{subfigure}[b]{0.45\textwidth}
		\centering
		\includegraphics[width=\linewidth]{fig/10-parallelRC_Schematic}
		\caption{Párhuzamos R-C kapcsolás és vektor diagramja}
		\label{fig:10-parallelrcschematic}
	\end{subfigure}
	\begin{subfigure}[b]{0.45\textwidth}
		\centering
		\includegraphics[width=\linewidth]{fig/10-parallelRC_Plot}
		\caption{Párhuzamos R-C áramkör impedanciájának és fázisszögének változása}
		\label{fig:10-parallelrcplot}
	\end{subfigure}
\end{figure}

\paragraph{Soros RLC} Másnéven \emph{sávzáró szűrő}. A soros kapcsolás miatt mindegyik elemen ugyanaz az I áram folyik át, tehát az \emph{áramerősség a közös mennyiség}. Az impedancia háromszög alalpján:
$$ Z = \sqrt{R^2 + (X_L - X_C)^2}$$
ahol $X_L$ és $X_C$ a tekrecs, ill. a kondenzátor reaktanciája.
\begin{figure}[h]
	\centering
	\includegraphics[width=0.5\linewidth]{fig/10-serialRLC}
	\caption{Soros R-L-C kapcsolás és vektor diagramja}
	\label{fig:10-serialrlc}
\end{figure}

\paragraph{Párhuzamos RLC} Másnéven \emph{sáváteresztő szűrő}. Itt a \emph{feszültség} a közös mennyiség. Minden áramot ($I_R, I_L, I_C$) a közös feszültséggel osztva admittancia háromszöget kapunk, melynek alapján:
$$ Y = \sqrt{G^2 + (B_L - B_C)^2}$$
\begin{figure}[h]
	\centering
	\includegraphics[width=0.5\linewidth]{fig/10-parallelRLC}
	\caption{Párhuzamos R-L-C kapcsolás és vektor diagramja}
	\label{fig:10-parallelrlc}
\end{figure}


\subsection{Diszkrét félvezető eszközök, aktív áramköri elemek, alapkapcsolások.}
\subsubsection{Dióda}
\emph{Egy P és Egy N réteget tartalmaz}. Jelölése kapcsolási rajzon: \includegraphics[width=0.2\linewidth]{fig/10-diode}

Működését egyszerűen jellemezhetjük: az egyik irányban engedi folyni az áramot, a másik irányban nem. A jelölés szerint a háromszög irányában folyhat az áram, visszafelé nem. Felhasználásának legáltalánosabb módja az \emph{egyenirányítás}.

A dióda legfőbb jellemzői:
\begin{itemize}[nosep]
	\item rajta átfolyó maximális áramerősség
	\item rajta eső feszültség 
	\item felhasználási terület
\end{itemize}
A diódák általában henger alakúak, két kivezetésük a henger két véglapján található, a rajta levő jelzés alapján megállapíthatjuk melyik vezeték a katód. A diódán az áram az anód irányából a katód irányába folyik (nyitó irányú kapcsolás). Záró irányú kapcsolás esetén elenyészően kicsi áram folyik keresztül a diódán.

A dióda működőképességét ellenállásmérő műszerrel ellenőrizhetjük a következőképpen: mindkét irányban megmérjük az ellenállását. Ha azt tapasztaljuk, hogy az egyik irányban mutat valamekkora ellenállást, a másikban pedig közel végtelen ellenállású akkor a dióda működőképes. A félvezetőkben a szabad töltéshordozók száma és anyag vezetőképessége a hőmérséklettel arányosan változik. Félvezető tulajdonsággal rendelkeznek az alábbi anyagok nagy tisztaságban: germánium (Ge), szilícium (Si), szelén (Se), valamint néhány vegyület: galliumarzenid (GaAs), indiumfoszfid (InP) stb. Nyitóirányú és záróirányú feszültség.(???)%TODO

\paragraph{Zener-dióda}
\includegraphics[width=0.2\linewidth]{fig/10-zener-diode}\\
Stabilizál. Adott nagyságú záróirányú feszültségnél hirtelen megnő a félvezető dióda árama. Konstrukciótól függően különböző letörési feszültségek vannak
\paragraph{Schottky dióda} nagyon gyorsan nyitó/záró dióda típus. Tápegységekben gyakran találkozunk velük.

\subsubsection{Alapkapcsolások}
\begin{theorem}[Kirchhoff I. törvénye]
	Egy csomópontba befolyó áramok összege megegyezik az onnan elfolyó áramok összegével. (csomóponti törvény) $$ \Sigma I = I_1 + I_2 + I_3 + \dots$$
\end{theorem}

\begin{theorem}[Kirchhoff II. törvénye]
	Bármely zárt hurokban az áramköri elemeken lévő feszültségek előjel helyesen vett összege nulla. (hurok törvény) $$ \Sigma U = 0 $$
\end{theorem}

\paragraph{Soros kapcsolás}
Soros kapcsolásban ugyanaz az áram folyik át minden ellenálláson, a feszültségek
összeadódnak. A sorosan kapcsolt ellenállások eredőjét az ellenállások összegzésével kapjuk, így az eredő nagyobb lesz bármely elem értékénél.
$$R_e = R_1 + R_2 + R_3\quad 
L_e = L_1 + L_2 + L_3\quad 
\frac{1}{C_e} = \frac{1}{C_1} + \frac{1}{C_2} + \frac{1}{C_3}$$

\paragraph{Párhuzamos kapcsolás}
Párhuzamos kapcsolásban azonos feszültség lép fel minden ellenálláson, az áramerősségek pedig összeadódnak. A párhuzamosan kapcsolt ellenállások eredőjét az ellenállások reciprokának összegével képezzük, ami még nem az eredőt,hanem annak a reciprokát adja ezért ennek is venni kell még a reciprokát.
$$\frac{1}{R_e} = \frac{1}{R_1} + \frac{1}{R_2} + \frac{1}{R_3}\quad \frac{1}{L_e} = \frac{1}{L_1} + \frac{1}{L_2} + \frac{1}{L_3}\quad 
C_e = C_1 + C_2 + C_3$$
%Párhuzamosan kapcsolt ellenállások, induktivitások, illetve sorosan kapcsolt kapacitások eredőjének kiszámításához használt speciális matematikai művelet a \emph{replusz}. Jele $\times$

\paragraph{Feszültségosztó}
Két ellenállás soros kapcsolása. A tápláló feszültség megoszlik az $R_1$ és $R_2$ ellenállás között. A feszültség az ellenállásokkal egyenes arányban oszlik meg.\\ %$$U_\text{ki}=\frac{R_2}{R_1 + R_2}$$
\includegraphics[width=0.25\linewidth]{fig/10-voltage_divider}

\paragraph{Wheatstone híd} A híd olyan négypólus, amelyben az áramköri elemek értékét úgy választjuk meg, hogy a kimeneti feszültség nulla legyen. Ezt nevezzük a híd kiegyenlített állapotának. A Wheatstone híd felhasználható ellenállás mérésre, mivel a szemben elhelyezkedő ellenállások szorzata = a másik két szemben lévő ellenállás szorzatával. Ha a négy ág közül három ismert, a negyedik kiszámítható. Kis áramok mérésére nem alkalmas.\\
%$$U_\text{ki} = 0\quad \text{ha}\quad R_1R_3 = R_2R_4 $$
\includegraphics[width=0.5\linewidth]{fig/10-wheatstone_bridge}

\subsection{Integrált műveleti erősítők.}
A műveleti erősítő kiváló minőségű differenciálerősítő integrált áramkör, amely egyenfeszültség erősítésére is alkalmas. Analóg számítás- és szabályzástechnikai alkalmazásokhoz fejlesztették ki, de igen sokoldalúan alkalmazzuk őket.\\
\includegraphics[width=0.25\linewidth]{fig/10-op_amp}\quad
Helyettesítő áramkör: \includegraphics[width=0.5\linewidth]{fig/10-op_amp1}\\

Ideális műveleti erősítő jellemzői:
\begin{itemize}
	\item $U_\text{ki} =A_\text{ol} \cdot (U_1 - U_2)$
	\item Végtelen nagy nyílt hurkú feszültségerősítés ($A_\text{ol} = \infty$)
	\item Végtelen nagy bemeneti ellenállás ($R_\text{be} = \infty$)
	\item Zéró kimeneti ellenállás ($R_\text{ki} = 0$)
	\item Végtelen nagy sávszélesség: minden frekvencián ugyanakkora az erősítése
	\item Zéró ofszet (tökéletesen szimmetrikus felépítés): ha a bemeneti feszültségek megegyeznek, akkor a kimeneti feszültség zéró.
\end{itemize}

Valódi műveleti erősítők jellemző értékei:
\begin{itemize}
	\item szimmetrikus tápfeszültségre van szüksége (tipikusan $\pm 15V$)
	\item $A_\text{ol} \approx 10^5 \text{--} 10^6$
	\item $R_\text{be} \approx 1 \text{--} 200 M\Omega$ bipoláris bemenet, $R_\text{be} \approx 1000 \text{--} 2000 M\Omega$ FET bemenet
	\item $R_\text{ki} \approx 10 \Omega$
	\item $f_\text{min} \approx 0 Hz, f_\text{max} \approx ~\text{MHz}$
	\item közös módusú elnyomási tényező: $\text{CMRR} \approx 90 \text{--} 100 \text{dB}$
	\item véges kimeneti feszültségtartomány: $-U_t < U_\text{ki} < +U_t$
	\item véges maximális jelváltozási sebesség (slew rate): $\text{SR} \approx 0.5 \text{--} 30 \nicefrac{V}{\mu s}$
\end{itemize}

\paragraph{Negatív visszacsatolás} A kimeneti jel egy részét visszavezetik és kivonják a bemeneti jelből, így erősítésre ténylegesen a bejövő jel és az adott hányadban visszacsatolt kimeneti jel különbsége kerül. Így a kimenet megváltoztatása a negatív visszacsatolás révén ellene hat az U’ különbség növekedésének. Állandó bemeneti jel esetén a kimeneti feszültség is stabil értékre áll be. a negatív visszacsatolást alkalmazó áramköröknél a műveleti erősítő két bemeneti feszültségének különbsége a nagy nyílthurkú erősítés miatt igen kicsi.\\
\includegraphics[width=0.5\linewidth]{fig/10-op_amp_neg_feedback}


\subsection{Tápegységek.}
\paragraph{Kapcsoló üzemű tápegységek} A korábban ismertetett tápegységek (hálózati transzformátor + egyenirányító + áteresztő tranzisztor) hatásfoka csak 25-50\%. Az áteresztő tranzisztor vesztesége jelentősen csökkenthető, ha kapcsolóval helyettesítjük. A transzformátor vesztesége (és mérete) pedig úgy csökkenthető, ha nagyfrekvenciás (20kHz-200kHz) váltakozó feszültséget transzformálnak.

A kapcsoló üzemű tápegységekben (switch-mode power supply (SMPS)) a feszültségszabályozást egy teljesítménytranzisztor változtatható kitöltési tényezővel történő egymást követő bekapcsolásával (teljes telítési állapot) és kikapcsolásával (lezárt állapot) valósítják meg. A terhelésnek megfelelő kitöltési tényezőjű gyorsan ismétlődő (10kHz\textendash100kHz) be- és kikapcsolás a szűrés után a kívánt nagyságú egyenfeszültséget biztosítja a kimeneten

A kapcsolóüzemű (feszültség csökkentő) tápegység működési elve: A K kapcsolót ciklikusan (T periódusidővel) kapcsolgatják: T\textsubscript{be} ideig az 1-es, T\textsubscript{ki} ideig a 2-es potícióba kapcsolják ($T=T be +T_\text{ki}$).

\subsection{Mérőműszerek.}
\begin{description}
	\item[Feszültségmérő] igen nagy belső ellenállású mérőműszer. Párhuzamosan kapcsolandó a mérendő alkatrésszel.
	\item[Áramerősségmérő] igen kis belső ellenállású mérőműszer, az áramkörbe sorosan kapcsolandó.
	\item[Digitális multiméter] Elsősorban egyenfeszültség, egyenáram, ellenállás és kapacitás mérésére használatos. Váltóáram és váltófeszültség effektív értékének mérésére is használható, de csak kisebb frekvenciákon.
	\item[Analóg multiméterek] elektronikus elven működő mérőműszerek,villamos mennyiségek mérésére alkalmasak, a mért mennyiség kijelzése analóg (pl. mutató) műszerrel történik.
	\item[Digitális multiméterek] többfunkciós mérőműszerek, a mért mennyiség számjegyes kijelzővel történik, ehhez mindegyik tartalmaz egy beépített A/D átalakítót és számkijelzőt.
	\item[Oszcilloszkóp] Mind feszültség, mind váltófeszültség mérésére alkalmas. A hagyományos analóg oszcilloszkóp képes megjeleníteni egy periodikus jel időbeli változását és így alkalmas a váltakozófeszültség olyan paramétereinek mérésére, mint pl. a periódusidő, felfutási idő, amplitúdó. Csak periodikus jeleket képesek stabil képpel megjeleníteni. A tárolófunkciós analóg oszcilloszkópok és a digitális oszcilloszkópok egyedi impulzusok megjelenítésére is alkalmasak.
	\item[Analóg oszcilloszkópok] Villamos jelek vizuális vizsgálatát teszi lehetővé egy katódsugárcsöves kijelző segítségével.
	\item[Digitális oszcilloszkópok] Villamos jelek vizuális vizsgálatát teszi lehetővé egy folyadékkristályos kijelző segítségével.
	\item[Szkópméterek] egy kétsugaras digitális tárolóoszcilloszkóp és egy digitális multiméter kombinációja, a mérési funkciók automatikusan a legjobb üzemi állapotra állítódnak be.
\end{description}
%----------------------------------------------------------------------------
\section{Digitális Technika}
{\footnotesize Logikai függvények kapcsolástechnikai megvalósítása. Digitális áramköri családok jellemzői (TTL, CMOS, NMOS). Különböző áramköri családok csatlakoztatása. Kombinációs és szekvenciális hálózatok. A/D és D/A átalakítók.}
%----------------------------------------------------------------------------
\subsection{Logikai függvények kapcsolástechnikai megvalósítása.}
Egy-egy alapáramkör megvalósítására több áramkörtechnikai megoldás született, ezek a teljesítményfelvételben, tápfeszültségigényben, H (high) és L (low) szint feszültségben, sebességben és a kimeneti terhelhetőségben térnek el egymástól. Az áramkörcsaládok helyes megválasztásához ismernünk kell a belső felépítésüket és a bemeneti-kimeneti terhelhetőségüket is. Egy logikai kapu minden bemeneti állapotához meghatározott kimeneti állapot tartozik. A logikai kapu által megvalósított függvény nem egyértelmű, ha a szintállapot és a logikai állapot közötti kapcsolat nincs tisztázva. Ez az összerendelés lehet: pozitív logika (H=1,L=0) és lehet negatív logika (H=0,L=1) ,attól függően, hogy azt az áramkörcsaládot használjuk, amelyikkel egyszerűbb a kapcsolás. Ha negatív logikára térünk át a függvényeket a következőképpen kel megcserélnünk:\\
Nem-vagy $\leftrightarrow$ Nem-és , vagy $\leftrightarrow$ és , nem $\leftrightarrow$ nem\\ 
A fizikai megvalósításukhoz: kapcsolókat,komparátorokat,tranzisztorokat használunk. A gyakorlatban szinte minden logikai kaput Nem-és kapukkal realizálnak, olcsósága miatt.
Egy (kombinációs) logikai áramkör tervezésénél miután felírtuk a függvényt boolean-algebrai alakban, azt egyszerűsítjük. Ezután átírjuk a függvényt úgy, hogy a független változókból képzett maxtermek logikai összege (Diszjunktív normálforma) vagy mintermek logikai szorzata (Konjunktív normálforma) legyen. Ezt az átírást megkönnyíti a Karnaugh--tábla alkalmazása. Az ilyen kanonikus alakra hozott függvény végül a de-morgan azonosságokat felhasználva átírható csak NAND műveleteket tartalmazó formulára.

Négynél több független logikai változó esetén a kézi egyszerűsítés már nem elég hatékony ezért ilyenkor számítógépes algoritmusokat alkalmazunk: Quine-McCluskey (kimerítő teljes algoritmus, melynek hosszú a futásideje, ezért csak közepesen sok változószámig alkalmazható), illetve az Espresso (heurisztikus algoritmus, amit szinte minden szintézisprogram használ) algoritmusok.



\subsection{Digitális áramköri családok jellemzői(TTL, CMOS, NMOS).}
\subsubsection{TTL --- Tranzisztor-tranzisztor logika}
Legnagyobb típus-választékú, univerzális célra készülő bipoláris integrált áramkör rendszer. A bemenetet egy többemitteres tranzisztor szolgáltatja. Kis sebességű, így nagyobb a késleltetés az egyes kapuknál. A fogyasztása a többi családhoz képest elég nagy, de az órajel emelkedésével csak kis mértékben emelkedik és az elektromos kisülések ellen is kellően védett. 

Kis és közepes bonyolultságú integrált áramkörökben a legelterjedtebb bipoláris áramkör. A logikai áramkörökben a kapcsolási és terjedési-késleltetési idők szabják meg a sebességet, a kapunkénti disszipált teljesítmény (fogyasztás) is fontos korlátot szab a rendszernek.
Típusai:
\begin{itemize}[nosep]
	\item H (high speed)
	\item S (schottky-diódás)
	\item L (low power)
	\item LS (low power schottky)
\end{itemize}
\begin{tabular}{|c|c|c|}
	\hline 
	TTL család & Egy kapura eső fogyasztás (mW) & Terjedési-késleltetési idő (ns) \\ 
	\hline 
	normál & 20 & 10 \\ 
	\hline 
	H TTL & 30 & 6 \\ 
	\hline 
	S TTL & 20 & 3 \\ 
	\hline 
	L TTL & 2 & 35 \\ 
	\hline 
	LS TTL & 2 & 15 \\ 
	\hline 
\end{tabular} 
\paragraph{TTL nem-és kapu felépítése és működése:}~\\
U\textsubscript{t} = +5V, L szint: 0V -- 0,8V, H szint: 2,4V -- 5V\\
\includegraphics[width=0.5\linewidth]{fig/11-TTL_NAND_schema}
Ha a Q\textsubscript{1} tranzisztor bármelyík emitter kivezetésére 0 logikai szint kerül, a kimeneten 1-es logikai szint lesz a Q\textsubscript{4} tranzisztor és az R\textsubscript{2} ellenállás miatt. Ha a Q\textsubscript{1} tranzisztor mindkét emitter kivezetésére 1 logikai szint kerül, akkor a Q\textsubscript{2} tranzisztor kinyit, a a Q\textsubscript{4} tranzisztor lezár és a Q\textsubscript{3} tranzisztor nyitott állapota miatt alacsony logikai szintre kerül a kimenet.

\subsubsection{MOS --- Metal-Oxid-Semiconductor}
A MOS áramkörök bemenetei érzékenyek a túlfeszültséggel szemben, a megengedettnél nagyobb GATE feszültség miatt átüthet a GATE alatti vékony oxidréteg. Ennél a családnál is tranzisztorokat használnak, de a bipoláris tranzisztorok helyett a fémoxid-félvezetőből készült térvezérlésű FET, vagyis Field-effect tranzisztorokat. Ezáltal sokkal nagyobb műveleti sebességet tudnak elérni.

\subsubsection{N-MOS --- N csatornás MOS (térvezérlésű) tranzisztor}
Csak N-csatornás MOSFET-eket használ, ezáltal csak a magas jelszintet tudja stabilan megjeleníteni. Kisebb a zavarvédettsége és nagyobb a fogyasztása, de a TTL családhoz képest hasonló a CMOS családhoz, csak egyszerűbb kialakítású, ezért olcsóbb is.

Felületigénye jóval kisebb, mint a TTL-nek, gyártási technológiája is gyorsabb (kevesebb művelet), jelentősen kisebb a teljesítményfelvétele. Az N-csatornás tranzisztorok működésében részt vevő Negatív töltéshordozók mozgékonysága majdnem 3-szor nagyobb,mint a (régi technikája miatt kiszorított) P-csatornás tranzisztorok Pozitív töltéshordozóinak mozgékonysága, ezért az N-MOS áramkörök kisebb Gate-kapacitása miatt nagyobb sebességre képesek, csökken a tranzisztor U TO küszöbfeszültsége is, amely alacsonyabb tápfeszültség alkalmazását teszi lehetővé,emiatt könnyebb az N-MOS tranzisztorokat könnyű illeszteni a TTL áramkörökhöz.

\paragraph{N-MOS nem-és kapu felépítése és működése:}~\\
\includegraphics[width=0.5\linewidth]{fig/11-NMOS_NAND_schema1}\\
A T\textsubscript{1} tranzisztort R\textsubscript{D} (100K\textOmega) ellenállás helyett alkalmazzák a térkihasználás és a nagyobb drain ellenállás miatt. Az Y kimeneten csak akkor jelenik meg alacsony logikai színt, ha a T\textsubscript{2} és T\textsubscript{3} tranzisztor is vezet, azaz A-ra és B-re is logikai magas szintet rakunk.

A MOS kapu egyenáramú bemeneti ellenállása nagyon nagy értékű. A bemeneti feszültség változása a gate-kapacitást töltő és kisütő áramot hoz létre. Ez a rövid idejű áramimpulzus nagyobb, mint a szivárgási áram. Ennek ellenére úgy lehet venni, hogy a MOS tranzisztor nem terheli le az előző kapu kimenetét. N-MOS-ok terjedési-késleltetési ideje 15 ns körüli, a teljesítményfelvételük pár szár \textmu W-ig emelkedik.

\subsubsection{C-MOS --- Komplementer MOS tranzisztor}
A működési elve az, hogy N- és P-csatornás MOSFET tranzisztorokat is alkalmaz a logika megvalósításához. A nagy sebesség mellett a fogyasztása is sokkal kisebb a TTL családhoz képest, és a zavarvédettsége is nagyon jó, továbbá széles a működési tápfeszültség tartománya. Egyetlen nagy hátránya, hogy a frekvencia emelkedésével nő a fogyasztása. Működése: egyik helyzetben a felső, P-csatornás tranzisztor nyitott és a kimenetet a pozitív tápfeszültséggel köti össze. A másik helyzetben az alsó, N-csatornás tranzisztor nyit ki és a kimenetet a 0V-tal köti össze. Tehát alacsony jelre a P-MOSFET, magas jelre az N-MOSFET nyit.

A C-MOS-t P-csatornás és N-csatornás növekményes MOS tranzisztorok alkotják.
Jellegzetességei a rendkívülien kis áramfogyasztás, széles működési tápfeszültség-tartomány és a nagy zavarvédettség. A C-MOS áramkörök alapeleme az inverter.

\paragraph{C-MOS nem-és kapu felépítése és működése:}~\\
\includegraphics[width=0.5\linewidth]{fig/11-CMOS_NAND_schema}\\
Alacsony logikai bemeneti fesztültségnél az N-csatornás Q\textsubscript{n} tranzisztor lezárt állapotban van és a P-csatornás Q\textsubscript{p} tranzisztor nyitott állapotban így a kimenet logikai magas szinten lesz (megközelítőleg V\textsubscript{DD} értékű). Ha a bemenetre logikai magas szint kerül, akkor Q\textsubscript{n} tranzisztor kinyit és Q\textsubscript{p} tranzisztor lezár, ezáltal logikai alacsony szint kerül a kimenetre. Ha a működési frekvencia megnövekszik, vele együtt nő a teljesítményfelvétel is a tranzisztorok kapcsolási ideje miatt kialakuló tápáramfogyasztás miatt. 

A C-MOS áramkörök tápfeszültsége +3 és +15 V közötti értéket vehet fel. Egy nem-és kapu terjedési késleltetési ideje átlagosan 25 ns, a nyugalmi teljesítmény: 50 nW

A C-MOS áramkörök nagy jelentőségű változata az SOS (Silicon on Sapphire). Szilícium helyett zafír hordozóra alakítják a a C-MOS-okat. Az áramköri elemek között a szigetelési ellenállás nagyon nagy értékű,ezáltal csökken a hordozók kapacitása és nagyságrendekkel nő a sebesség is (1-2 ns). Jelentősen csökken az áramkör nyugalmi teljesítménye is.

\subsection{Különböző áramköri családok csatlakoztatása.}
A logikai függvények fizikai megvalósításához a gyártó cégek bizonyos alapáramkör választékot (családokat) alakítanak ki. Az áramkörök összekapcsolását a katalógusokban található jellemző adataik közlése és egyeztetése teszi lehetővé. A digitális technikában a pozitív logika terjedt el, és az áramkörök pozitív feszültségrendszerben dolgoznak. Amikor
csatlakoztatjuk a különböző áramkörcsaládokat, a következő jellemzőket kell megvizsgálni:
\begin{enumdescript}
	\item[Tápfeszültség] az áramkör működéséhez szükséges feszültség
	\item[Logikai szintek] U\textsubscript{Hmin} és U\textsubscript{Hmax} valamint U\textsubscript{Lmin} és U\textsubscript{Lmax} jellemző értékek
	\item[Zajtartalék, zajérzékenység] feszültségingadozás, ami még nem változtatja meg az állapotot
	\item[Bementeti terhelhetőség] egységterhelés, mind alacsony mind magas logikai szinten
	\item[Kimenteti terhelhetőség] károsodás nélkül a kimenetet képes hajtani
	\item[Teljesítményfelvétel] teljesítményigény 50\%-os terhelésnél
	\item[Jelkésleltetési idő] a kimenet reagál a bemenet hatására, amihez idő kell
\end{enumdescript}

\subsection{Kombinációs és szekvenciális hálózatok. A/D és D/A átalakítók.}
\paragraph{Kombinációs hálózatok} \emph{időfüggetlen} logikai függvényeket valósítanak meg, memória nélküli logikai áramkörök, a kimeneti logikai változókat az adott időpontban megjelenő bemeneti logikai változók határozzák meg, az áramköröket IC-kel (Integrated circuit) valósítják meg, a logikai alapfüggvényeket megvalósító áramköröket logikai kapuknak nevezzük, egy IC-n belül több logikai kapu is található.

\paragraph{Szekvenciális (sorrendi) hálózatok} \emph{időfüggő} logikai függvényeket valósítanak meg, a kimeneti események alakulását a pillanatnyi bemeneti feltételek mellett a korábbi időpillanatokban bekövetkezett kimeneti események is befolyásolják, attól függően, hogy az állapotváltozás hogyan következik be két csoportot különbözetünk meg:
\begin{description}
	\item[Aszinkron] a kimenet előző állapotától való függést visszacsatolással vagy tárolókkal valósítják meg, a kimenet a bemenetre azonnal reagál
	\item[Szinkron] az állapotváltozás a kimeneten egy engedélyező jel (Clock) hatására, azzal azonos fázisban zajlik le, a kimenet előző állapotától való függést tárolókkal oldják meg.
\end{description}

Kombinációs és szekvenciális hálózatok típusai:
\begin{itemize}[nosep]
	\item R-S tároló
	\item Inverz R-S tároló
	\item J-K tároló
	\item T tároló
	\item D tároló
\end{itemize}

\paragraph{A/D átalakítók}
Feladatuk, hogy a bemenetre érkező „A” analóg jelnek megfelelő „D” digitális jelet (bináris számot) állítson elő a kimeneten, a működéshez szükséges egy R referencia (ált. egy U R referenciafeszültség), melyhez a konverter az A analóg mennyiségét viszonyítja és amely a kimeneti feszültség maximális értékét is meghatározza. Az A/D átalakító kvantál (viszonyítási tartományt használ) a digitális jelek előállításához. A digitalizáláshoz elemi lépcsőket kell használni, és minden lépcsőhöz egy digitális mennyiséget (bináris szám) kell rendelni. Az analóg jel annál pontosabban ábrázolható minél kisebb egy elemi lépcső, vagy kvantum nagysága. Az A/D átalakítóknak annál nagyobb a felbontóképessége minél több bit áll rendelkezésre az ábrázoláshoz. Az átalakítók felbontásának növelése az áramköri megvalósítást nehezíti, drágítja. A kis pontosság-igény esetén elég a 8 bit-es (256 elemi lépcső), nagyobb pontosságot biztosítanak a 10,12,14 bites átalakítók és nagy pontosságú rendszerekben 16,18,20 biteseket alkalmaznak.

\paragraph{D/A átalakítók}
Feladatuk, hogy a bemenetre érkező „D” digitális jelnek megfelelő „A” analóg jelet (általában feszültséget vagy áramot) állítson elő a kimeneten, működéséhez szükséges egy U\textsubscript{R} referenciafeszültség (nagyon pontos feszültségforrás), amelyből a kimeneti feszültséget származtatjuk és ez határozza meg a kimeneti feszültség maximális értékét (végkitérését) is. A digitális technikában többnyire bináris alakban állnak rendelkezésre, valamilyen meghatározott kódban kifejezve, ezt a kódot a D/A átalakítónak ismerni kell, csak ezeket a meghatározott bináris kódokat tudják analóg jellé alakítani. Az ideális D/A átalakítók kimeneti jele egyenesen arányos a bemenetükre digitálisan adott szám értékével. A pontosságot itt is az elemi lépcsők száma határozza meg, minél kisebb egy ilyen lépcső, annál pontosabb az átalakítás. Vannak soros és párhuzamos működésű átalakítók és megkülönböztetünk közvetlen és közvetett elvű átalakítókat.

\section[Infokommunikációs hálózatok]{Infokommunikációs hálózatok specializáció}
%----------------------------------------------------------------------------
\section{Távközlő hálózatok}
{\footnotesize Fizikai jelátviteli közegek. Forráskódolás, csatornakódolás és moduláció. Csatornafelosztás és multiplexelési technikák. Vezetékes és a mobil távközlő hálózatok. Műholdas kommunikáció és helymeghatározás.}
%----------------------------------------------------------------------------
\subsection{Fizikai jelátviteli közegek.}

\subsection{Forráskódolás, csatornakódolás és moduláció.}

\subsection{Csatornafelosztás és multiplexelési technikák.}

\subsection{Vezetékes és a mobil távközlő hálózatok.}

\subsection{Műholdas kommunikáció és helymeghatározás.}


%----------------------------------------------------------------------------
\subsection{Hálózatok hatékonyságanalízise}
%----------------------------------------------------------------------------
\subsubsection{Markov-láncok, születési-kihalási folyamatok.}

\subsubsection{A legalapvetőbb sorbanállási rendszerek vizsgálata.}

\subsubsection{A rendszerjellemzők meghatározásának módszerei, meghatározásuk számítógépes támogatása.}

%----------------------------------------------------------------------------
\section{Adatbiztonság}
{\footnotesize Fizikai, ügyviteli és algoritmusos adatvédelem, az informatikai biztonság szabályozása. Kriptográfiai alapfogalmak. Klasszikus titkosító módszerek. Digitális aláírás, a DSA protokoll.}
%----------------------------------------------------------------------------
\subsection{Fizikai, ügyviteli és algoritmusos adatvédelem, az informatikai biztonság szabályozása.}
\begin{definition}[Adatvédelem]
	azon fizikai, ügyviteli és algoritmikus eszközök együttes felhasználását értjük, amelyek segítségével a véletlen adatvesztések és szándékos adatrongálódások és információ kiszivárogtatások megelőzhetők, vagy jelentős mértékben megnehezíthetők
\end{definition}
\paragraph{Fizikai adatvédelem} két lényegi dolgot takar: Egyrészt biztosítani kell az optimális, de legalább a még elfogadható \textbf{üzemi körülményeket} (hőmérséklet, páratartalom, por, tartalék alkatrészek stb.), másrészt pedig a szükséges \textbf{vagyonvédelmi intézkedésekről} sem szabad megfeledkezni. Például: Villamos hálózat helyes kialakítása; Szünetmentes tápegységek használata; Megfelelő szerverterem kialakítása(klimatizálás, füstérzékelés, árnyékolás,\dots); Megfelelő adattároló eszközfajták használata; Betörésvédelem.

\paragraph{Ügyviteli adatvédelem} a folyamatok szabályozásának, a szabályzatoknak a kialakítása és védelme. A fizikai adatvédelem önmagában ugyanis nem elegendő. Példa: hiába zárjuk be a szerverszoba ajtaját, ha a portás beengedi azt, aki egy szerszámos táskával érkezvén arról tájékoztatja, hogy ’zsírozni kell a switcheket’ (social hacking/engineering). Tehát szükséges \textbf{pontosan szabályozni}, hogy \textbf{ki}, \text{mikor}, \textbf{mit} és \textbf{hogyan} tehet meg, illetve nem tehet meg. Szükség van \textbf{informatikai biztonsági szabályzatra} is, amely mindezt egységes módon áttekinti. Megfelelő felhasználó menedzselési rendet kell kialakítani, hogy a felhasználók, hozzáférési jogosultságaik, munkájukból adódó szerepköreik kezelése összhangba hozható legyen. Példák: Feladat- és jogkörök szétválasztása; Hozzáférések és tevékenységek regisztrálása; Személyazonosítás; Hatáskörök és felelősségek szétválasztása vagy átlapolása.

\paragraph{Algoritmikus adatvédelem} feladata olyan programok és eljárások alkalmazása, amelyek segítik az előző két terület feladatait és létrehozzák azokat a számítógépes védelmi funkciókat, amik ezen a területen meggátolják az adatokhoz való illetéktelen hozzáférést és módosítást. Példák: Hálózati azonosítás; \textbf{Titkosítás}; Behatolásvédelem; Automatikus adatmentés; Többforrásos adattárolás.

\paragraph{Az informatikai biztonsági szabályrendszer szükségessége:} (1) az adatok egyre inkább elektronikus formában jelennek meg; (2) a Szervezetek informatika nélkül működésképtelenek; (3) az informatikai függőség egyre nagyobb; (4) ugyanakkor a fenyegetettség is egyre növekszik; (5) az üzletfolytonossághoz kritikus fontosságú; (6) a kárpotenciál és a kockázati tényezők szervezetenként eltérőek lehetnek!

\subsubsection{Biztonsági célok}
Alapkövetelmények, amelyek teljesülése az üzemszerű használhatóság előfeltétele:
\begin{enumerate}
	\item rendelkezésre állás (elérhetőség az arra jogosultak számára)
	\item sértetlenség (valódiság)
	\item bizalmasság (jellegtől függően)
	\item nyomon követhetőség, hitelesség
	\item Biztosítékok (az információs rendszer teljességére nézve)
\end{enumerate}
Ez alapján úgy lehet meghatározni az Informatikai Biztonság fogalmát, hogy az akkor áll fenn, ha az információs rendszer védelme az alapkövetelmények szempontjából
\begin{itemize}
	\item \textbf{zárt:} minden fontos fenyegetést figyelembe vesz
	\item \textbf{teljes körű:} a rendszer összes elemére kiterjedő
	\item \textbf{folyamatos:} megszakítás nélküli, az időben változó körülmények ellenére is
	\item \textbf{kockázatarányos:} a feltehető kárérték és a kár valószínűségének szorzata nem haladhat meg egy előre rögzített küszöböt, amely egy üzleti döntés.
\end{itemize}

\subsection{Kriptográfiai alapfogalmak.}
\begin{center}
	\includegraphics[width=0.7\linewidth]{fig/14-Crypto_scheme}
\end{center}
\begin{definition}[titkostási rendszer]
	A $(\mathcal{P}, \mathcal{C}, \mathcal{K},Enc,Dec,Key)$ hatost titkosítási rendszernek (sémának) nevezzük, ahol:
	\begin{itemize}[nosep]
		\item $\mathcal{P}, \mathcal{C}, \mathcal{K}$ a nyílt és titkos üzenetek, valamint lehetséges kulcsok véges, nemüres halmaza ($1<|\mathcal{P}|,|\mathcal{C}|,|\mathcal{K}| < \infty$).
		\item $Enc:\; \mathcal{K}\times\mathcal{P} \to c\;$ egy titkosító függvény ($c \in \mathcal{C}$: titkosított üzenet)
		\item $Dec:\; \mathcal{K}\times\mathcal{C} \to m\;$ egy visszafejtő függvény ($m \in \mathcal{P}$: nyílt üzenet)
		\item $Key \subseteq \mathcal{K}\times\mathcal{K}$ kulcspárok halmaza
	\end{itemize}
\end{definition}
\begin{definition}[Korrekt visszafejtés]
	A titkosító rendszer korrekt visszafejtést biztosít, ha minden $(k_E,k_D)\in\mathcal{K},\; m\in\mathcal{P}$ esetén
	$$Dec(k_D,Enc(k_E,m)) = m$$
\end{definition}
\begin{note}~\\
	\begin{itemize}[nosep]
		\item $(k_E,k_D)$ titkosító-visszafejtő kulcspár
		\item $c = Enc(k_E,m)$
		\item definíció alapján, rögzített $(k_E,k_D)$ esetén az $Enc_{k_E}(m) = Enc(k_E,m)$ függvény injektív, azaz:
		$$Enc_{k_E}(m_1) = Enc_{k_E}(m_2) \Leftrightarrow m_1 = m_2$$
	\end{itemize}
\end{note}
\begin{definition}[Teljes titkosítási rendszer]
	A titkosítási rendszert teljesnek nevezzük, ha minden $(k_E,k_D)$ esetén az $Enc_{k_E}(m)$ függvény szürjektív, azaz minden $c\in\mathcal{C}$ titkos üzenet esetén létezik $m\in\mathcal{P}$ nyílt üzenet, amelyikre $Enc_{k_E}(m) = c$. Ebben az esetben $|\mathcal{P}| = |\mathcal{C}|$
\end{definition}
Természetes elvárás egy titkosítási rendszerrel szemben, hogy ha vannak $k_{E1} \neq k_{E2}$ kulcsok, akkor $Enc_{k_{E1}}(m) \neq Enc_{k_{E2}(m)}$. Továbbá ha $(k_E,k_{D1}), (k_E,k_{D2})\in Key$, akkor $k_{D1} = k_{D2}$.
\begin{definition}[Teljes kulcstér]
	Legyen $|\mathcal{P}| = |\mathcal{C}|$ . A kulcsteret teljesnek nevezzük, ha minden $f:\;\mathcal{P} \to \mathcal{C}$ bijektív leképezéshez létezik $k_E \in \mathcal{K}$, amelyikre $Enc_{k_E} = f$.
\end{definition}
\begin{description}[nosep]
	\item[Kriptográfia] A kulcs alapú titkosítási technikák tudománya
	\item[Kriptanalízis] A kulcs alapú titkosítási technikák feltörésének, támadásának tudománya
	\item[Kriptológia] Kriptográfia + Kriptanalízis
	\item[Nyílt szöveg (plain text)] Az eredeti, mindenki által értelmezhető információ
	\item[Titkosított szöveg (ciphertext)] A titkosított információ
	\item[Titkosítás (encryption)] Eljárás, mely során az információt a birtokosa titkossá nyilvánít.
	\item[Visszafejtés (decryption)] A titkosított információból az eredeti visszaállítása.
	\item[Rejtjelezés] Adott módszer a nyílt szöveg kódolására (és visszafejtésére)
	\item[Kulcs (key)] Az az információ, amelynek segítségével a rejtjelezés történik. Két típus: nyilvános és titkos kulcs.
	\item[Szimmetrikus titkos. rendszer] Ha van $(k_E,k_D)$ kulcspárunk, $k_E$ nyílvános kulcs megegyezik a $k_D$ titkos kulccsal vagy $k_D$ polinomiális időben kiszámolható $k_E$-ből.
	\item[Aszimmetrikus titkos. rendszer] A tetszőlegesen választott $k_E$ és $k_D$ kulcspárok annyira különböznek, hogy nem létezik polinomiális időbonyolultságú algoritmus, mely $k_E$-ből kiszámolja $k_D$-t
\end{description}

\subsection{Klasszikus titkosító módszerek.}
A számítógép megjelenése előtt (before computers, BC) használt rendszereket, történelmi vagy klasszikus titkosítási rendszereknek nevezzük. Nem egyértelműen definiált, de nagyjából a számítógéppel könnyen támadható rendszereket nevezzük így. Jellemzően élő nyelvi szöveg titkosítására használták. Főbb típusai:
\begin{easylist}[enumerate]
	# Helyettesítéses titk. rendszerek: Az üzenet egy betűjét más betűre vagy szimbólumra cseréli
	## monoalfabetikus: a csere transzformációja változatlan
	## polialfabetikus: az egymás után következő betűket más-más transzformációval titkosítjuk
	# Permutációs titk. rendszerek: az üzenet betűit más sorrendben írjuk fel (mint egy anagramma, de a titkos szöveg nem feltétlenül értelmes)
\end{easylist}
\paragraph{Betűk kicserélése absztrakt szimbólumokra} Helyettesítéses, monoalfabetikus
\paragraph{Caesar--titkosítás} Monoalfabetikus, helyettesítéses.\\
$Enc_k(m) = k+m (mod\; 26)$\\
$Dec_k(c) = c-k (mod\; 26)$
\paragraph{Affin titkosítás} Monoalfabetikus\\
$k_E = (\alpha,\beta)\; k_D = (\alpha^{-1},\beta)\quad lnko(\alpha,26) = 1$\\
$Enc = \alpha \cdot m + \beta (mod\; 26)$\\
$Dec = \alpha^{-1} \cdot (c - \beta) (mod\; 26)$
\paragraph{Vigènre-titkosítás} polialfabetikus, a kulcs valamilyen karaktersorozat. A Vernam--rejtjelező (OTP) hasonló hozzá, azonban az kulcs hossza megegyezik az üzenettel. Bináris esetben a Vernam--rejtjelező az üzeneten és a kulcson bitenkénti XOR műveletet végrehajtva állítja elő a titkos üzenetet.\\
$c[i] = m[i] + k_E[i\; mod\; n] (mod\; 26)$\\
$m[i]= c[i]- k_E[i\; mod\; n] (mod\; 26)$

\paragraph{Enigma} Polialfabetikus. Elektromechanikus rejtjelező, ahol a kulcsot a gépben található tárcsák kezdőpozíciója jelentette.

\subsection{Modern titkosítási rendszerek}
\subsubsection{Folyamtitkosítás}
Bitsorozat titkosítása bitenként. Legegyszerűbb módszer a One-Time-Pad titkosítás, ahol titkosított bitsorozat, az eredeti sorozat egy álvéletlen bitsorozattal történő XOR művelet eredménye $c[i] = m[i]XORk[i]$. Shannon bebizonyította, hogy egy teljesen véletlen kulcsfolyam esetén a titkos üzenet elméletileg feltörhetetlen. Tehát ennek a titkosításnak az erőssége az álvéletlenszám-generátor erősségén múlik. Alkalmazásakor a mesterkulcsot a kulcsgenerátor kezdeti állapota jelenti. A kulcsfolyam-generálásnak több variánsa is létezik:
\paragraph{Lineárisan visszacsatolt léptető regiszter -- LFSR}
A kulcsfolyam $l$ bitjein egy $a = (a_0,a_1,\dots,a_{l-1})$ értékkel bitenkénti ÉS műveletet hajtunk végre, majd a részeredményen XOR műveletet hajtunk végre. Így kapjuk meg a $k_i$ kulcsot:
$$ k_i = a_{l-1}k_{i-1}\oplus a_{l-2}k_{i-2}\oplus\dots\oplus a_{0}k_{i-l}$$
\begin{center}
	\includegraphics[width=0.7\linewidth]{fig/14-LFSR}
\end{center}
Az $a_i$ és kezdeti $k_i$ értékei adják a generátor kezdeti állapotát. Fontos, hogy ezeket az értékeket jól válasszuk meg, mert ezen múlik az álvéletlen sorozat minősége. Ez leginkább két dolgot takar: sorozat eloszlása (fontos, hogy egyenletes legyen) és a periódus hossza (minél hosszabb annál jobb).
\paragraph{Lineárisan rekurzív sorozat -- LSR}
Az LFSR általánosítása. C/C++ rand() függvénye implementálja.\\
$a = 31835, b = 1906, k_0 = 41$\\
$k_i = a \dots k_{i-1} + b (mod 2^{15})$\\
periódusa $2^{15}$
\paragraph{Blum--Blum--Shub generátor -- BBS}
$p, q$ nem közeli nagy prímek: $m = p\times q$ és $a_0$ úgy választjuk, hogy $lnko(a_0,m)=1$.\\
$a_i = a_{i-1}^2 \;(mod\; n)$
$k_i = a_i(mod 2)$ vagy például $a_i$ paritása.\\
A következő $a_i$ az előzőekből exponenciális időben számolható ki. Periódusa kb. $p\cdot q$, ha $lnko(p-1,q-1)$ kicsi, ezért kell, hogy p és q ne legyenek egymáshoz közel.
\paragraph{RC4}
\begin{multicols}{2}
Egy S kezdeti tömb feltöltése és összekeverése után a következő bitet az  %\ref{fig:14-rc4}~
ábrán látható módon kapjuk meg, (i = (i + 1) mod 256 j = j + S i mod 256).
Felhasználja az SSL/TLS, és a WEP protokoll.
\begin{center}
	\includegraphics[width=\linewidth]{fig/14-RC4}
%	\caption{A következő elem (K) meghatározása az S tömbből. Minden kiolvasás előtt S[i] és S[j] értéket felcseréljük a tömbben}
%	\label{fig:14-rc4}
\end{center}
\end{multicols}

\subsubsection{DES}
\begin{multicols}{2}
\begin{easylist}[itemize]
# 1975-ben jelent meg, a LUCIFER titkosítási rendszer egyik változata.
# Ma már nem létezik szabványként, mert feltörték, helyette a Triple DES-t ajánlott használni. 
# Szimmetrikus, blokktitkosítási rendszer, a blokk mérete 64 bit. Ha ennél nagyobb méretű üzenetet kell titkosítani, akkor azt fel kell tördelni valamilyen blokktitkosítási módszerrel.
# Matematikai alapja a Fiestel struktúra.
# A Kulcs 64 bites, amelyből 56 bit random és 8 a random bitek alapján meghatározott. 
# S-BOX-ok használata a titkosítás és a visszafejtés során.
# 16 körből áll az algoritmus, ezért 16 körkulcs kerül legenerálásra az eredeti kulcsból. Visszafejtéskor a körkulcsokat fordított sorrendben kell alkalmazni.
# Jellemzők:
## az S-BOX-okat kivéve minden lépése az algoritmusnak lineáris
## az S-BOX-ok eredete nem ismert, de a feltételeknek megfelelnek.
## nagy problémája a kis kulcstér, ezért kell inkább TDES-t alkalmazni ma már.
\end{easylist}
\end{multicols}
\begin{figure}[h]
	\centering
	\includegraphics[width=0.2\linewidth]{fig/14-DES_Feistel_net}
	\includegraphics[width=0.7\linewidth]{fig/14-DES_Feistel_func}
	\caption{A feistel-háló és a feistel függvény (F - Feistel függvény; IP,FP - egymással ellentétes permutációk; E - kiterjesztő függvény; S - S-boxok; P - permutáció;sk - a mesterkulcsból képzett 16 részkulcs)}
	\label{fig:14-desfeistelnet}
\end{figure}

\subsubsection{AES}
\begin{multicols}{2}
\begin{easylist}[itemize]
# Szimmetrikus, blokktitkosítási rendszer. Kifejlesztői: RIJNDAEL (Vincent Rijmen, Joan Daemen).
# A (T)DES-nél hatékonyabb titkosításra képes, mivel matematikai alapja nem Fiestel struktúra, itt a bitek nem csak keverednek, de meg is változnak!
# 128, 192, 256 bites blokkhossz és kulcshossz kezelésére képes bármilyen kombinációban.
# Bonyolult matematikai háttere van, a bájtokat {0,1} együtthatós polinomként kezeli.
# Tervezésekor fontos volt az ismert támadásokkal szembeni védelem, a gyorsaság és tömör kódolhatóság
# A belső algoritmus ciklusainak száma a blokkhossz és a kulcshossz függvényében lehet 10, 12, 14
# Visszafejtésnél az összes transzformáció inverzét kell végrehajtani.
\end{easylist}
Működése 3 fő részből és 4 műveletből áll:
\begin{easylist}[enumerate]
# Előkészítés:
## AddRoundKey
# Ismételt rész:
## SubBytes
## ShiftRows
## MixColumns
## AddRoundKey
# Utófeldolgozás:
## SubBytes
## ShiftRows
## AddRoundKey
\end{easylist}
\begin{description}[nosep]
	\item[AddRoundKey] XOR művelet az állapotmátrix és a megfelelő sorszámú kulcs között
	\item[SubBytes] speciális S-boxokat használ, mely egy nemlineáris helyettesítési kódolást valósít meg
	\item[ShiftRows] a sorokat ciklikusan eltolják. Az 1. sor 0, a 2. sor 1,\dots, 4. sor 3 hellyel
	\item[MixColumns] oszloponként egy lineáris transzformáció:
	$\begin{matrix}
	2 & 3 & 1 & 1\\
	1 & 2 & 3 & 1\\
	1 & 1 & 2 & 3\\
	3 & 1 & 1 & 2
	\end{matrix}$
\end{description}
\end{multicols}

\subsubsection{RSA}
\begin{easylist}[itemize]
# Feltalálói: Ron Rivest, Adi Shamir, Leonard Adleman, 1977-ben.
# Az első aszimmetrikus, PKI-t használó kriptográfiai rendszer. (NEM BLOKKTITKOSÍTÓ RENDSZER)
# A kommunikációban résztvevő mindkét félnek rendelkeznie kell egy Publikus és egy Titkos kulccsal (PK; SK), amelyeket a kommunikáció titkosságának biztosítása érdekében felhasználnak.
# Kulcsgeneráló algoritmus, PK=(n,e) és SK=(n,d)
## Véletlenül választunk két nagy prímet: p,q
## Kiszámoljuk: n=p*q
## Kiszámoljuk: \textphi(n)=(p-1)*(q-1)
## Választunk véletlen egy e-t, hogy 1 < e < \textphi(n) és (e,\textphi(n)) = 1
## Kiszámítjuk d-t, hogy 1 < d < \textphi(n) és e*d= 1 (mod \textphi(n))
# A PK és SK ismeretében a titkosító algoritmus: 	c = ENC\textsubscript{PK}(m) = m\textsuperscript{e} (mod n)
# A PK és SK ismeretében a visszafejtő algoritmus: 	m = DEC\textsubscript{SK}(c) = c\textsuperscript{d} (mod n)
# Az RSA feltörésének nehézségét a prím-faktorizáció nehézsége adja:  \{c,n,e\} $\leftarrow$ \{m\}  nehéz!
# p és q választásánál fontos, hogy hosszuk bitekben hasonló legyen, viszont ne legyenek túl közeliek
# e választásánál, annál jobb minél kisebb, általában 65537 szokott lenni.
\end{easylist}

\subsection{Digitális aláírás, a DSA protokoll.}
A digitális aláírás lényege:
\begin{easylist}[enumerate]
# Letagadhatatlanság --- alkalmas az aláíró azonosítására 
# Hitelesítés --- más nem tudja létrehozni 
# Hamisíthatatlanság --- akár az aláírás, akár a dokumentum módosul, észrevehető.
\end{easylist}
A letagadhatatlanság és harmadik fél számára is elfogadható hitelesítés
alapvetően kétféle megoldással lehetséges:
\begin{easylist}[itemize]
# megbízható harmadik féllel (Trusted Third Party, TTP) (szigorúan véve nem DS)
# közvetlen digitális aláírás (nyilvános kulcsú titkosítással)
\end{easylist}
Digitális aláírás fajtái:
\begin{easylist}[itemize]
# RSA alapú --- Az aláíró a titkos kulcsával titkosítja az üzenetét vagy annak kivonatát. A hitelességet az adja, hogy egyedül az aláíró nyilvános kulcsa tudja visszafejteni a dokumentumot (vagy a kivonatot). Hátrányai:
## nem megbízható --> egzisztenciális hamisítás: egy tetszőleges s értéket választva előállíthatjuk az $ m \equiv s^e (n) $ üzenetet.
## alakíthatóság --> $(m_1,s_{m_1}),(m_2,s_{m_2})$ létező aláírásokból előállítható az $(m_1m_2,s_{m_1}s_{m_2})$ új aláírás
## Ezek a hátrányok nem jelentkeznek, ha az üzenet helyett annak kivonatát írjuk alá (hash-and-sign):\\
$S_m\equiv Hash(m)^d (n) \Rightarrow (m,S_m)$\\
ellenőrzés: $Hash(m) \equiv S_m^d (n)$
# ElGamal alapú --- Hátránya, hogy az aláírás kétszer akkora, mint a dokumentum
# DSA protokoll --- Digital Signature Algorithm. A digitális aláírás szabvány (DSS) felhasználja.
\end{easylist}

\subsubsection{DSA}
\begin{multicols}{2}
Kulcsgenerálás:
\begin{enumerate}[nosep]
	\item $p, q$ olyan prímek, hogy $q|p-1$ (p-1 q többszöröse)
	\item meghatározzuk $g$, melynek rendje $q$ moduló $p$, azaz $g^t \equiv 1 (p) \Leftrightarrow t= k\cdot q$
	\item x egy véletlen szám és $a \equiv g^x (p)$
	\item PK=(p,q,g,a), SK=(x)
\end{enumerate}
Aláírás:
\begin{enumerate}[nosep]
	\item $k$ véletlen szám kiválasztása
	\item $r \equiv (g^k\;(mod\;p))\; (q)$
	\item $t \equiv k^{-1}(Hash(m)+xr)\; (q)$
	\item $s_m = (r,t)$
\end{enumerate}
Ellenőrzés:
\begin{enumerate}[nosep]
	\item $v_1 \equiv Hash(m)\cdot t^{-1}\; (q)$
	\item $v_2 \equiv r\cdot t^{-1}\; (q)$
	\item Az aláírás rendben van, ha $r \equiv (q^{v_1}a^{v_2}\;(mod\;p))\;(q)$
\end{enumerate}
\end{multicols}
Az üzenet ellátható időbélyegzővel is. Az üzenet kivonatát átadjuk egy időbélyegző szolgáltatónak, aki hozzáfűz egy T időpontot és így aláírja a kivonatot. Ezután a bélyegzett üzenet így néz ki $(m,(m|T,s_{m|T}))$. Az ellenőrző biztosan tudja, hogy T időpillanatban már létezett a dokumentum.
%----------------------------------------------------------------------------
\section{A RIP protokoll működése és paramétereinek beállítása (konfigurációja).}
%----------------------------------------------------------------------------

%----------------------------------------------------------------------------
\section{Bevezetés a Cisco eszközök programozásába 1}
{\footnotesize A forgalomszűrés, forgalomszabályozás (Trafficfiltering, ACL) céljai és beállítása (konfigurációja) egy választott példa alapján.}
%----------------------------------------------------------------------------
\subsection{A forgalomszűrés, forgalomszabályozás (Trafficfiltering, ACL) céljai és beállítása (konfigurációja) egy választott példa alapján.}
A rendszergazdáknak meg kell találniuk annak a módját, hogy megakadályozzák a hálózat
jogosulatlan elérését, miközben a belső felhasználók számára lehetővé teszik a szükséges
szolgáltatások használatát. Bár a biztonsági eszközök, mint például a jelszavak, a visszahívó
berendezések és a fizikai biztonsági eszközök sok segítséget nyújtanak, gyakran nem
rendelkeznek az alapvető forgalomszűréshez szükséges rugalmassággal és a rendszergazdák
igényeinek megfelelő vezérlési lehetőségekkel. Előfordulhat például a hálózati rendszergazda
lehetővé kívánja tenni a felhasználók számára az internet elérését, de nem akarja, hogy külső
felhasználók telnettel hozzáférhessenek a LAN-hoz.

A forgalomirányítók alapvető forgalomszűrési lehetőségeket biztosítanak, például hozzáférési
listákat (access control list, ACL) az internetes forgalom letiltásához. Az ACL engedélyező és
tiltó utasítások sorozata, melyek címekre vagy felsőbb rétegbeli protokollokra alkalmazhatók.
Ebben a modulban a normál és kiterjesztett ACL-ek használatáról, mint a hálózati forgalom
vezérlési módszeréről fogunk tanulni, továbbá arról, hogyan használhatók az ACL-ek
valamilyen biztonsági megoldás részeként.

Ezenkívül a fejezetben tippek, elgondolások, javaslatok és általános útmutatások is
szerepelnek az ACL-ek használatával, valamint a létrehozásukhoz szükséges parancsokkal és
beállításokkal kapcsolatban. Végül a fejezet példákat mutat a normál és a kiterjesztett ACL-
ekre és a forgalomirányítók interfészein történő alkalmazásukra.

Az ACL-ek akár egyetlen sorból is állhatnak, ha a cél egy adott állomásról származó
csomagok továbbításának engedélyezése, de tartalmazhatják szabályok és feltételek rendkívül
összetett halmazát is, amelyek pontosan megadják az engedélyezett forgalom jellegét, illetve
befolyásolják a forgalomirányító folyamatok teljesítményét.

\subsubsection{A hozzáférési listák működésének alapelvei}
Az ACL-ek feltétellisták, amelyek a forgalomirányító interfészén keresztülhaladó forgalomra
vonatkozóan lépnek érvénybe. Ezek a listák írják elő a forgalomirányító számára, hogy a
csomagokat fogadja el vagy utasítsa vissza. Az elfogadás és a visszautasítás meghatározott
feltételek alapján is történhet. Az ACL-ek segítségével lehetővé válik a forgalom felügyelete,
valamint a hálózatról kiinduló és az oda befutó elérések biztonságossá tétele.

ACL-ek minden irányított protokollhoz létrehozhatók, többek közt az internetprotokollhoz
(IP) és a hálózatközi csomagcseréhez (IPX) is. Az ACL-ek a forgalomirányítón konfigurálva
adott hálózat vagy alhálózat elérhetőségének vezérlésére használhatók.

Az ACL-ek úgy szűrik a hálózati forgalmat, hogy az irányított csomagok továbbítását vagy
eldobását írják elő a forgalomirányító interfészein. A forgalomirányító minden csomagot
megvizsgál annak meghatározásához, hogy azt az ACL-ben meghatározott feltételek szerint
továbbítani vagy eldobni kell. Az ACL-ek a forrás- és a célcímekre, a protokollokra és a
felsőbb rétegbeli portszámokra nézve adnak meg feltételeket.

Az ACL-eket protokollonként, irányonként és portonként kell létrehozni. Ha szabályozni
szeretnénk egy adott interfész forgalmát, akkor a rajta engedélyezett protokollok
mindegyikéhez külön ACL-t kell készíteni. Egy ACL egy interfészen egyszerre csak egy
irány forgalmát szabályozza. Minden irányhoz külön ACL-t kell létrehozni, egyet a kimenő,
egyet pedig a bejövő forgalomhoz. Végül minden interfészen több protokoll és irány is
definiálható. Ha például egy forgalomirányítónak két interfésze van, és ezeken IP, AppleTalk
és IPX alapú forgalom zajlik, akkor 12 külön ACL-t kell rajta létrehozni: minden
protokollhoz egy-egy ACL, szorozva kettővel a kimenő és a bejövő forgalom miatt, szorozva
kettővel, vagyis a portok számával.

ACL-eket többek között a következő okok miatt szokás létrehozni:
\begin{enumerate}[nosep]
	\item Korlátozható a hálózat forgalma, és növelhető a teljesítménye.
	\item Biztosítják a forgalom szabályozását.
	\item Az útvonalfrissítések továbbítása is korlátozható. Ha a hálózati környezet miatt nincs szükség a frissítésekre, sávszélességet lehet megtakarítani.
	\item Alapszintű hálózati hozzáférés-szabályozást biztosítanak. Az ACL-ek engedélyezhetik például a hálózat egy részének elérését egy állomás számára, és megtilthatják egy másiknak. Lehetséges például, hogy az A állomás hozzáférhet a személyzeti osztály hálózatához, míg a B állomásnak ezt megtiltjuk.
	\item Eldönthetjük, hogy milyen típusú forgalom továbbítódjon és milyen törlődjön a forgalomirányító interfészein. Például engedélyezhetjük az elektronikus levelek továbbítását, de a telnetet tilthatjuk.
	\item A rendszergazdák szabályozhatják, hogy az ügyfelek a hálózat mely részeihez férhetnek hozzá.
	\item Ki lehet választani bizonyos állomásokat, és számukra engedélyezni vagy megtiltani a hálózat egy részének elérését.
	\item A felhasználók számára engedélyezni vagy tiltani lehet bizonyos szolgáltatások, például az FTP vagy a HTTP elérését.
\end{enumerate}

Ha egy forgalomirányítón nincsenek ACL-ek megadva, akkor a forgalomirányítóba befutó csomagok mindegyike továbbhaladhat a hálózat bármely része felé.
\begin{figure}[h]
	\centering
	\begin{subfigure}{0.45\linewidth}
		\includegraphics[width=\linewidth]{fig/16-ACL}
		\caption{}
	\end{subfigure}
	\begin{subfigure}{0.45\linewidth}
		\includegraphics[width=\linewidth]{fig/16-ACL_list-schema}
		\caption{}
	\end{subfigure}
	\caption{}
\end{figure}

\subsubsection{ACL-ek létrehozása}
Az ACL-ek létrehozása globális konfigurációs módban történik.
Számos különböző típusú ACL létezik, így normál, kiterjesztett, IPX, AppleTalk stb. ACL.
Amikor egy forgalomirányítón ACL-eket konfigurálunk, akkor egy-egy szám
hozzárendelésével mindegyiket egyértelműen azonosítanunk kell. A szám megadja az ACL
típusát; ebből következően az adott típushoz tartozó értéktartományba kell esnie:\\
\begin{tabular}{l|l}
	\hline 
	\rule[-1ex]{0pt}{2.5ex} Protokoll & Tartomány \\ 
	\hline 
	\rule[-1ex]{0pt}{2.5ex} IP & 1-99, 1300-1999, 2000-2699 \\ 
	\hline 
	\rule[-1ex]{0pt}{2.5ex} Kiterj. IP & 100-199, 2000-2699 \\ 
	\hline 
	\rule[-1ex]{0pt}{2.5ex} AppleTalk & 600-699 \\ 
	\hline 
	\rule[-1ex]{0pt}{2.5ex} IPX & 800-899 \\ 
	\hline 
	\rule[-1ex]{0pt}{2.5ex} Kiterj. IPX & 900-999 \\ 
	\hline 
	\rule[-1ex]{0pt}{2.5ex} IPX szolgáltatáshirdetés (SAP) & 1000-1099 \\ 
	\hline 
\end{tabular}\\

Miután beléptünk a megfelelő parancsmódba és eldöntöttük, hogy milyen típusú listát
szeretnénk létrehozni, az \verb|access-list| paranccsal, illetve a szükséges paraméterek segítségével
kell megadnunk a hozzáférési lista utasításait.\\
A hozzáférési listák létrehozása az első lépés.\\
A második végrehajtandó művelet a listák hozzárendelése a megfelelő interfészekhez.\\

Az ACL definiálása a következő paranccsal:\\
\verb|Router(config) #access-list access-list-number {permit / deny test-conditions}|

\paragraph{1. lépés:} Az ACL-t egy globális utasítás azonosítja. A normál IP-címekhez az 1-99 tartomány van
fenntartva. Ez a szám mutatja az ACL típusát. A Cisco lOS 11 .2-es verziójától kezdődően az
ACL-ekhez nem csak szám, de név is rendelhető, például oktatasi\_csoport. A globális ACL
utasítás permit illetve deny kifejezése adja meg, hogy a Cisco lOS szoftver hogyan kezelje a
tesztfeltételeket kielégítő csomagokat. A permit általában azt jelenti, hogy a csomag
használhat egy vagy több - később meghatározandó - interfészt. A záró kifejezés(ek) az ACL
állítás által használandó tesztfeltételeket adják meg.

\paragraph{2. lépés:} Ezután alkalmazni kell az ACL-eket egy interfészre az \verb|access-group| paranccsal.\\
Példa: \verb|Router (config-if) #{protocol) access-group access-list number|\\
A hozzáférési lista számával azonosított összes ACL utasítás hozzárendelődik egy vagy több
interfészhez. Az ACL tesztfeltételeket kielégítő csomagok a hozzáférési csoport bármely
interfészét használhatják.

TCP/IP használata esetén az ACL-eket egy vagy több interfészhez lehet hozzárendelni.
Az \verb|ip access-group| parancs segítségével a bejövő vagy a kimenő forgalom szűrésére
állíthatjuk be.

\begin{verbatim}
	Router(config)#access-list 2 deny 172.16.1.1
	Router(config)#access-list 2 permit 172.16.1.0 0.0.0.255
	Router(config)#access-list 2 deny 172.16.0.0 0.0.255.255
	Router(config)#access-list 2 permit 172.0.0.0 0.255.255.255
	Router(config) #interface ethernet 0
	Router(config)#ip access-group 2 in
\end{verbatim}

Az access-group parancsot interfészkonfigurációs módban kell kiadni.
Amikor egy ACL-t hozzárendelünk egy interfészhez, akkor ki kell választanunk, hogy a
bejövő vagy a kimenő forgalomra vonatkozzon. A szűrés tehát az adott interfészre beérkező
és a róla távozó csomagokra vonatkozhat. Annak megállapításához, hogy az ACL a bejövő
vagy a kimenő forgalmat szűrje, az egyes interfészeket a forgalomirányító belsejéből kell
szemlélnünk. Ezt a szemléletet mindvégig meg kell őrizni. A valamilyen interfészen keresztül
beérkező forgalmat bejövő ACL, a kimenő forgalmat pedig kimenő ACL alapján szűrjük.
A számozott ACL-t létrehozása után hozzá kell rendelni egy interfészhez.
Számozott ACL-utasításokat tartalmazó ACL nem módosítható. Előbb törölnünk kell a 
\verb|no access-list lista-szám| paranccsal, majd újra be kell vinnünk a parancsokat.
(\verb|Router(config)#no access-list 2|)\\
Az ACL-ek létrehozásakor és életbe léptetésekor a következő alapvető szabályokat kell
betartani:
\begin{enumerate}[nosep]
	\item Irányonként és protokollonként egy ACL-t kell létrehozni.
	\item A normál hozzáférési listákat a célhoz a lehető legközelebb kell alkalmazni.
	\item A kiterjesztett hozzáférési listákat a forráshoz a lehető legközelebb kell alkalmazni.
	\item A kimenő és a bejövő jelzőket úgy kell használni, mintha a forgalomirányító belsejéből néznénk a portokat.
	\item Az utasítások feldolgozása sorban, a lista tetejétől az alja felé haladva történik, amíg a forgalomirányító egyezést nem talál. Ha nincs egyezés, a forgalomirányító eldobja a csomagot.
	\item Minden hozzáférési lista alján egy implicit deny any (mindent letilt) szabály található. Ez szabály nem jelenik meg az utasításlista alján.
	\item A hozzáférési listák utasításait a specifikusabbaktól az általánosabbak felé haladva kell megadni. Az egyes állomásokra vonatkozó tiltásokat kell először megadni, a csoportokra vonatkozó vagy általános szűrőket utolsóként kell elhelyezni.
	\item Elsőként az egyezési feltétel vizsgálata történik meg. Az engedélyező vagy tiltó részre kizárólag akkor kerül át a vezérlés, ha az egyezés igaz volt.
	\item Soha ne dolgozzunk aktívan működő hozzáférési listával!
	\item Először a logikai utasításokat felvázoló megjegyzéseket készítsük el szövegszerkesztővel, a tényleges végrehajtó műveleteket csak ezt követően írjuk meg.
	\item Az új sorok mindig a hozzáférési lista végére kerülnek. A no access-list x parancs a teljes listát törli. Számozott ACL-ek sorainak egyenként való hozzáadására vagy eltávolítására nincs lehetőség.
	\item Az IP alapú hozzáférési listák a célállomás elérhetetlenségét jelző ICMP-üzenetet küldenek az elutasított csomagok forrásainak, majd a bitszemetesbe dobják a csomagokat.
	\item Hozzáférési lista eltávolítását mindig körültekintően kell végezni. Ha a hozzáférési lista aktív interfészre vonatkozik, és eltávolítjuk, akkor az IOS verziójától függően alapértelmezett tiltó szabály léphet érvénybe az interfészen, ami a forgalom teljes leállását okozza.
	\item A kimenő szűrők nem vonatkoznak a helyi forgalomirányítóról kiinduló forgalomra.
\end{enumerate}

\subsubsection{Normál ACL-ek}
A normál ACL-ek az irányítandó IP-csomagok forráscímét ellenőrzik. Az összehasonlítás a
hálózati, alhálózati és állomáscím alapján egy egész protokollkészlet számára eredményez
engedélyezést vagy tiltást.\\
Például a Fa0/0 interfészen keresztül beérkező csomagok forráscímét és protokollját egyaránt
ellenőrizzük. Ha mindkettő engedélyezve van, a csomagok a forgalomirányítón keresztülvalamelyik kimenő interfészre kerülnek. Ha tiltva vannak, akkor eldobásuk a bejövő
interfészen történik meg.

A globális konfigurációs mód \verb|access-list| parancsának normál változatával normál, 1 és 99
közötti számú ACL definiálható. A Cisco IOS Software Release 12.0.1 és újabb változataiban
a sorszám 1300 és 1999 között is lehet, így akár 798 normál ACL-t is készíthetünk. Ezt az
újabb tartományt kibővített IP ACL-nek nevezzük.

\begin{verbatim}
	access-list 2 deny 172.16.1.1
	access-list 2 permit 172.16.1.0 0.0.0.255
	access—list 2 deny 172.16.0.0 0.0.255.255
	access-list 2 permit 172.0.0.0 0.255.255.255
\end{verbatim}

\begin{itemize}[nosep]
	\item Hozzáférési lista tartományok: I - 99 és 1300 - 1999
	\item Szürés csak az IP-forráscim alapján
	\item Helyettesítö maszkok
	\item A célhoz legközelebbi portra kell alkalmazni
\end{itemize}

Vegyük észre, hogy az első ACL-utasításnál nincs megadva helyettesítő maszk. Ilyenkor a
forgalomirányító az alapértelmezett 0.0.0.0 maszkot használja, vagyis vagy a teljes címnek
egyeznie kell, vagy az ACL ezen sora nem fog illeszkedni, és a forgalomirányító az ACL
következő sorára lép tovább.

A normál ACL-utasítások teljes szintaxisa a következő:\\
{\small\verb+Router(config)#access-list hozzáférési-lista-száma {deny | permit | remark} forrás [forrás-helyettesítő-maszkja] [log]+}

A \verb|remark| (megjegyzés) kulcsszó az ACL-eket könnyebben érthetővé teszi. A megjegyzések
hossza nem haladhatja meg a 100 karaktert.
Példa:
\begin{verbatim}
	access-list 1 remark Csak Jones állomását engedélyezzük
	access-list 1 permit 171.69.2.88
\end{verbatim}

Normál ACL-t törölni a parancs no változatával lehet. Ennek szintaxisa:\\
\verb|Router(config)#no access-list hozzáférési-lista-száma|

Az ip access-group parancs hozzákapcsol egy normál ACL-t egy interfészhez:
\verb+Router(config)#ip access-group {access-list-number | access-list-name} {in | out}+

A táblázat a szintaxisban használt paraméterek leírásait tartalmazza:\\
\begin{tabularx}{\linewidth}{l|X}
	Paraméter & Leírás\\
	\hline
	access-list-nurnber & Az ACL azonosító száma. Decimális szám 1 -99 (normál IP ACL) és 1300 - 1999 (kibővitett IP ACL).\\[1pt]
	deny & Megtagadja a hozzáférést, ha a feltételek teljesülnek.\\[1pt]
	permit & Engedélyezi a hozzáférést, ha a feltételek teljesülnek.\\[1pt]
	remark & A remark (megjegyzés) parancs használata a listák könnyebb megértését és megkeresését segíti.\\[1pt]
	source & Annak a hálózatnak vagy állomásnak a címe, ahonnan a csomagot elküldik. A forrás kétféleképpen adható meg:
		\begin{enumerate}[nosep]
		\item 32 bites cím megadása négy részből álló, pontokkal elválasztott decimális formátumban.
		\item az any kulcsszó a forrás rövidítése: source-wildcard of 0.0.0.0 255.255.255.55.
		\end{enumerate}\\
	source-wildcard & (Opcionális) A forrásra alkalmazandó helyettesítő bitek. A forráshelyettesítő maszkja kétféleképpen adható meg:
	\begin{enumerate}[nosep]
	\item 32 bites cím megadása négy részből álló, pontokkal elválasztott decimális formátumban. A figyelmen kívül agyandó bitpozíciókba 1-et kell írni.
	\item Az any kulcsszó a 0.0.0.0 255.255.255.255 értékű forrás és forráshelyettesítő maszk rövidítése.
	\end{enumerate}\\
	log & (Opcionális) A bejegyzésnek megfelelő csomagról egy tájékoztató célú naplózási üzenet jut a konzolra, (A konzolra küldött naplózási üzenetek részletessége a logging console paranccsal vezérelhető.)Az üzenet tartalmazza az ACL számát, a forráscímet, a csomagok számát és azt, hogy a csomagot engedélyezték-e vagy letiltották. Az üzenet az első megfelelő csomag megtalálásakor generálódik, ezután ötpercenként, megmutatva azoknak a csomagoknak a számát is, amelyek a megelőző öt percben lettek engedélyezve vagy elutasítva.
\end{tabularx}

\subsubsection{Kiterjesztett ACL-ek}
A kiterjesztett ACL-eket a normál ACL-eknél gyakrabban használjuk, mivel szélesebb körű
ellenőrzést tesznek lehetővé.\\
A kiterjesztett ACL-ek a csomagok forrás- és célcímét egyaránt ellenőrzik, illetve a
protokollok és a portszámok egyeztetésére is alkalmasak. Ezek a lehetőségek nagyobb
szabadságot biztosítanak az ACL által vizsgált adatok körülhatárolására. A csomagok
engedélyezése és tiltása forrás, cél, protokolltípus és portszám alapján egyaránt történhet.
Egy kiterjesztett ACL például engedélyezheti az elektronikus levelezést a Fa0/0 interfészről
megadott S0/0 célok felé, miközben tilthatja a fájlátviteleket és a webböngészést. A csomagok
eldobásakor bizonyos protokollok egy visszhangcsomagot küldenek a forrásnak, jelezve, hogy
a cél nem érhető el.

Egy-egy ACL-hez több utasítás is konfigurálható. Az utasítások mindegyikének ugyanazt a
hozzáférési lista számot kell tartalmaznia, így tud hivatkozni az azonos ACL-en belüli
utasításokra. Feltételutasításból tetszőleges számú adható meg, az ilyen utasítások
mennyiségét csak a forgalomirányító memóriájának nagysága korlátozza. Természetesen
minél több az utasítás, annál nehezebb az ACL megértése és kezelése.

A kiterjesztett ACL-utasítások szintaxisa meglehetősen hosszadalmas is lehet, akár a teljes
terminálablakot is kitöltheti. A helyettesítéseknél ugyancsak mód nyílik a host vagy az any
kulcsszó használatára a parancsokon belül.

A kiterjesztett ACL-utasítások végén az opcionálisan megadható TCP vagy UDP
portszámokkal tovább pontosíthatók a szabályok. A TCP/IP protokollkészlet jól ismert
portszámai az ábrán láthatók.\\
{\centering
\includegraphics[width=\linewidth]{fig/16-extACL-TCPIP_protocolls}}

A kiterjesztett ACL-ek logikai műveleteket – egyenlő (equal, eq), nem egyenlő (not equal,
neq), nagyobb mint (greater than, gt), kisebb mint (less than, lt) – is képesek végezni a
megadott protokollokon. A kiterjesztett ACL-ek hozzáférési lista száma 100 és 199 között
lehet. (Az újabb IOS-változatoknál a sorszám a 2000–2699 tartományba is tartozhat.)

Az \verb|ip access-group| paranccsal egy meglévő kiterjesztett ACL köthető hozzá egy interfészhez.
Ne feledjük, hogy interfészenként, irányonként és protokollonként csak egy ACL adható meg!
A parancs formátuma: \verb+Router(config-if)#ip access-group hozzáférési-lista-száma {in | out}+

%----------------------------------------------------------------------------
\subsection{Bevezetés a Cisco eszközök programozásába 2}
%----------------------------------------------------------------------------
\subsubsection{A forgalomirányítási táblázatok felépítése, statikus és dinamikus routing összehasonlítása.}

%\section[Mérés és folyamatirányítás]{Mérés és folyamatirányítás specializáció}
%%----------------------------------------------------------------------------
\section{Távközlő hálózatok}
{\footnotesize Fizikai jelátviteli közegek. Forráskódolás, csatornakódolás és moduláció. Csatornafelosztás és multiplexelési technikák. Vezetékes és a mobil távközlő hálózatok. Műholdas kommunikáció és helymeghatározás.}
%----------------------------------------------------------------------------
\subsection{Fizikai jelátviteli közegek.}

\subsection{Forráskódolás, csatornakódolás és moduláció.}

\subsection{Csatornafelosztás és multiplexelési technikák.}

\subsection{Vezetékes és a mobil távközlő hálózatok.}

\subsection{Műholdas kommunikáció és helymeghatározás.}


%%----------------------------------------------------------------------------
\subsection{Hálózatok hatékonyságanalízise}
%----------------------------------------------------------------------------
\subsubsection{Markov-láncok, születési-kihalási folyamatok.}

\subsubsection{A legalapvetőbb sorbanállási rendszerek vizsgálata.}

\subsubsection{A rendszerjellemzők meghatározásának módszerei, meghatározásuk számítógépes támogatása.}

%%----------------------------------------------------------------------------
\section{Adatbiztonság}
{\footnotesize Fizikai, ügyviteli és algoritmusos adatvédelem, az informatikai biztonság szabályozása. Kriptográfiai alapfogalmak. Klasszikus titkosító módszerek. Digitális aláírás, a DSA protokoll.}
%----------------------------------------------------------------------------
\subsection{Fizikai, ügyviteli és algoritmusos adatvédelem, az informatikai biztonság szabályozása.}
\begin{definition}[Adatvédelem]
	azon fizikai, ügyviteli és algoritmikus eszközök együttes felhasználását értjük, amelyek segítségével a véletlen adatvesztések és szándékos adatrongálódások és információ kiszivárogtatások megelőzhetők, vagy jelentős mértékben megnehezíthetők
\end{definition}
\paragraph{Fizikai adatvédelem} két lényegi dolgot takar: Egyrészt biztosítani kell az optimális, de legalább a még elfogadható \textbf{üzemi körülményeket} (hőmérséklet, páratartalom, por, tartalék alkatrészek stb.), másrészt pedig a szükséges \textbf{vagyonvédelmi intézkedésekről} sem szabad megfeledkezni. Például: Villamos hálózat helyes kialakítása; Szünetmentes tápegységek használata; Megfelelő szerverterem kialakítása(klimatizálás, füstérzékelés, árnyékolás,\dots); Megfelelő adattároló eszközfajták használata; Betörésvédelem.

\paragraph{Ügyviteli adatvédelem} a folyamatok szabályozásának, a szabályzatoknak a kialakítása és védelme. A fizikai adatvédelem önmagában ugyanis nem elegendő. Példa: hiába zárjuk be a szerverszoba ajtaját, ha a portás beengedi azt, aki egy szerszámos táskával érkezvén arról tájékoztatja, hogy ’zsírozni kell a switcheket’ (social hacking/engineering). Tehát szükséges \textbf{pontosan szabályozni}, hogy \textbf{ki}, \text{mikor}, \textbf{mit} és \textbf{hogyan} tehet meg, illetve nem tehet meg. Szükség van \textbf{informatikai biztonsági szabályzatra} is, amely mindezt egységes módon áttekinti. Megfelelő felhasználó menedzselési rendet kell kialakítani, hogy a felhasználók, hozzáférési jogosultságaik, munkájukból adódó szerepköreik kezelése összhangba hozható legyen. Példák: Feladat- és jogkörök szétválasztása; Hozzáférések és tevékenységek regisztrálása; Személyazonosítás; Hatáskörök és felelősségek szétválasztása vagy átlapolása.

\paragraph{Algoritmikus adatvédelem} feladata olyan programok és eljárások alkalmazása, amelyek segítik az előző két terület feladatait és létrehozzák azokat a számítógépes védelmi funkciókat, amik ezen a területen meggátolják az adatokhoz való illetéktelen hozzáférést és módosítást. Példák: Hálózati azonosítás; \textbf{Titkosítás}; Behatolásvédelem; Automatikus adatmentés; Többforrásos adattárolás.

\paragraph{Az informatikai biztonsági szabályrendszer szükségessége:} (1) az adatok egyre inkább elektronikus formában jelennek meg; (2) a Szervezetek informatika nélkül működésképtelenek; (3) az informatikai függőség egyre nagyobb; (4) ugyanakkor a fenyegetettség is egyre növekszik; (5) az üzletfolytonossághoz kritikus fontosságú; (6) a kárpotenciál és a kockázati tényezők szervezetenként eltérőek lehetnek!

\subsubsection{Biztonsági célok}
Alapkövetelmények, amelyek teljesülése az üzemszerű használhatóság előfeltétele:
\begin{enumerate}
	\item rendelkezésre állás (elérhetőség az arra jogosultak számára)
	\item sértetlenség (valódiság)
	\item bizalmasság (jellegtől függően)
	\item nyomon követhetőség, hitelesség
	\item Biztosítékok (az információs rendszer teljességére nézve)
\end{enumerate}
Ez alapján úgy lehet meghatározni az Informatikai Biztonság fogalmát, hogy az akkor áll fenn, ha az információs rendszer védelme az alapkövetelmények szempontjából
\begin{itemize}
	\item \textbf{zárt:} minden fontos fenyegetést figyelembe vesz
	\item \textbf{teljes körű:} a rendszer összes elemére kiterjedő
	\item \textbf{folyamatos:} megszakítás nélküli, az időben változó körülmények ellenére is
	\item \textbf{kockázatarányos:} a feltehető kárérték és a kár valószínűségének szorzata nem haladhat meg egy előre rögzített küszöböt, amely egy üzleti döntés.
\end{itemize}

\subsection{Kriptográfiai alapfogalmak.}
\begin{center}
	\includegraphics[width=0.7\linewidth]{fig/14-Crypto_scheme}
\end{center}
\begin{definition}[titkostási rendszer]
	A $(\mathcal{P}, \mathcal{C}, \mathcal{K},Enc,Dec,Key)$ hatost titkosítási rendszernek (sémának) nevezzük, ahol:
	\begin{itemize}[nosep]
		\item $\mathcal{P}, \mathcal{C}, \mathcal{K}$ a nyílt és titkos üzenetek, valamint lehetséges kulcsok véges, nemüres halmaza ($1<|\mathcal{P}|,|\mathcal{C}|,|\mathcal{K}| < \infty$).
		\item $Enc:\; \mathcal{K}\times\mathcal{P} \to c\;$ egy titkosító függvény ($c \in \mathcal{C}$: titkosított üzenet)
		\item $Dec:\; \mathcal{K}\times\mathcal{C} \to m\;$ egy visszafejtő függvény ($m \in \mathcal{P}$: nyílt üzenet)
		\item $Key \subseteq \mathcal{K}\times\mathcal{K}$ kulcspárok halmaza
	\end{itemize}
\end{definition}
\begin{definition}[Korrekt visszafejtés]
	A titkosító rendszer korrekt visszafejtést biztosít, ha minden $(k_E,k_D)\in\mathcal{K},\; m\in\mathcal{P}$ esetén
	$$Dec(k_D,Enc(k_E,m)) = m$$
\end{definition}
\begin{note}~\\
	\begin{itemize}[nosep]
		\item $(k_E,k_D)$ titkosító-visszafejtő kulcspár
		\item $c = Enc(k_E,m)$
		\item definíció alapján, rögzített $(k_E,k_D)$ esetén az $Enc_{k_E}(m) = Enc(k_E,m)$ függvény injektív, azaz:
		$$Enc_{k_E}(m_1) = Enc_{k_E}(m_2) \Leftrightarrow m_1 = m_2$$
	\end{itemize}
\end{note}
\begin{definition}[Teljes titkosítási rendszer]
	A titkosítási rendszert teljesnek nevezzük, ha minden $(k_E,k_D)$ esetén az $Enc_{k_E}(m)$ függvény szürjektív, azaz minden $c\in\mathcal{C}$ titkos üzenet esetén létezik $m\in\mathcal{P}$ nyílt üzenet, amelyikre $Enc_{k_E}(m) = c$. Ebben az esetben $|\mathcal{P}| = |\mathcal{C}|$
\end{definition}
Természetes elvárás egy titkosítási rendszerrel szemben, hogy ha vannak $k_{E1} \neq k_{E2}$ kulcsok, akkor $Enc_{k_{E1}}(m) \neq Enc_{k_{E2}(m)}$. Továbbá ha $(k_E,k_{D1}), (k_E,k_{D2})\in Key$, akkor $k_{D1} = k_{D2}$.
\begin{definition}[Teljes kulcstér]
	Legyen $|\mathcal{P}| = |\mathcal{C}|$ . A kulcsteret teljesnek nevezzük, ha minden $f:\;\mathcal{P} \to \mathcal{C}$ bijektív leképezéshez létezik $k_E \in \mathcal{K}$, amelyikre $Enc_{k_E} = f$.
\end{definition}
\begin{description}[nosep]
	\item[Kriptográfia] A kulcs alapú titkosítási technikák tudománya
	\item[Kriptanalízis] A kulcs alapú titkosítási technikák feltörésének, támadásának tudománya
	\item[Kriptológia] Kriptográfia + Kriptanalízis
	\item[Nyílt szöveg (plain text)] Az eredeti, mindenki által értelmezhető információ
	\item[Titkosított szöveg (ciphertext)] A titkosított információ
	\item[Titkosítás (encryption)] Eljárás, mely során az információt a birtokosa titkossá nyilvánít.
	\item[Visszafejtés (decryption)] A titkosított információból az eredeti visszaállítása.
	\item[Rejtjelezés] Adott módszer a nyílt szöveg kódolására (és visszafejtésére)
	\item[Kulcs (key)] Az az információ, amelynek segítségével a rejtjelezés történik. Két típus: nyilvános és titkos kulcs.
	\item[Szimmetrikus titkos. rendszer] Ha van $(k_E,k_D)$ kulcspárunk, $k_E$ nyílvános kulcs megegyezik a $k_D$ titkos kulccsal vagy $k_D$ polinomiális időben kiszámolható $k_E$-ből.
	\item[Aszimmetrikus titkos. rendszer] A tetszőlegesen választott $k_E$ és $k_D$ kulcspárok annyira különböznek, hogy nem létezik polinomiális időbonyolultságú algoritmus, mely $k_E$-ből kiszámolja $k_D$-t
\end{description}

\subsection{Klasszikus titkosító módszerek.}
A számítógép megjelenése előtt (before computers, BC) használt rendszereket, történelmi vagy klasszikus titkosítási rendszereknek nevezzük. Nem egyértelműen definiált, de nagyjából a számítógéppel könnyen támadható rendszereket nevezzük így. Jellemzően élő nyelvi szöveg titkosítására használták. Főbb típusai:
\begin{easylist}[enumerate]
	# Helyettesítéses titk. rendszerek: Az üzenet egy betűjét más betűre vagy szimbólumra cseréli
	## monoalfabetikus: a csere transzformációja változatlan
	## polialfabetikus: az egymás után következő betűket más-más transzformációval titkosítjuk
	# Permutációs titk. rendszerek: az üzenet betűit más sorrendben írjuk fel (mint egy anagramma, de a titkos szöveg nem feltétlenül értelmes)
\end{easylist}
\paragraph{Betűk kicserélése absztrakt szimbólumokra} Helyettesítéses, monoalfabetikus
\paragraph{Caesar--titkosítás} Monoalfabetikus, helyettesítéses.\\
$Enc_k(m) = k+m (mod\; 26)$\\
$Dec_k(c) = c-k (mod\; 26)$
\paragraph{Affin titkosítás} Monoalfabetikus\\
$k_E = (\alpha,\beta)\; k_D = (\alpha^{-1},\beta)\quad lnko(\alpha,26) = 1$\\
$Enc = \alpha \cdot m + \beta (mod\; 26)$\\
$Dec = \alpha^{-1} \cdot (c - \beta) (mod\; 26)$
\paragraph{Vigènre-titkosítás} polialfabetikus, a kulcs valamilyen karaktersorozat. A Vernam--rejtjelező (OTP) hasonló hozzá, azonban az kulcs hossza megegyezik az üzenettel. Bináris esetben a Vernam--rejtjelező az üzeneten és a kulcson bitenkénti XOR műveletet végrehajtva állítja elő a titkos üzenetet.\\
$c[i] = m[i] + k_E[i\; mod\; n] (mod\; 26)$\\
$m[i]= c[i]- k_E[i\; mod\; n] (mod\; 26)$

\paragraph{Enigma} Polialfabetikus. Elektromechanikus rejtjelező, ahol a kulcsot a gépben található tárcsák kezdőpozíciója jelentette.

\subsection{Modern titkosítási rendszerek}
\subsubsection{Folyamtitkosítás}
Bitsorozat titkosítása bitenként. Legegyszerűbb módszer a One-Time-Pad titkosítás, ahol titkosított bitsorozat, az eredeti sorozat egy álvéletlen bitsorozattal történő XOR művelet eredménye $c[i] = m[i]XORk[i]$. Shannon bebizonyította, hogy egy teljesen véletlen kulcsfolyam esetén a titkos üzenet elméletileg feltörhetetlen. Tehát ennek a titkosításnak az erőssége az álvéletlenszám-generátor erősségén múlik. Alkalmazásakor a mesterkulcsot a kulcsgenerátor kezdeti állapota jelenti. A kulcsfolyam-generálásnak több variánsa is létezik:
\paragraph{Lineárisan visszacsatolt léptető regiszter -- LFSR}
A kulcsfolyam $l$ bitjein egy $a = (a_0,a_1,\dots,a_{l-1})$ értékkel bitenkénti ÉS műveletet hajtunk végre, majd a részeredményen XOR műveletet hajtunk végre. Így kapjuk meg a $k_i$ kulcsot:
$$ k_i = a_{l-1}k_{i-1}\oplus a_{l-2}k_{i-2}\oplus\dots\oplus a_{0}k_{i-l}$$
\begin{center}
	\includegraphics[width=0.7\linewidth]{fig/14-LFSR}
\end{center}
Az $a_i$ és kezdeti $k_i$ értékei adják a generátor kezdeti állapotát. Fontos, hogy ezeket az értékeket jól válasszuk meg, mert ezen múlik az álvéletlen sorozat minősége. Ez leginkább két dolgot takar: sorozat eloszlása (fontos, hogy egyenletes legyen) és a periódus hossza (minél hosszabb annál jobb).
\paragraph{Lineárisan rekurzív sorozat -- LSR}
Az LFSR általánosítása. C/C++ rand() függvénye implementálja.\\
$a = 31835, b = 1906, k_0 = 41$\\
$k_i = a \dots k_{i-1} + b (mod 2^{15})$\\
periódusa $2^{15}$
\paragraph{Blum--Blum--Shub generátor -- BBS}
$p, q$ nem közeli nagy prímek: $m = p\times q$ és $a_0$ úgy választjuk, hogy $lnko(a_0,m)=1$.\\
$a_i = a_{i-1}^2 \;(mod\; n)$
$k_i = a_i(mod 2)$ vagy például $a_i$ paritása.\\
A következő $a_i$ az előzőekből exponenciális időben számolható ki. Periódusa kb. $p\cdot q$, ha $lnko(p-1,q-1)$ kicsi, ezért kell, hogy p és q ne legyenek egymáshoz közel.
\paragraph{RC4}
\begin{multicols}{2}
Egy S kezdeti tömb feltöltése és összekeverése után a következő bitet az  %\ref{fig:14-rc4}~
ábrán látható módon kapjuk meg, (i = (i + 1) mod 256 j = j + S i mod 256).
Felhasználja az SSL/TLS, és a WEP protokoll.
\begin{center}
	\includegraphics[width=\linewidth]{fig/14-RC4}
%	\caption{A következő elem (K) meghatározása az S tömbből. Minden kiolvasás előtt S[i] és S[j] értéket felcseréljük a tömbben}
%	\label{fig:14-rc4}
\end{center}
\end{multicols}

\subsubsection{DES}
\begin{multicols}{2}
\begin{easylist}[itemize]
# 1975-ben jelent meg, a LUCIFER titkosítási rendszer egyik változata.
# Ma már nem létezik szabványként, mert feltörték, helyette a Triple DES-t ajánlott használni. 
# Szimmetrikus, blokktitkosítási rendszer, a blokk mérete 64 bit. Ha ennél nagyobb méretű üzenetet kell titkosítani, akkor azt fel kell tördelni valamilyen blokktitkosítási módszerrel.
# Matematikai alapja a Fiestel struktúra.
# A Kulcs 64 bites, amelyből 56 bit random és 8 a random bitek alapján meghatározott. 
# S-BOX-ok használata a titkosítás és a visszafejtés során.
# 16 körből áll az algoritmus, ezért 16 körkulcs kerül legenerálásra az eredeti kulcsból. Visszafejtéskor a körkulcsokat fordított sorrendben kell alkalmazni.
# Jellemzők:
## az S-BOX-okat kivéve minden lépése az algoritmusnak lineáris
## az S-BOX-ok eredete nem ismert, de a feltételeknek megfelelnek.
## nagy problémája a kis kulcstér, ezért kell inkább TDES-t alkalmazni ma már.
\end{easylist}
\end{multicols}
\begin{figure}[h]
	\centering
	\includegraphics[width=0.2\linewidth]{fig/14-DES_Feistel_net}
	\includegraphics[width=0.7\linewidth]{fig/14-DES_Feistel_func}
	\caption{A feistel-háló és a feistel függvény (F - Feistel függvény; IP,FP - egymással ellentétes permutációk; E - kiterjesztő függvény; S - S-boxok; P - permutáció;sk - a mesterkulcsból képzett 16 részkulcs)}
	\label{fig:14-desfeistelnet}
\end{figure}

\subsubsection{AES}
\begin{multicols}{2}
\begin{easylist}[itemize]
# Szimmetrikus, blokktitkosítási rendszer. Kifejlesztői: RIJNDAEL (Vincent Rijmen, Joan Daemen).
# A (T)DES-nél hatékonyabb titkosításra képes, mivel matematikai alapja nem Fiestel struktúra, itt a bitek nem csak keverednek, de meg is változnak!
# 128, 192, 256 bites blokkhossz és kulcshossz kezelésére képes bármilyen kombinációban.
# Bonyolult matematikai háttere van, a bájtokat {0,1} együtthatós polinomként kezeli.
# Tervezésekor fontos volt az ismert támadásokkal szembeni védelem, a gyorsaság és tömör kódolhatóság
# A belső algoritmus ciklusainak száma a blokkhossz és a kulcshossz függvényében lehet 10, 12, 14
# Visszafejtésnél az összes transzformáció inverzét kell végrehajtani.
\end{easylist}
Működése 3 fő részből és 4 műveletből áll:
\begin{easylist}[enumerate]
# Előkészítés:
## AddRoundKey
# Ismételt rész:
## SubBytes
## ShiftRows
## MixColumns
## AddRoundKey
# Utófeldolgozás:
## SubBytes
## ShiftRows
## AddRoundKey
\end{easylist}
\begin{description}[nosep]
	\item[AddRoundKey] XOR művelet az állapotmátrix és a megfelelő sorszámú kulcs között
	\item[SubBytes] speciális S-boxokat használ, mely egy nemlineáris helyettesítési kódolást valósít meg
	\item[ShiftRows] a sorokat ciklikusan eltolják. Az 1. sor 0, a 2. sor 1,\dots, 4. sor 3 hellyel
	\item[MixColumns] oszloponként egy lineáris transzformáció:
	$\begin{matrix}
	2 & 3 & 1 & 1\\
	1 & 2 & 3 & 1\\
	1 & 1 & 2 & 3\\
	3 & 1 & 1 & 2
	\end{matrix}$
\end{description}
\end{multicols}

\subsubsection{RSA}
\begin{easylist}[itemize]
# Feltalálói: Ron Rivest, Adi Shamir, Leonard Adleman, 1977-ben.
# Az első aszimmetrikus, PKI-t használó kriptográfiai rendszer. (NEM BLOKKTITKOSÍTÓ RENDSZER)
# A kommunikációban résztvevő mindkét félnek rendelkeznie kell egy Publikus és egy Titkos kulccsal (PK; SK), amelyeket a kommunikáció titkosságának biztosítása érdekében felhasználnak.
# Kulcsgeneráló algoritmus, PK=(n,e) és SK=(n,d)
## Véletlenül választunk két nagy prímet: p,q
## Kiszámoljuk: n=p*q
## Kiszámoljuk: \textphi(n)=(p-1)*(q-1)
## Választunk véletlen egy e-t, hogy 1 < e < \textphi(n) és (e,\textphi(n)) = 1
## Kiszámítjuk d-t, hogy 1 < d < \textphi(n) és e*d= 1 (mod \textphi(n))
# A PK és SK ismeretében a titkosító algoritmus: 	c = ENC\textsubscript{PK}(m) = m\textsuperscript{e} (mod n)
# A PK és SK ismeretében a visszafejtő algoritmus: 	m = DEC\textsubscript{SK}(c) = c\textsuperscript{d} (mod n)
# Az RSA feltörésének nehézségét a prím-faktorizáció nehézsége adja:  \{c,n,e\} $\leftarrow$ \{m\}  nehéz!
# p és q választásánál fontos, hogy hosszuk bitekben hasonló legyen, viszont ne legyenek túl közeliek
# e választásánál, annál jobb minél kisebb, általában 65537 szokott lenni.
\end{easylist}

\subsection{Digitális aláírás, a DSA protokoll.}
A digitális aláírás lényege:
\begin{easylist}[enumerate]
# Letagadhatatlanság --- alkalmas az aláíró azonosítására 
# Hitelesítés --- más nem tudja létrehozni 
# Hamisíthatatlanság --- akár az aláírás, akár a dokumentum módosul, észrevehető.
\end{easylist}
A letagadhatatlanság és harmadik fél számára is elfogadható hitelesítés
alapvetően kétféle megoldással lehetséges:
\begin{easylist}[itemize]
# megbízható harmadik féllel (Trusted Third Party, TTP) (szigorúan véve nem DS)
# közvetlen digitális aláírás (nyilvános kulcsú titkosítással)
\end{easylist}
Digitális aláírás fajtái:
\begin{easylist}[itemize]
# RSA alapú --- Az aláíró a titkos kulcsával titkosítja az üzenetét vagy annak kivonatát. A hitelességet az adja, hogy egyedül az aláíró nyilvános kulcsa tudja visszafejteni a dokumentumot (vagy a kivonatot). Hátrányai:
## nem megbízható --> egzisztenciális hamisítás: egy tetszőleges s értéket választva előállíthatjuk az $ m \equiv s^e (n) $ üzenetet.
## alakíthatóság --> $(m_1,s_{m_1}),(m_2,s_{m_2})$ létező aláírásokból előállítható az $(m_1m_2,s_{m_1}s_{m_2})$ új aláírás
## Ezek a hátrányok nem jelentkeznek, ha az üzenet helyett annak kivonatát írjuk alá (hash-and-sign):\\
$S_m\equiv Hash(m)^d (n) \Rightarrow (m,S_m)$\\
ellenőrzés: $Hash(m) \equiv S_m^d (n)$
# ElGamal alapú --- Hátránya, hogy az aláírás kétszer akkora, mint a dokumentum
# DSA protokoll --- Digital Signature Algorithm. A digitális aláírás szabvány (DSS) felhasználja.
\end{easylist}

\subsubsection{DSA}
\begin{multicols}{2}
Kulcsgenerálás:
\begin{enumerate}[nosep]
	\item $p, q$ olyan prímek, hogy $q|p-1$ (p-1 q többszöröse)
	\item meghatározzuk $g$, melynek rendje $q$ moduló $p$, azaz $g^t \equiv 1 (p) \Leftrightarrow t= k\cdot q$
	\item x egy véletlen szám és $a \equiv g^x (p)$
	\item PK=(p,q,g,a), SK=(x)
\end{enumerate}
Aláírás:
\begin{enumerate}[nosep]
	\item $k$ véletlen szám kiválasztása
	\item $r \equiv (g^k\;(mod\;p))\; (q)$
	\item $t \equiv k^{-1}(Hash(m)+xr)\; (q)$
	\item $s_m = (r,t)$
\end{enumerate}
Ellenőrzés:
\begin{enumerate}[nosep]
	\item $v_1 \equiv Hash(m)\cdot t^{-1}\; (q)$
	\item $v_2 \equiv r\cdot t^{-1}\; (q)$
	\item Az aláírás rendben van, ha $r \equiv (q^{v_1}a^{v_2}\;(mod\;p))\;(q)$
\end{enumerate}
\end{multicols}
Az üzenet ellátható időbélyegzővel is. Az üzenet kivonatát átadjuk egy időbélyegző szolgáltatónak, aki hozzáfűz egy T időpontot és így aláírja a kivonatot. Ezután a bélyegzett üzenet így néz ki $(m,(m|T,s_{m|T}))$. Az ellenőrző biztosan tudja, hogy T időpillanatban már létezett a dokumentum.
%%----------------------------------------------------------------------------
\section{A RIP protokoll működése és paramétereinek beállítása (konfigurációja).}
%----------------------------------------------------------------------------

%%----------------------------------------------------------------------------
\section{Bevezetés a Cisco eszközök programozásába 1}
{\footnotesize A forgalomszűrés, forgalomszabályozás (Trafficfiltering, ACL) céljai és beállítása (konfigurációja) egy választott példa alapján.}
%----------------------------------------------------------------------------
\subsection{A forgalomszűrés, forgalomszabályozás (Trafficfiltering, ACL) céljai és beállítása (konfigurációja) egy választott példa alapján.}
A rendszergazdáknak meg kell találniuk annak a módját, hogy megakadályozzák a hálózat
jogosulatlan elérését, miközben a belső felhasználók számára lehetővé teszik a szükséges
szolgáltatások használatát. Bár a biztonsági eszközök, mint például a jelszavak, a visszahívó
berendezések és a fizikai biztonsági eszközök sok segítséget nyújtanak, gyakran nem
rendelkeznek az alapvető forgalomszűréshez szükséges rugalmassággal és a rendszergazdák
igényeinek megfelelő vezérlési lehetőségekkel. Előfordulhat például a hálózati rendszergazda
lehetővé kívánja tenni a felhasználók számára az internet elérését, de nem akarja, hogy külső
felhasználók telnettel hozzáférhessenek a LAN-hoz.

A forgalomirányítók alapvető forgalomszűrési lehetőségeket biztosítanak, például hozzáférési
listákat (access control list, ACL) az internetes forgalom letiltásához. Az ACL engedélyező és
tiltó utasítások sorozata, melyek címekre vagy felsőbb rétegbeli protokollokra alkalmazhatók.
Ebben a modulban a normál és kiterjesztett ACL-ek használatáról, mint a hálózati forgalom
vezérlési módszeréről fogunk tanulni, továbbá arról, hogyan használhatók az ACL-ek
valamilyen biztonsági megoldás részeként.

Ezenkívül a fejezetben tippek, elgondolások, javaslatok és általános útmutatások is
szerepelnek az ACL-ek használatával, valamint a létrehozásukhoz szükséges parancsokkal és
beállításokkal kapcsolatban. Végül a fejezet példákat mutat a normál és a kiterjesztett ACL-
ekre és a forgalomirányítók interfészein történő alkalmazásukra.

Az ACL-ek akár egyetlen sorból is állhatnak, ha a cél egy adott állomásról származó
csomagok továbbításának engedélyezése, de tartalmazhatják szabályok és feltételek rendkívül
összetett halmazát is, amelyek pontosan megadják az engedélyezett forgalom jellegét, illetve
befolyásolják a forgalomirányító folyamatok teljesítményét.

\subsubsection{A hozzáférési listák működésének alapelvei}
Az ACL-ek feltétellisták, amelyek a forgalomirányító interfészén keresztülhaladó forgalomra
vonatkozóan lépnek érvénybe. Ezek a listák írják elő a forgalomirányító számára, hogy a
csomagokat fogadja el vagy utasítsa vissza. Az elfogadás és a visszautasítás meghatározott
feltételek alapján is történhet. Az ACL-ek segítségével lehetővé válik a forgalom felügyelete,
valamint a hálózatról kiinduló és az oda befutó elérések biztonságossá tétele.

ACL-ek minden irányított protokollhoz létrehozhatók, többek közt az internetprotokollhoz
(IP) és a hálózatközi csomagcseréhez (IPX) is. Az ACL-ek a forgalomirányítón konfigurálva
adott hálózat vagy alhálózat elérhetőségének vezérlésére használhatók.

Az ACL-ek úgy szűrik a hálózati forgalmat, hogy az irányított csomagok továbbítását vagy
eldobását írják elő a forgalomirányító interfészein. A forgalomirányító minden csomagot
megvizsgál annak meghatározásához, hogy azt az ACL-ben meghatározott feltételek szerint
továbbítani vagy eldobni kell. Az ACL-ek a forrás- és a célcímekre, a protokollokra és a
felsőbb rétegbeli portszámokra nézve adnak meg feltételeket.

Az ACL-eket protokollonként, irányonként és portonként kell létrehozni. Ha szabályozni
szeretnénk egy adott interfész forgalmát, akkor a rajta engedélyezett protokollok
mindegyikéhez külön ACL-t kell készíteni. Egy ACL egy interfészen egyszerre csak egy
irány forgalmát szabályozza. Minden irányhoz külön ACL-t kell létrehozni, egyet a kimenő,
egyet pedig a bejövő forgalomhoz. Végül minden interfészen több protokoll és irány is
definiálható. Ha például egy forgalomirányítónak két interfésze van, és ezeken IP, AppleTalk
és IPX alapú forgalom zajlik, akkor 12 külön ACL-t kell rajta létrehozni: minden
protokollhoz egy-egy ACL, szorozva kettővel a kimenő és a bejövő forgalom miatt, szorozva
kettővel, vagyis a portok számával.

ACL-eket többek között a következő okok miatt szokás létrehozni:
\begin{enumerate}[nosep]
	\item Korlátozható a hálózat forgalma, és növelhető a teljesítménye.
	\item Biztosítják a forgalom szabályozását.
	\item Az útvonalfrissítések továbbítása is korlátozható. Ha a hálózati környezet miatt nincs szükség a frissítésekre, sávszélességet lehet megtakarítani.
	\item Alapszintű hálózati hozzáférés-szabályozást biztosítanak. Az ACL-ek engedélyezhetik például a hálózat egy részének elérését egy állomás számára, és megtilthatják egy másiknak. Lehetséges például, hogy az A állomás hozzáférhet a személyzeti osztály hálózatához, míg a B állomásnak ezt megtiltjuk.
	\item Eldönthetjük, hogy milyen típusú forgalom továbbítódjon és milyen törlődjön a forgalomirányító interfészein. Például engedélyezhetjük az elektronikus levelek továbbítását, de a telnetet tilthatjuk.
	\item A rendszergazdák szabályozhatják, hogy az ügyfelek a hálózat mely részeihez férhetnek hozzá.
	\item Ki lehet választani bizonyos állomásokat, és számukra engedélyezni vagy megtiltani a hálózat egy részének elérését.
	\item A felhasználók számára engedélyezni vagy tiltani lehet bizonyos szolgáltatások, például az FTP vagy a HTTP elérését.
\end{enumerate}

Ha egy forgalomirányítón nincsenek ACL-ek megadva, akkor a forgalomirányítóba befutó csomagok mindegyike továbbhaladhat a hálózat bármely része felé.
\begin{figure}[h]
	\centering
	\begin{subfigure}{0.45\linewidth}
		\includegraphics[width=\linewidth]{fig/16-ACL}
		\caption{}
	\end{subfigure}
	\begin{subfigure}{0.45\linewidth}
		\includegraphics[width=\linewidth]{fig/16-ACL_list-schema}
		\caption{}
	\end{subfigure}
	\caption{}
\end{figure}

\subsubsection{ACL-ek létrehozása}
Az ACL-ek létrehozása globális konfigurációs módban történik.
Számos különböző típusú ACL létezik, így normál, kiterjesztett, IPX, AppleTalk stb. ACL.
Amikor egy forgalomirányítón ACL-eket konfigurálunk, akkor egy-egy szám
hozzárendelésével mindegyiket egyértelműen azonosítanunk kell. A szám megadja az ACL
típusát; ebből következően az adott típushoz tartozó értéktartományba kell esnie:\\
\begin{tabular}{l|l}
	\hline 
	\rule[-1ex]{0pt}{2.5ex} Protokoll & Tartomány \\ 
	\hline 
	\rule[-1ex]{0pt}{2.5ex} IP & 1-99, 1300-1999, 2000-2699 \\ 
	\hline 
	\rule[-1ex]{0pt}{2.5ex} Kiterj. IP & 100-199, 2000-2699 \\ 
	\hline 
	\rule[-1ex]{0pt}{2.5ex} AppleTalk & 600-699 \\ 
	\hline 
	\rule[-1ex]{0pt}{2.5ex} IPX & 800-899 \\ 
	\hline 
	\rule[-1ex]{0pt}{2.5ex} Kiterj. IPX & 900-999 \\ 
	\hline 
	\rule[-1ex]{0pt}{2.5ex} IPX szolgáltatáshirdetés (SAP) & 1000-1099 \\ 
	\hline 
\end{tabular}\\

Miután beléptünk a megfelelő parancsmódba és eldöntöttük, hogy milyen típusú listát
szeretnénk létrehozni, az \verb|access-list| paranccsal, illetve a szükséges paraméterek segítségével
kell megadnunk a hozzáférési lista utasításait.\\
A hozzáférési listák létrehozása az első lépés.\\
A második végrehajtandó művelet a listák hozzárendelése a megfelelő interfészekhez.\\

Az ACL definiálása a következő paranccsal:\\
\verb|Router(config) #access-list access-list-number {permit / deny test-conditions}|

\paragraph{1. lépés:} Az ACL-t egy globális utasítás azonosítja. A normál IP-címekhez az 1-99 tartomány van
fenntartva. Ez a szám mutatja az ACL típusát. A Cisco lOS 11 .2-es verziójától kezdődően az
ACL-ekhez nem csak szám, de név is rendelhető, például oktatasi\_csoport. A globális ACL
utasítás permit illetve deny kifejezése adja meg, hogy a Cisco lOS szoftver hogyan kezelje a
tesztfeltételeket kielégítő csomagokat. A permit általában azt jelenti, hogy a csomag
használhat egy vagy több - később meghatározandó - interfészt. A záró kifejezés(ek) az ACL
állítás által használandó tesztfeltételeket adják meg.

\paragraph{2. lépés:} Ezután alkalmazni kell az ACL-eket egy interfészre az \verb|access-group| paranccsal.\\
Példa: \verb|Router (config-if) #{protocol) access-group access-list number|\\
A hozzáférési lista számával azonosított összes ACL utasítás hozzárendelődik egy vagy több
interfészhez. Az ACL tesztfeltételeket kielégítő csomagok a hozzáférési csoport bármely
interfészét használhatják.

TCP/IP használata esetén az ACL-eket egy vagy több interfészhez lehet hozzárendelni.
Az \verb|ip access-group| parancs segítségével a bejövő vagy a kimenő forgalom szűrésére
állíthatjuk be.

\begin{verbatim}
	Router(config)#access-list 2 deny 172.16.1.1
	Router(config)#access-list 2 permit 172.16.1.0 0.0.0.255
	Router(config)#access-list 2 deny 172.16.0.0 0.0.255.255
	Router(config)#access-list 2 permit 172.0.0.0 0.255.255.255
	Router(config) #interface ethernet 0
	Router(config)#ip access-group 2 in
\end{verbatim}

Az access-group parancsot interfészkonfigurációs módban kell kiadni.
Amikor egy ACL-t hozzárendelünk egy interfészhez, akkor ki kell választanunk, hogy a
bejövő vagy a kimenő forgalomra vonatkozzon. A szűrés tehát az adott interfészre beérkező
és a róla távozó csomagokra vonatkozhat. Annak megállapításához, hogy az ACL a bejövő
vagy a kimenő forgalmat szűrje, az egyes interfészeket a forgalomirányító belsejéből kell
szemlélnünk. Ezt a szemléletet mindvégig meg kell őrizni. A valamilyen interfészen keresztül
beérkező forgalmat bejövő ACL, a kimenő forgalmat pedig kimenő ACL alapján szűrjük.
A számozott ACL-t létrehozása után hozzá kell rendelni egy interfészhez.
Számozott ACL-utasításokat tartalmazó ACL nem módosítható. Előbb törölnünk kell a 
\verb|no access-list lista-szám| paranccsal, majd újra be kell vinnünk a parancsokat.
(\verb|Router(config)#no access-list 2|)\\
Az ACL-ek létrehozásakor és életbe léptetésekor a következő alapvető szabályokat kell
betartani:
\begin{enumerate}[nosep]
	\item Irányonként és protokollonként egy ACL-t kell létrehozni.
	\item A normál hozzáférési listákat a célhoz a lehető legközelebb kell alkalmazni.
	\item A kiterjesztett hozzáférési listákat a forráshoz a lehető legközelebb kell alkalmazni.
	\item A kimenő és a bejövő jelzőket úgy kell használni, mintha a forgalomirányító belsejéből néznénk a portokat.
	\item Az utasítások feldolgozása sorban, a lista tetejétől az alja felé haladva történik, amíg a forgalomirányító egyezést nem talál. Ha nincs egyezés, a forgalomirányító eldobja a csomagot.
	\item Minden hozzáférési lista alján egy implicit deny any (mindent letilt) szabály található. Ez szabály nem jelenik meg az utasításlista alján.
	\item A hozzáférési listák utasításait a specifikusabbaktól az általánosabbak felé haladva kell megadni. Az egyes állomásokra vonatkozó tiltásokat kell először megadni, a csoportokra vonatkozó vagy általános szűrőket utolsóként kell elhelyezni.
	\item Elsőként az egyezési feltétel vizsgálata történik meg. Az engedélyező vagy tiltó részre kizárólag akkor kerül át a vezérlés, ha az egyezés igaz volt.
	\item Soha ne dolgozzunk aktívan működő hozzáférési listával!
	\item Először a logikai utasításokat felvázoló megjegyzéseket készítsük el szövegszerkesztővel, a tényleges végrehajtó műveleteket csak ezt követően írjuk meg.
	\item Az új sorok mindig a hozzáférési lista végére kerülnek. A no access-list x parancs a teljes listát törli. Számozott ACL-ek sorainak egyenként való hozzáadására vagy eltávolítására nincs lehetőség.
	\item Az IP alapú hozzáférési listák a célállomás elérhetetlenségét jelző ICMP-üzenetet küldenek az elutasított csomagok forrásainak, majd a bitszemetesbe dobják a csomagokat.
	\item Hozzáférési lista eltávolítását mindig körültekintően kell végezni. Ha a hozzáférési lista aktív interfészre vonatkozik, és eltávolítjuk, akkor az IOS verziójától függően alapértelmezett tiltó szabály léphet érvénybe az interfészen, ami a forgalom teljes leállását okozza.
	\item A kimenő szűrők nem vonatkoznak a helyi forgalomirányítóról kiinduló forgalomra.
\end{enumerate}

\subsubsection{Normál ACL-ek}
A normál ACL-ek az irányítandó IP-csomagok forráscímét ellenőrzik. Az összehasonlítás a
hálózati, alhálózati és állomáscím alapján egy egész protokollkészlet számára eredményez
engedélyezést vagy tiltást.\\
Például a Fa0/0 interfészen keresztül beérkező csomagok forráscímét és protokollját egyaránt
ellenőrizzük. Ha mindkettő engedélyezve van, a csomagok a forgalomirányítón keresztülvalamelyik kimenő interfészre kerülnek. Ha tiltva vannak, akkor eldobásuk a bejövő
interfészen történik meg.

A globális konfigurációs mód \verb|access-list| parancsának normál változatával normál, 1 és 99
közötti számú ACL definiálható. A Cisco IOS Software Release 12.0.1 és újabb változataiban
a sorszám 1300 és 1999 között is lehet, így akár 798 normál ACL-t is készíthetünk. Ezt az
újabb tartományt kibővített IP ACL-nek nevezzük.

\begin{verbatim}
	access-list 2 deny 172.16.1.1
	access-list 2 permit 172.16.1.0 0.0.0.255
	access—list 2 deny 172.16.0.0 0.0.255.255
	access-list 2 permit 172.0.0.0 0.255.255.255
\end{verbatim}

\begin{itemize}[nosep]
	\item Hozzáférési lista tartományok: I - 99 és 1300 - 1999
	\item Szürés csak az IP-forráscim alapján
	\item Helyettesítö maszkok
	\item A célhoz legközelebbi portra kell alkalmazni
\end{itemize}

Vegyük észre, hogy az első ACL-utasításnál nincs megadva helyettesítő maszk. Ilyenkor a
forgalomirányító az alapértelmezett 0.0.0.0 maszkot használja, vagyis vagy a teljes címnek
egyeznie kell, vagy az ACL ezen sora nem fog illeszkedni, és a forgalomirányító az ACL
következő sorára lép tovább.

A normál ACL-utasítások teljes szintaxisa a következő:\\
{\small\verb+Router(config)#access-list hozzáférési-lista-száma {deny | permit | remark} forrás [forrás-helyettesítő-maszkja] [log]+}

A \verb|remark| (megjegyzés) kulcsszó az ACL-eket könnyebben érthetővé teszi. A megjegyzések
hossza nem haladhatja meg a 100 karaktert.
Példa:
\begin{verbatim}
	access-list 1 remark Csak Jones állomását engedélyezzük
	access-list 1 permit 171.69.2.88
\end{verbatim}

Normál ACL-t törölni a parancs no változatával lehet. Ennek szintaxisa:\\
\verb|Router(config)#no access-list hozzáférési-lista-száma|

Az ip access-group parancs hozzákapcsol egy normál ACL-t egy interfészhez:
\verb+Router(config)#ip access-group {access-list-number | access-list-name} {in | out}+

A táblázat a szintaxisban használt paraméterek leírásait tartalmazza:\\
\begin{tabularx}{\linewidth}{l|X}
	Paraméter & Leírás\\
	\hline
	access-list-nurnber & Az ACL azonosító száma. Decimális szám 1 -99 (normál IP ACL) és 1300 - 1999 (kibővitett IP ACL).\\[1pt]
	deny & Megtagadja a hozzáférést, ha a feltételek teljesülnek.\\[1pt]
	permit & Engedélyezi a hozzáférést, ha a feltételek teljesülnek.\\[1pt]
	remark & A remark (megjegyzés) parancs használata a listák könnyebb megértését és megkeresését segíti.\\[1pt]
	source & Annak a hálózatnak vagy állomásnak a címe, ahonnan a csomagot elküldik. A forrás kétféleképpen adható meg:
		\begin{enumerate}[nosep]
		\item 32 bites cím megadása négy részből álló, pontokkal elválasztott decimális formátumban.
		\item az any kulcsszó a forrás rövidítése: source-wildcard of 0.0.0.0 255.255.255.55.
		\end{enumerate}\\
	source-wildcard & (Opcionális) A forrásra alkalmazandó helyettesítő bitek. A forráshelyettesítő maszkja kétféleképpen adható meg:
	\begin{enumerate}[nosep]
	\item 32 bites cím megadása négy részből álló, pontokkal elválasztott decimális formátumban. A figyelmen kívül agyandó bitpozíciókba 1-et kell írni.
	\item Az any kulcsszó a 0.0.0.0 255.255.255.255 értékű forrás és forráshelyettesítő maszk rövidítése.
	\end{enumerate}\\
	log & (Opcionális) A bejegyzésnek megfelelő csomagról egy tájékoztató célú naplózási üzenet jut a konzolra, (A konzolra küldött naplózási üzenetek részletessége a logging console paranccsal vezérelhető.)Az üzenet tartalmazza az ACL számát, a forráscímet, a csomagok számát és azt, hogy a csomagot engedélyezték-e vagy letiltották. Az üzenet az első megfelelő csomag megtalálásakor generálódik, ezután ötpercenként, megmutatva azoknak a csomagoknak a számát is, amelyek a megelőző öt percben lettek engedélyezve vagy elutasítva.
\end{tabularx}

\subsubsection{Kiterjesztett ACL-ek}
A kiterjesztett ACL-eket a normál ACL-eknél gyakrabban használjuk, mivel szélesebb körű
ellenőrzést tesznek lehetővé.\\
A kiterjesztett ACL-ek a csomagok forrás- és célcímét egyaránt ellenőrzik, illetve a
protokollok és a portszámok egyeztetésére is alkalmasak. Ezek a lehetőségek nagyobb
szabadságot biztosítanak az ACL által vizsgált adatok körülhatárolására. A csomagok
engedélyezése és tiltása forrás, cél, protokolltípus és portszám alapján egyaránt történhet.
Egy kiterjesztett ACL például engedélyezheti az elektronikus levelezést a Fa0/0 interfészről
megadott S0/0 célok felé, miközben tilthatja a fájlátviteleket és a webböngészést. A csomagok
eldobásakor bizonyos protokollok egy visszhangcsomagot küldenek a forrásnak, jelezve, hogy
a cél nem érhető el.

Egy-egy ACL-hez több utasítás is konfigurálható. Az utasítások mindegyikének ugyanazt a
hozzáférési lista számot kell tartalmaznia, így tud hivatkozni az azonos ACL-en belüli
utasításokra. Feltételutasításból tetszőleges számú adható meg, az ilyen utasítások
mennyiségét csak a forgalomirányító memóriájának nagysága korlátozza. Természetesen
minél több az utasítás, annál nehezebb az ACL megértése és kezelése.

A kiterjesztett ACL-utasítások szintaxisa meglehetősen hosszadalmas is lehet, akár a teljes
terminálablakot is kitöltheti. A helyettesítéseknél ugyancsak mód nyílik a host vagy az any
kulcsszó használatára a parancsokon belül.

A kiterjesztett ACL-utasítások végén az opcionálisan megadható TCP vagy UDP
portszámokkal tovább pontosíthatók a szabályok. A TCP/IP protokollkészlet jól ismert
portszámai az ábrán láthatók.\\
{\centering
\includegraphics[width=\linewidth]{fig/16-extACL-TCPIP_protocolls}}

A kiterjesztett ACL-ek logikai műveleteket – egyenlő (equal, eq), nem egyenlő (not equal,
neq), nagyobb mint (greater than, gt), kisebb mint (less than, lt) – is képesek végezni a
megadott protokollokon. A kiterjesztett ACL-ek hozzáférési lista száma 100 és 199 között
lehet. (Az újabb IOS-változatoknál a sorszám a 2000–2699 tartományba is tartozhat.)

Az \verb|ip access-group| paranccsal egy meglévő kiterjesztett ACL köthető hozzá egy interfészhez.
Ne feledjük, hogy interfészenként, irányonként és protokollonként csak egy ACL adható meg!
A parancs formátuma: \verb+Router(config-if)#ip access-group hozzáférési-lista-száma {in | out}+

%%----------------------------------------------------------------------------
\subsection{Bevezetés a Cisco eszközök programozásába 2}
%----------------------------------------------------------------------------
\subsubsection{A forgalomirányítási táblázatok felépítése, statikus és dinamikus routing összehasonlítása.}

%\section[Vállalati információs rendszerek]{Vállalati információs rendszerek specializáció}
%%----------------------------------------------------------------------------
\section{Távközlő hálózatok}
{\footnotesize Fizikai jelátviteli közegek. Forráskódolás, csatornakódolás és moduláció. Csatornafelosztás és multiplexelési technikák. Vezetékes és a mobil távközlő hálózatok. Műholdas kommunikáció és helymeghatározás.}
%----------------------------------------------------------------------------
\subsection{Fizikai jelátviteli közegek.}

\subsection{Forráskódolás, csatornakódolás és moduláció.}

\subsection{Csatornafelosztás és multiplexelési technikák.}

\subsection{Vezetékes és a mobil távközlő hálózatok.}

\subsection{Műholdas kommunikáció és helymeghatározás.}


%%----------------------------------------------------------------------------
\subsection{Hálózatok hatékonyságanalízise}
%----------------------------------------------------------------------------
\subsubsection{Markov-láncok, születési-kihalási folyamatok.}

\subsubsection{A legalapvetőbb sorbanállási rendszerek vizsgálata.}

\subsubsection{A rendszerjellemzők meghatározásának módszerei, meghatározásuk számítógépes támogatása.}

%%----------------------------------------------------------------------------
\section{Adatbiztonság}
{\footnotesize Fizikai, ügyviteli és algoritmusos adatvédelem, az informatikai biztonság szabályozása. Kriptográfiai alapfogalmak. Klasszikus titkosító módszerek. Digitális aláírás, a DSA protokoll.}
%----------------------------------------------------------------------------
\subsection{Fizikai, ügyviteli és algoritmusos adatvédelem, az informatikai biztonság szabályozása.}
\begin{definition}[Adatvédelem]
	azon fizikai, ügyviteli és algoritmikus eszközök együttes felhasználását értjük, amelyek segítségével a véletlen adatvesztések és szándékos adatrongálódások és információ kiszivárogtatások megelőzhetők, vagy jelentős mértékben megnehezíthetők
\end{definition}
\paragraph{Fizikai adatvédelem} két lényegi dolgot takar: Egyrészt biztosítani kell az optimális, de legalább a még elfogadható \textbf{üzemi körülményeket} (hőmérséklet, páratartalom, por, tartalék alkatrészek stb.), másrészt pedig a szükséges \textbf{vagyonvédelmi intézkedésekről} sem szabad megfeledkezni. Például: Villamos hálózat helyes kialakítása; Szünetmentes tápegységek használata; Megfelelő szerverterem kialakítása(klimatizálás, füstérzékelés, árnyékolás,\dots); Megfelelő adattároló eszközfajták használata; Betörésvédelem.

\paragraph{Ügyviteli adatvédelem} a folyamatok szabályozásának, a szabályzatoknak a kialakítása és védelme. A fizikai adatvédelem önmagában ugyanis nem elegendő. Példa: hiába zárjuk be a szerverszoba ajtaját, ha a portás beengedi azt, aki egy szerszámos táskával érkezvén arról tájékoztatja, hogy ’zsírozni kell a switcheket’ (social hacking/engineering). Tehát szükséges \textbf{pontosan szabályozni}, hogy \textbf{ki}, \text{mikor}, \textbf{mit} és \textbf{hogyan} tehet meg, illetve nem tehet meg. Szükség van \textbf{informatikai biztonsági szabályzatra} is, amely mindezt egységes módon áttekinti. Megfelelő felhasználó menedzselési rendet kell kialakítani, hogy a felhasználók, hozzáférési jogosultságaik, munkájukból adódó szerepköreik kezelése összhangba hozható legyen. Példák: Feladat- és jogkörök szétválasztása; Hozzáférések és tevékenységek regisztrálása; Személyazonosítás; Hatáskörök és felelősségek szétválasztása vagy átlapolása.

\paragraph{Algoritmikus adatvédelem} feladata olyan programok és eljárások alkalmazása, amelyek segítik az előző két terület feladatait és létrehozzák azokat a számítógépes védelmi funkciókat, amik ezen a területen meggátolják az adatokhoz való illetéktelen hozzáférést és módosítást. Példák: Hálózati azonosítás; \textbf{Titkosítás}; Behatolásvédelem; Automatikus adatmentés; Többforrásos adattárolás.

\paragraph{Az informatikai biztonsági szabályrendszer szükségessége:} (1) az adatok egyre inkább elektronikus formában jelennek meg; (2) a Szervezetek informatika nélkül működésképtelenek; (3) az informatikai függőség egyre nagyobb; (4) ugyanakkor a fenyegetettség is egyre növekszik; (5) az üzletfolytonossághoz kritikus fontosságú; (6) a kárpotenciál és a kockázati tényezők szervezetenként eltérőek lehetnek!

\subsubsection{Biztonsági célok}
Alapkövetelmények, amelyek teljesülése az üzemszerű használhatóság előfeltétele:
\begin{enumerate}
	\item rendelkezésre állás (elérhetőség az arra jogosultak számára)
	\item sértetlenség (valódiság)
	\item bizalmasság (jellegtől függően)
	\item nyomon követhetőség, hitelesség
	\item Biztosítékok (az információs rendszer teljességére nézve)
\end{enumerate}
Ez alapján úgy lehet meghatározni az Informatikai Biztonság fogalmát, hogy az akkor áll fenn, ha az információs rendszer védelme az alapkövetelmények szempontjából
\begin{itemize}
	\item \textbf{zárt:} minden fontos fenyegetést figyelembe vesz
	\item \textbf{teljes körű:} a rendszer összes elemére kiterjedő
	\item \textbf{folyamatos:} megszakítás nélküli, az időben változó körülmények ellenére is
	\item \textbf{kockázatarányos:} a feltehető kárérték és a kár valószínűségének szorzata nem haladhat meg egy előre rögzített küszöböt, amely egy üzleti döntés.
\end{itemize}

\subsection{Kriptográfiai alapfogalmak.}
\begin{center}
	\includegraphics[width=0.7\linewidth]{fig/14-Crypto_scheme}
\end{center}
\begin{definition}[titkostási rendszer]
	A $(\mathcal{P}, \mathcal{C}, \mathcal{K},Enc,Dec,Key)$ hatost titkosítási rendszernek (sémának) nevezzük, ahol:
	\begin{itemize}[nosep]
		\item $\mathcal{P}, \mathcal{C}, \mathcal{K}$ a nyílt és titkos üzenetek, valamint lehetséges kulcsok véges, nemüres halmaza ($1<|\mathcal{P}|,|\mathcal{C}|,|\mathcal{K}| < \infty$).
		\item $Enc:\; \mathcal{K}\times\mathcal{P} \to c\;$ egy titkosító függvény ($c \in \mathcal{C}$: titkosított üzenet)
		\item $Dec:\; \mathcal{K}\times\mathcal{C} \to m\;$ egy visszafejtő függvény ($m \in \mathcal{P}$: nyílt üzenet)
		\item $Key \subseteq \mathcal{K}\times\mathcal{K}$ kulcspárok halmaza
	\end{itemize}
\end{definition}
\begin{definition}[Korrekt visszafejtés]
	A titkosító rendszer korrekt visszafejtést biztosít, ha minden $(k_E,k_D)\in\mathcal{K},\; m\in\mathcal{P}$ esetén
	$$Dec(k_D,Enc(k_E,m)) = m$$
\end{definition}
\begin{note}~\\
	\begin{itemize}[nosep]
		\item $(k_E,k_D)$ titkosító-visszafejtő kulcspár
		\item $c = Enc(k_E,m)$
		\item definíció alapján, rögzített $(k_E,k_D)$ esetén az $Enc_{k_E}(m) = Enc(k_E,m)$ függvény injektív, azaz:
		$$Enc_{k_E}(m_1) = Enc_{k_E}(m_2) \Leftrightarrow m_1 = m_2$$
	\end{itemize}
\end{note}
\begin{definition}[Teljes titkosítási rendszer]
	A titkosítási rendszert teljesnek nevezzük, ha minden $(k_E,k_D)$ esetén az $Enc_{k_E}(m)$ függvény szürjektív, azaz minden $c\in\mathcal{C}$ titkos üzenet esetén létezik $m\in\mathcal{P}$ nyílt üzenet, amelyikre $Enc_{k_E}(m) = c$. Ebben az esetben $|\mathcal{P}| = |\mathcal{C}|$
\end{definition}
Természetes elvárás egy titkosítási rendszerrel szemben, hogy ha vannak $k_{E1} \neq k_{E2}$ kulcsok, akkor $Enc_{k_{E1}}(m) \neq Enc_{k_{E2}(m)}$. Továbbá ha $(k_E,k_{D1}), (k_E,k_{D2})\in Key$, akkor $k_{D1} = k_{D2}$.
\begin{definition}[Teljes kulcstér]
	Legyen $|\mathcal{P}| = |\mathcal{C}|$ . A kulcsteret teljesnek nevezzük, ha minden $f:\;\mathcal{P} \to \mathcal{C}$ bijektív leképezéshez létezik $k_E \in \mathcal{K}$, amelyikre $Enc_{k_E} = f$.
\end{definition}
\begin{description}[nosep]
	\item[Kriptográfia] A kulcs alapú titkosítási technikák tudománya
	\item[Kriptanalízis] A kulcs alapú titkosítási technikák feltörésének, támadásának tudománya
	\item[Kriptológia] Kriptográfia + Kriptanalízis
	\item[Nyílt szöveg (plain text)] Az eredeti, mindenki által értelmezhető információ
	\item[Titkosított szöveg (ciphertext)] A titkosított információ
	\item[Titkosítás (encryption)] Eljárás, mely során az információt a birtokosa titkossá nyilvánít.
	\item[Visszafejtés (decryption)] A titkosított információból az eredeti visszaállítása.
	\item[Rejtjelezés] Adott módszer a nyílt szöveg kódolására (és visszafejtésére)
	\item[Kulcs (key)] Az az információ, amelynek segítségével a rejtjelezés történik. Két típus: nyilvános és titkos kulcs.
	\item[Szimmetrikus titkos. rendszer] Ha van $(k_E,k_D)$ kulcspárunk, $k_E$ nyílvános kulcs megegyezik a $k_D$ titkos kulccsal vagy $k_D$ polinomiális időben kiszámolható $k_E$-ből.
	\item[Aszimmetrikus titkos. rendszer] A tetszőlegesen választott $k_E$ és $k_D$ kulcspárok annyira különböznek, hogy nem létezik polinomiális időbonyolultságú algoritmus, mely $k_E$-ből kiszámolja $k_D$-t
\end{description}

\subsection{Klasszikus titkosító módszerek.}
A számítógép megjelenése előtt (before computers, BC) használt rendszereket, történelmi vagy klasszikus titkosítási rendszereknek nevezzük. Nem egyértelműen definiált, de nagyjából a számítógéppel könnyen támadható rendszereket nevezzük így. Jellemzően élő nyelvi szöveg titkosítására használták. Főbb típusai:
\begin{easylist}[enumerate]
	# Helyettesítéses titk. rendszerek: Az üzenet egy betűjét más betűre vagy szimbólumra cseréli
	## monoalfabetikus: a csere transzformációja változatlan
	## polialfabetikus: az egymás után következő betűket más-más transzformációval titkosítjuk
	# Permutációs titk. rendszerek: az üzenet betűit más sorrendben írjuk fel (mint egy anagramma, de a titkos szöveg nem feltétlenül értelmes)
\end{easylist}
\paragraph{Betűk kicserélése absztrakt szimbólumokra} Helyettesítéses, monoalfabetikus
\paragraph{Caesar--titkosítás} Monoalfabetikus, helyettesítéses.\\
$Enc_k(m) = k+m (mod\; 26)$\\
$Dec_k(c) = c-k (mod\; 26)$
\paragraph{Affin titkosítás} Monoalfabetikus\\
$k_E = (\alpha,\beta)\; k_D = (\alpha^{-1},\beta)\quad lnko(\alpha,26) = 1$\\
$Enc = \alpha \cdot m + \beta (mod\; 26)$\\
$Dec = \alpha^{-1} \cdot (c - \beta) (mod\; 26)$
\paragraph{Vigènre-titkosítás} polialfabetikus, a kulcs valamilyen karaktersorozat. A Vernam--rejtjelező (OTP) hasonló hozzá, azonban az kulcs hossza megegyezik az üzenettel. Bináris esetben a Vernam--rejtjelező az üzeneten és a kulcson bitenkénti XOR műveletet végrehajtva állítja elő a titkos üzenetet.\\
$c[i] = m[i] + k_E[i\; mod\; n] (mod\; 26)$\\
$m[i]= c[i]- k_E[i\; mod\; n] (mod\; 26)$

\paragraph{Enigma} Polialfabetikus. Elektromechanikus rejtjelező, ahol a kulcsot a gépben található tárcsák kezdőpozíciója jelentette.

\subsection{Modern titkosítási rendszerek}
\subsubsection{Folyamtitkosítás}
Bitsorozat titkosítása bitenként. Legegyszerűbb módszer a One-Time-Pad titkosítás, ahol titkosított bitsorozat, az eredeti sorozat egy álvéletlen bitsorozattal történő XOR művelet eredménye $c[i] = m[i]XORk[i]$. Shannon bebizonyította, hogy egy teljesen véletlen kulcsfolyam esetén a titkos üzenet elméletileg feltörhetetlen. Tehát ennek a titkosításnak az erőssége az álvéletlenszám-generátor erősségén múlik. Alkalmazásakor a mesterkulcsot a kulcsgenerátor kezdeti állapota jelenti. A kulcsfolyam-generálásnak több variánsa is létezik:
\paragraph{Lineárisan visszacsatolt léptető regiszter -- LFSR}
A kulcsfolyam $l$ bitjein egy $a = (a_0,a_1,\dots,a_{l-1})$ értékkel bitenkénti ÉS műveletet hajtunk végre, majd a részeredményen XOR műveletet hajtunk végre. Így kapjuk meg a $k_i$ kulcsot:
$$ k_i = a_{l-1}k_{i-1}\oplus a_{l-2}k_{i-2}\oplus\dots\oplus a_{0}k_{i-l}$$
\begin{center}
	\includegraphics[width=0.7\linewidth]{fig/14-LFSR}
\end{center}
Az $a_i$ és kezdeti $k_i$ értékei adják a generátor kezdeti állapotát. Fontos, hogy ezeket az értékeket jól válasszuk meg, mert ezen múlik az álvéletlen sorozat minősége. Ez leginkább két dolgot takar: sorozat eloszlása (fontos, hogy egyenletes legyen) és a periódus hossza (minél hosszabb annál jobb).
\paragraph{Lineárisan rekurzív sorozat -- LSR}
Az LFSR általánosítása. C/C++ rand() függvénye implementálja.\\
$a = 31835, b = 1906, k_0 = 41$\\
$k_i = a \dots k_{i-1} + b (mod 2^{15})$\\
periódusa $2^{15}$
\paragraph{Blum--Blum--Shub generátor -- BBS}
$p, q$ nem közeli nagy prímek: $m = p\times q$ és $a_0$ úgy választjuk, hogy $lnko(a_0,m)=1$.\\
$a_i = a_{i-1}^2 \;(mod\; n)$
$k_i = a_i(mod 2)$ vagy például $a_i$ paritása.\\
A következő $a_i$ az előzőekből exponenciális időben számolható ki. Periódusa kb. $p\cdot q$, ha $lnko(p-1,q-1)$ kicsi, ezért kell, hogy p és q ne legyenek egymáshoz közel.
\paragraph{RC4}
\begin{multicols}{2}
Egy S kezdeti tömb feltöltése és összekeverése után a következő bitet az  %\ref{fig:14-rc4}~
ábrán látható módon kapjuk meg, (i = (i + 1) mod 256 j = j + S i mod 256).
Felhasználja az SSL/TLS, és a WEP protokoll.
\begin{center}
	\includegraphics[width=\linewidth]{fig/14-RC4}
%	\caption{A következő elem (K) meghatározása az S tömbből. Minden kiolvasás előtt S[i] és S[j] értéket felcseréljük a tömbben}
%	\label{fig:14-rc4}
\end{center}
\end{multicols}

\subsubsection{DES}
\begin{multicols}{2}
\begin{easylist}[itemize]
# 1975-ben jelent meg, a LUCIFER titkosítási rendszer egyik változata.
# Ma már nem létezik szabványként, mert feltörték, helyette a Triple DES-t ajánlott használni. 
# Szimmetrikus, blokktitkosítási rendszer, a blokk mérete 64 bit. Ha ennél nagyobb méretű üzenetet kell titkosítani, akkor azt fel kell tördelni valamilyen blokktitkosítási módszerrel.
# Matematikai alapja a Fiestel struktúra.
# A Kulcs 64 bites, amelyből 56 bit random és 8 a random bitek alapján meghatározott. 
# S-BOX-ok használata a titkosítás és a visszafejtés során.
# 16 körből áll az algoritmus, ezért 16 körkulcs kerül legenerálásra az eredeti kulcsból. Visszafejtéskor a körkulcsokat fordított sorrendben kell alkalmazni.
# Jellemzők:
## az S-BOX-okat kivéve minden lépése az algoritmusnak lineáris
## az S-BOX-ok eredete nem ismert, de a feltételeknek megfelelnek.
## nagy problémája a kis kulcstér, ezért kell inkább TDES-t alkalmazni ma már.
\end{easylist}
\end{multicols}
\begin{figure}[h]
	\centering
	\includegraphics[width=0.2\linewidth]{fig/14-DES_Feistel_net}
	\includegraphics[width=0.7\linewidth]{fig/14-DES_Feistel_func}
	\caption{A feistel-háló és a feistel függvény (F - Feistel függvény; IP,FP - egymással ellentétes permutációk; E - kiterjesztő függvény; S - S-boxok; P - permutáció;sk - a mesterkulcsból képzett 16 részkulcs)}
	\label{fig:14-desfeistelnet}
\end{figure}

\subsubsection{AES}
\begin{multicols}{2}
\begin{easylist}[itemize]
# Szimmetrikus, blokktitkosítási rendszer. Kifejlesztői: RIJNDAEL (Vincent Rijmen, Joan Daemen).
# A (T)DES-nél hatékonyabb titkosításra képes, mivel matematikai alapja nem Fiestel struktúra, itt a bitek nem csak keverednek, de meg is változnak!
# 128, 192, 256 bites blokkhossz és kulcshossz kezelésére képes bármilyen kombinációban.
# Bonyolult matematikai háttere van, a bájtokat {0,1} együtthatós polinomként kezeli.
# Tervezésekor fontos volt az ismert támadásokkal szembeni védelem, a gyorsaság és tömör kódolhatóság
# A belső algoritmus ciklusainak száma a blokkhossz és a kulcshossz függvényében lehet 10, 12, 14
# Visszafejtésnél az összes transzformáció inverzét kell végrehajtani.
\end{easylist}
Működése 3 fő részből és 4 műveletből áll:
\begin{easylist}[enumerate]
# Előkészítés:
## AddRoundKey
# Ismételt rész:
## SubBytes
## ShiftRows
## MixColumns
## AddRoundKey
# Utófeldolgozás:
## SubBytes
## ShiftRows
## AddRoundKey
\end{easylist}
\begin{description}[nosep]
	\item[AddRoundKey] XOR művelet az állapotmátrix és a megfelelő sorszámú kulcs között
	\item[SubBytes] speciális S-boxokat használ, mely egy nemlineáris helyettesítési kódolást valósít meg
	\item[ShiftRows] a sorokat ciklikusan eltolják. Az 1. sor 0, a 2. sor 1,\dots, 4. sor 3 hellyel
	\item[MixColumns] oszloponként egy lineáris transzformáció:
	$\begin{matrix}
	2 & 3 & 1 & 1\\
	1 & 2 & 3 & 1\\
	1 & 1 & 2 & 3\\
	3 & 1 & 1 & 2
	\end{matrix}$
\end{description}
\end{multicols}

\subsubsection{RSA}
\begin{easylist}[itemize]
# Feltalálói: Ron Rivest, Adi Shamir, Leonard Adleman, 1977-ben.
# Az első aszimmetrikus, PKI-t használó kriptográfiai rendszer. (NEM BLOKKTITKOSÍTÓ RENDSZER)
# A kommunikációban résztvevő mindkét félnek rendelkeznie kell egy Publikus és egy Titkos kulccsal (PK; SK), amelyeket a kommunikáció titkosságának biztosítása érdekében felhasználnak.
# Kulcsgeneráló algoritmus, PK=(n,e) és SK=(n,d)
## Véletlenül választunk két nagy prímet: p,q
## Kiszámoljuk: n=p*q
## Kiszámoljuk: \textphi(n)=(p-1)*(q-1)
## Választunk véletlen egy e-t, hogy 1 < e < \textphi(n) és (e,\textphi(n)) = 1
## Kiszámítjuk d-t, hogy 1 < d < \textphi(n) és e*d= 1 (mod \textphi(n))
# A PK és SK ismeretében a titkosító algoritmus: 	c = ENC\textsubscript{PK}(m) = m\textsuperscript{e} (mod n)
# A PK és SK ismeretében a visszafejtő algoritmus: 	m = DEC\textsubscript{SK}(c) = c\textsuperscript{d} (mod n)
# Az RSA feltörésének nehézségét a prím-faktorizáció nehézsége adja:  \{c,n,e\} $\leftarrow$ \{m\}  nehéz!
# p és q választásánál fontos, hogy hosszuk bitekben hasonló legyen, viszont ne legyenek túl közeliek
# e választásánál, annál jobb minél kisebb, általában 65537 szokott lenni.
\end{easylist}

\subsection{Digitális aláírás, a DSA protokoll.}
A digitális aláírás lényege:
\begin{easylist}[enumerate]
# Letagadhatatlanság --- alkalmas az aláíró azonosítására 
# Hitelesítés --- más nem tudja létrehozni 
# Hamisíthatatlanság --- akár az aláírás, akár a dokumentum módosul, észrevehető.
\end{easylist}
A letagadhatatlanság és harmadik fél számára is elfogadható hitelesítés
alapvetően kétféle megoldással lehetséges:
\begin{easylist}[itemize]
# megbízható harmadik féllel (Trusted Third Party, TTP) (szigorúan véve nem DS)
# közvetlen digitális aláírás (nyilvános kulcsú titkosítással)
\end{easylist}
Digitális aláírás fajtái:
\begin{easylist}[itemize]
# RSA alapú --- Az aláíró a titkos kulcsával titkosítja az üzenetét vagy annak kivonatát. A hitelességet az adja, hogy egyedül az aláíró nyilvános kulcsa tudja visszafejteni a dokumentumot (vagy a kivonatot). Hátrányai:
## nem megbízható --> egzisztenciális hamisítás: egy tetszőleges s értéket választva előállíthatjuk az $ m \equiv s^e (n) $ üzenetet.
## alakíthatóság --> $(m_1,s_{m_1}),(m_2,s_{m_2})$ létező aláírásokból előállítható az $(m_1m_2,s_{m_1}s_{m_2})$ új aláírás
## Ezek a hátrányok nem jelentkeznek, ha az üzenet helyett annak kivonatát írjuk alá (hash-and-sign):\\
$S_m\equiv Hash(m)^d (n) \Rightarrow (m,S_m)$\\
ellenőrzés: $Hash(m) \equiv S_m^d (n)$
# ElGamal alapú --- Hátránya, hogy az aláírás kétszer akkora, mint a dokumentum
# DSA protokoll --- Digital Signature Algorithm. A digitális aláírás szabvány (DSS) felhasználja.
\end{easylist}

\subsubsection{DSA}
\begin{multicols}{2}
Kulcsgenerálás:
\begin{enumerate}[nosep]
	\item $p, q$ olyan prímek, hogy $q|p-1$ (p-1 q többszöröse)
	\item meghatározzuk $g$, melynek rendje $q$ moduló $p$, azaz $g^t \equiv 1 (p) \Leftrightarrow t= k\cdot q$
	\item x egy véletlen szám és $a \equiv g^x (p)$
	\item PK=(p,q,g,a), SK=(x)
\end{enumerate}
Aláírás:
\begin{enumerate}[nosep]
	\item $k$ véletlen szám kiválasztása
	\item $r \equiv (g^k\;(mod\;p))\; (q)$
	\item $t \equiv k^{-1}(Hash(m)+xr)\; (q)$
	\item $s_m = (r,t)$
\end{enumerate}
Ellenőrzés:
\begin{enumerate}[nosep]
	\item $v_1 \equiv Hash(m)\cdot t^{-1}\; (q)$
	\item $v_2 \equiv r\cdot t^{-1}\; (q)$
	\item Az aláírás rendben van, ha $r \equiv (q^{v_1}a^{v_2}\;(mod\;p))\;(q)$
\end{enumerate}
\end{multicols}
Az üzenet ellátható időbélyegzővel is. Az üzenet kivonatát átadjuk egy időbélyegző szolgáltatónak, aki hozzáfűz egy T időpontot és így aláírja a kivonatot. Ezután a bélyegzett üzenet így néz ki $(m,(m|T,s_{m|T}))$. Az ellenőrző biztosan tudja, hogy T időpillanatban már létezett a dokumentum.
%%----------------------------------------------------------------------------
\section{A RIP protokoll működése és paramétereinek beállítása (konfigurációja).}
%----------------------------------------------------------------------------

%%----------------------------------------------------------------------------
\section{Bevezetés a Cisco eszközök programozásába 1}
{\footnotesize A forgalomszűrés, forgalomszabályozás (Trafficfiltering, ACL) céljai és beállítása (konfigurációja) egy választott példa alapján.}
%----------------------------------------------------------------------------
\subsection{A forgalomszűrés, forgalomszabályozás (Trafficfiltering, ACL) céljai és beállítása (konfigurációja) egy választott példa alapján.}
A rendszergazdáknak meg kell találniuk annak a módját, hogy megakadályozzák a hálózat
jogosulatlan elérését, miközben a belső felhasználók számára lehetővé teszik a szükséges
szolgáltatások használatát. Bár a biztonsági eszközök, mint például a jelszavak, a visszahívó
berendezések és a fizikai biztonsági eszközök sok segítséget nyújtanak, gyakran nem
rendelkeznek az alapvető forgalomszűréshez szükséges rugalmassággal és a rendszergazdák
igényeinek megfelelő vezérlési lehetőségekkel. Előfordulhat például a hálózati rendszergazda
lehetővé kívánja tenni a felhasználók számára az internet elérését, de nem akarja, hogy külső
felhasználók telnettel hozzáférhessenek a LAN-hoz.

A forgalomirányítók alapvető forgalomszűrési lehetőségeket biztosítanak, például hozzáférési
listákat (access control list, ACL) az internetes forgalom letiltásához. Az ACL engedélyező és
tiltó utasítások sorozata, melyek címekre vagy felsőbb rétegbeli protokollokra alkalmazhatók.
Ebben a modulban a normál és kiterjesztett ACL-ek használatáról, mint a hálózati forgalom
vezérlési módszeréről fogunk tanulni, továbbá arról, hogyan használhatók az ACL-ek
valamilyen biztonsági megoldás részeként.

Ezenkívül a fejezetben tippek, elgondolások, javaslatok és általános útmutatások is
szerepelnek az ACL-ek használatával, valamint a létrehozásukhoz szükséges parancsokkal és
beállításokkal kapcsolatban. Végül a fejezet példákat mutat a normál és a kiterjesztett ACL-
ekre és a forgalomirányítók interfészein történő alkalmazásukra.

Az ACL-ek akár egyetlen sorból is állhatnak, ha a cél egy adott állomásról származó
csomagok továbbításának engedélyezése, de tartalmazhatják szabályok és feltételek rendkívül
összetett halmazát is, amelyek pontosan megadják az engedélyezett forgalom jellegét, illetve
befolyásolják a forgalomirányító folyamatok teljesítményét.

\subsubsection{A hozzáférési listák működésének alapelvei}
Az ACL-ek feltétellisták, amelyek a forgalomirányító interfészén keresztülhaladó forgalomra
vonatkozóan lépnek érvénybe. Ezek a listák írják elő a forgalomirányító számára, hogy a
csomagokat fogadja el vagy utasítsa vissza. Az elfogadás és a visszautasítás meghatározott
feltételek alapján is történhet. Az ACL-ek segítségével lehetővé válik a forgalom felügyelete,
valamint a hálózatról kiinduló és az oda befutó elérések biztonságossá tétele.

ACL-ek minden irányított protokollhoz létrehozhatók, többek közt az internetprotokollhoz
(IP) és a hálózatközi csomagcseréhez (IPX) is. Az ACL-ek a forgalomirányítón konfigurálva
adott hálózat vagy alhálózat elérhetőségének vezérlésére használhatók.

Az ACL-ek úgy szűrik a hálózati forgalmat, hogy az irányított csomagok továbbítását vagy
eldobását írják elő a forgalomirányító interfészein. A forgalomirányító minden csomagot
megvizsgál annak meghatározásához, hogy azt az ACL-ben meghatározott feltételek szerint
továbbítani vagy eldobni kell. Az ACL-ek a forrás- és a célcímekre, a protokollokra és a
felsőbb rétegbeli portszámokra nézve adnak meg feltételeket.

Az ACL-eket protokollonként, irányonként és portonként kell létrehozni. Ha szabályozni
szeretnénk egy adott interfész forgalmát, akkor a rajta engedélyezett protokollok
mindegyikéhez külön ACL-t kell készíteni. Egy ACL egy interfészen egyszerre csak egy
irány forgalmát szabályozza. Minden irányhoz külön ACL-t kell létrehozni, egyet a kimenő,
egyet pedig a bejövő forgalomhoz. Végül minden interfészen több protokoll és irány is
definiálható. Ha például egy forgalomirányítónak két interfésze van, és ezeken IP, AppleTalk
és IPX alapú forgalom zajlik, akkor 12 külön ACL-t kell rajta létrehozni: minden
protokollhoz egy-egy ACL, szorozva kettővel a kimenő és a bejövő forgalom miatt, szorozva
kettővel, vagyis a portok számával.

ACL-eket többek között a következő okok miatt szokás létrehozni:
\begin{enumerate}[nosep]
	\item Korlátozható a hálózat forgalma, és növelhető a teljesítménye.
	\item Biztosítják a forgalom szabályozását.
	\item Az útvonalfrissítések továbbítása is korlátozható. Ha a hálózati környezet miatt nincs szükség a frissítésekre, sávszélességet lehet megtakarítani.
	\item Alapszintű hálózati hozzáférés-szabályozást biztosítanak. Az ACL-ek engedélyezhetik például a hálózat egy részének elérését egy állomás számára, és megtilthatják egy másiknak. Lehetséges például, hogy az A állomás hozzáférhet a személyzeti osztály hálózatához, míg a B állomásnak ezt megtiltjuk.
	\item Eldönthetjük, hogy milyen típusú forgalom továbbítódjon és milyen törlődjön a forgalomirányító interfészein. Például engedélyezhetjük az elektronikus levelek továbbítását, de a telnetet tilthatjuk.
	\item A rendszergazdák szabályozhatják, hogy az ügyfelek a hálózat mely részeihez férhetnek hozzá.
	\item Ki lehet választani bizonyos állomásokat, és számukra engedélyezni vagy megtiltani a hálózat egy részének elérését.
	\item A felhasználók számára engedélyezni vagy tiltani lehet bizonyos szolgáltatások, például az FTP vagy a HTTP elérését.
\end{enumerate}

Ha egy forgalomirányítón nincsenek ACL-ek megadva, akkor a forgalomirányítóba befutó csomagok mindegyike továbbhaladhat a hálózat bármely része felé.
\begin{figure}[h]
	\centering
	\begin{subfigure}{0.45\linewidth}
		\includegraphics[width=\linewidth]{fig/16-ACL}
		\caption{}
	\end{subfigure}
	\begin{subfigure}{0.45\linewidth}
		\includegraphics[width=\linewidth]{fig/16-ACL_list-schema}
		\caption{}
	\end{subfigure}
	\caption{}
\end{figure}

\subsubsection{ACL-ek létrehozása}
Az ACL-ek létrehozása globális konfigurációs módban történik.
Számos különböző típusú ACL létezik, így normál, kiterjesztett, IPX, AppleTalk stb. ACL.
Amikor egy forgalomirányítón ACL-eket konfigurálunk, akkor egy-egy szám
hozzárendelésével mindegyiket egyértelműen azonosítanunk kell. A szám megadja az ACL
típusát; ebből következően az adott típushoz tartozó értéktartományba kell esnie:\\
\begin{tabular}{l|l}
	\hline 
	\rule[-1ex]{0pt}{2.5ex} Protokoll & Tartomány \\ 
	\hline 
	\rule[-1ex]{0pt}{2.5ex} IP & 1-99, 1300-1999, 2000-2699 \\ 
	\hline 
	\rule[-1ex]{0pt}{2.5ex} Kiterj. IP & 100-199, 2000-2699 \\ 
	\hline 
	\rule[-1ex]{0pt}{2.5ex} AppleTalk & 600-699 \\ 
	\hline 
	\rule[-1ex]{0pt}{2.5ex} IPX & 800-899 \\ 
	\hline 
	\rule[-1ex]{0pt}{2.5ex} Kiterj. IPX & 900-999 \\ 
	\hline 
	\rule[-1ex]{0pt}{2.5ex} IPX szolgáltatáshirdetés (SAP) & 1000-1099 \\ 
	\hline 
\end{tabular}\\

Miután beléptünk a megfelelő parancsmódba és eldöntöttük, hogy milyen típusú listát
szeretnénk létrehozni, az \verb|access-list| paranccsal, illetve a szükséges paraméterek segítségével
kell megadnunk a hozzáférési lista utasításait.\\
A hozzáférési listák létrehozása az első lépés.\\
A második végrehajtandó művelet a listák hozzárendelése a megfelelő interfészekhez.\\

Az ACL definiálása a következő paranccsal:\\
\verb|Router(config) #access-list access-list-number {permit / deny test-conditions}|

\paragraph{1. lépés:} Az ACL-t egy globális utasítás azonosítja. A normál IP-címekhez az 1-99 tartomány van
fenntartva. Ez a szám mutatja az ACL típusát. A Cisco lOS 11 .2-es verziójától kezdődően az
ACL-ekhez nem csak szám, de név is rendelhető, például oktatasi\_csoport. A globális ACL
utasítás permit illetve deny kifejezése adja meg, hogy a Cisco lOS szoftver hogyan kezelje a
tesztfeltételeket kielégítő csomagokat. A permit általában azt jelenti, hogy a csomag
használhat egy vagy több - később meghatározandó - interfészt. A záró kifejezés(ek) az ACL
állítás által használandó tesztfeltételeket adják meg.

\paragraph{2. lépés:} Ezután alkalmazni kell az ACL-eket egy interfészre az \verb|access-group| paranccsal.\\
Példa: \verb|Router (config-if) #{protocol) access-group access-list number|\\
A hozzáférési lista számával azonosított összes ACL utasítás hozzárendelődik egy vagy több
interfészhez. Az ACL tesztfeltételeket kielégítő csomagok a hozzáférési csoport bármely
interfészét használhatják.

TCP/IP használata esetén az ACL-eket egy vagy több interfészhez lehet hozzárendelni.
Az \verb|ip access-group| parancs segítségével a bejövő vagy a kimenő forgalom szűrésére
állíthatjuk be.

\begin{verbatim}
	Router(config)#access-list 2 deny 172.16.1.1
	Router(config)#access-list 2 permit 172.16.1.0 0.0.0.255
	Router(config)#access-list 2 deny 172.16.0.0 0.0.255.255
	Router(config)#access-list 2 permit 172.0.0.0 0.255.255.255
	Router(config) #interface ethernet 0
	Router(config)#ip access-group 2 in
\end{verbatim}

Az access-group parancsot interfészkonfigurációs módban kell kiadni.
Amikor egy ACL-t hozzárendelünk egy interfészhez, akkor ki kell választanunk, hogy a
bejövő vagy a kimenő forgalomra vonatkozzon. A szűrés tehát az adott interfészre beérkező
és a róla távozó csomagokra vonatkozhat. Annak megállapításához, hogy az ACL a bejövő
vagy a kimenő forgalmat szűrje, az egyes interfészeket a forgalomirányító belsejéből kell
szemlélnünk. Ezt a szemléletet mindvégig meg kell őrizni. A valamilyen interfészen keresztül
beérkező forgalmat bejövő ACL, a kimenő forgalmat pedig kimenő ACL alapján szűrjük.
A számozott ACL-t létrehozása után hozzá kell rendelni egy interfészhez.
Számozott ACL-utasításokat tartalmazó ACL nem módosítható. Előbb törölnünk kell a 
\verb|no access-list lista-szám| paranccsal, majd újra be kell vinnünk a parancsokat.
(\verb|Router(config)#no access-list 2|)\\
Az ACL-ek létrehozásakor és életbe léptetésekor a következő alapvető szabályokat kell
betartani:
\begin{enumerate}[nosep]
	\item Irányonként és protokollonként egy ACL-t kell létrehozni.
	\item A normál hozzáférési listákat a célhoz a lehető legközelebb kell alkalmazni.
	\item A kiterjesztett hozzáférési listákat a forráshoz a lehető legközelebb kell alkalmazni.
	\item A kimenő és a bejövő jelzőket úgy kell használni, mintha a forgalomirányító belsejéből néznénk a portokat.
	\item Az utasítások feldolgozása sorban, a lista tetejétől az alja felé haladva történik, amíg a forgalomirányító egyezést nem talál. Ha nincs egyezés, a forgalomirányító eldobja a csomagot.
	\item Minden hozzáférési lista alján egy implicit deny any (mindent letilt) szabály található. Ez szabály nem jelenik meg az utasításlista alján.
	\item A hozzáférési listák utasításait a specifikusabbaktól az általánosabbak felé haladva kell megadni. Az egyes állomásokra vonatkozó tiltásokat kell először megadni, a csoportokra vonatkozó vagy általános szűrőket utolsóként kell elhelyezni.
	\item Elsőként az egyezési feltétel vizsgálata történik meg. Az engedélyező vagy tiltó részre kizárólag akkor kerül át a vezérlés, ha az egyezés igaz volt.
	\item Soha ne dolgozzunk aktívan működő hozzáférési listával!
	\item Először a logikai utasításokat felvázoló megjegyzéseket készítsük el szövegszerkesztővel, a tényleges végrehajtó műveleteket csak ezt követően írjuk meg.
	\item Az új sorok mindig a hozzáférési lista végére kerülnek. A no access-list x parancs a teljes listát törli. Számozott ACL-ek sorainak egyenként való hozzáadására vagy eltávolítására nincs lehetőség.
	\item Az IP alapú hozzáférési listák a célállomás elérhetetlenségét jelző ICMP-üzenetet küldenek az elutasított csomagok forrásainak, majd a bitszemetesbe dobják a csomagokat.
	\item Hozzáférési lista eltávolítását mindig körültekintően kell végezni. Ha a hozzáférési lista aktív interfészre vonatkozik, és eltávolítjuk, akkor az IOS verziójától függően alapértelmezett tiltó szabály léphet érvénybe az interfészen, ami a forgalom teljes leállását okozza.
	\item A kimenő szűrők nem vonatkoznak a helyi forgalomirányítóról kiinduló forgalomra.
\end{enumerate}

\subsubsection{Normál ACL-ek}
A normál ACL-ek az irányítandó IP-csomagok forráscímét ellenőrzik. Az összehasonlítás a
hálózati, alhálózati és állomáscím alapján egy egész protokollkészlet számára eredményez
engedélyezést vagy tiltást.\\
Például a Fa0/0 interfészen keresztül beérkező csomagok forráscímét és protokollját egyaránt
ellenőrizzük. Ha mindkettő engedélyezve van, a csomagok a forgalomirányítón keresztülvalamelyik kimenő interfészre kerülnek. Ha tiltva vannak, akkor eldobásuk a bejövő
interfészen történik meg.

A globális konfigurációs mód \verb|access-list| parancsának normál változatával normál, 1 és 99
közötti számú ACL definiálható. A Cisco IOS Software Release 12.0.1 és újabb változataiban
a sorszám 1300 és 1999 között is lehet, így akár 798 normál ACL-t is készíthetünk. Ezt az
újabb tartományt kibővített IP ACL-nek nevezzük.

\begin{verbatim}
	access-list 2 deny 172.16.1.1
	access-list 2 permit 172.16.1.0 0.0.0.255
	access—list 2 deny 172.16.0.0 0.0.255.255
	access-list 2 permit 172.0.0.0 0.255.255.255
\end{verbatim}

\begin{itemize}[nosep]
	\item Hozzáférési lista tartományok: I - 99 és 1300 - 1999
	\item Szürés csak az IP-forráscim alapján
	\item Helyettesítö maszkok
	\item A célhoz legközelebbi portra kell alkalmazni
\end{itemize}

Vegyük észre, hogy az első ACL-utasításnál nincs megadva helyettesítő maszk. Ilyenkor a
forgalomirányító az alapértelmezett 0.0.0.0 maszkot használja, vagyis vagy a teljes címnek
egyeznie kell, vagy az ACL ezen sora nem fog illeszkedni, és a forgalomirányító az ACL
következő sorára lép tovább.

A normál ACL-utasítások teljes szintaxisa a következő:\\
{\small\verb+Router(config)#access-list hozzáférési-lista-száma {deny | permit | remark} forrás [forrás-helyettesítő-maszkja] [log]+}

A \verb|remark| (megjegyzés) kulcsszó az ACL-eket könnyebben érthetővé teszi. A megjegyzések
hossza nem haladhatja meg a 100 karaktert.
Példa:
\begin{verbatim}
	access-list 1 remark Csak Jones állomását engedélyezzük
	access-list 1 permit 171.69.2.88
\end{verbatim}

Normál ACL-t törölni a parancs no változatával lehet. Ennek szintaxisa:\\
\verb|Router(config)#no access-list hozzáférési-lista-száma|

Az ip access-group parancs hozzákapcsol egy normál ACL-t egy interfészhez:
\verb+Router(config)#ip access-group {access-list-number | access-list-name} {in | out}+

A táblázat a szintaxisban használt paraméterek leírásait tartalmazza:\\
\begin{tabularx}{\linewidth}{l|X}
	Paraméter & Leírás\\
	\hline
	access-list-nurnber & Az ACL azonosító száma. Decimális szám 1 -99 (normál IP ACL) és 1300 - 1999 (kibővitett IP ACL).\\[1pt]
	deny & Megtagadja a hozzáférést, ha a feltételek teljesülnek.\\[1pt]
	permit & Engedélyezi a hozzáférést, ha a feltételek teljesülnek.\\[1pt]
	remark & A remark (megjegyzés) parancs használata a listák könnyebb megértését és megkeresését segíti.\\[1pt]
	source & Annak a hálózatnak vagy állomásnak a címe, ahonnan a csomagot elküldik. A forrás kétféleképpen adható meg:
		\begin{enumerate}[nosep]
		\item 32 bites cím megadása négy részből álló, pontokkal elválasztott decimális formátumban.
		\item az any kulcsszó a forrás rövidítése: source-wildcard of 0.0.0.0 255.255.255.55.
		\end{enumerate}\\
	source-wildcard & (Opcionális) A forrásra alkalmazandó helyettesítő bitek. A forráshelyettesítő maszkja kétféleképpen adható meg:
	\begin{enumerate}[nosep]
	\item 32 bites cím megadása négy részből álló, pontokkal elválasztott decimális formátumban. A figyelmen kívül agyandó bitpozíciókba 1-et kell írni.
	\item Az any kulcsszó a 0.0.0.0 255.255.255.255 értékű forrás és forráshelyettesítő maszk rövidítése.
	\end{enumerate}\\
	log & (Opcionális) A bejegyzésnek megfelelő csomagról egy tájékoztató célú naplózási üzenet jut a konzolra, (A konzolra küldött naplózási üzenetek részletessége a logging console paranccsal vezérelhető.)Az üzenet tartalmazza az ACL számát, a forráscímet, a csomagok számát és azt, hogy a csomagot engedélyezték-e vagy letiltották. Az üzenet az első megfelelő csomag megtalálásakor generálódik, ezután ötpercenként, megmutatva azoknak a csomagoknak a számát is, amelyek a megelőző öt percben lettek engedélyezve vagy elutasítva.
\end{tabularx}

\subsubsection{Kiterjesztett ACL-ek}
A kiterjesztett ACL-eket a normál ACL-eknél gyakrabban használjuk, mivel szélesebb körű
ellenőrzést tesznek lehetővé.\\
A kiterjesztett ACL-ek a csomagok forrás- és célcímét egyaránt ellenőrzik, illetve a
protokollok és a portszámok egyeztetésére is alkalmasak. Ezek a lehetőségek nagyobb
szabadságot biztosítanak az ACL által vizsgált adatok körülhatárolására. A csomagok
engedélyezése és tiltása forrás, cél, protokolltípus és portszám alapján egyaránt történhet.
Egy kiterjesztett ACL például engedélyezheti az elektronikus levelezést a Fa0/0 interfészről
megadott S0/0 célok felé, miközben tilthatja a fájlátviteleket és a webböngészést. A csomagok
eldobásakor bizonyos protokollok egy visszhangcsomagot küldenek a forrásnak, jelezve, hogy
a cél nem érhető el.

Egy-egy ACL-hez több utasítás is konfigurálható. Az utasítások mindegyikének ugyanazt a
hozzáférési lista számot kell tartalmaznia, így tud hivatkozni az azonos ACL-en belüli
utasításokra. Feltételutasításból tetszőleges számú adható meg, az ilyen utasítások
mennyiségét csak a forgalomirányító memóriájának nagysága korlátozza. Természetesen
minél több az utasítás, annál nehezebb az ACL megértése és kezelése.

A kiterjesztett ACL-utasítások szintaxisa meglehetősen hosszadalmas is lehet, akár a teljes
terminálablakot is kitöltheti. A helyettesítéseknél ugyancsak mód nyílik a host vagy az any
kulcsszó használatára a parancsokon belül.

A kiterjesztett ACL-utasítások végén az opcionálisan megadható TCP vagy UDP
portszámokkal tovább pontosíthatók a szabályok. A TCP/IP protokollkészlet jól ismert
portszámai az ábrán láthatók.\\
{\centering
\includegraphics[width=\linewidth]{fig/16-extACL-TCPIP_protocolls}}

A kiterjesztett ACL-ek logikai műveleteket – egyenlő (equal, eq), nem egyenlő (not equal,
neq), nagyobb mint (greater than, gt), kisebb mint (less than, lt) – is képesek végezni a
megadott protokollokon. A kiterjesztett ACL-ek hozzáférési lista száma 100 és 199 között
lehet. (Az újabb IOS-változatoknál a sorszám a 2000–2699 tartományba is tartozhat.)

Az \verb|ip access-group| paranccsal egy meglévő kiterjesztett ACL köthető hozzá egy interfészhez.
Ne feledjük, hogy interfészenként, irányonként és protokollonként csak egy ACL adható meg!
A parancs formátuma: \verb+Router(config-if)#ip access-group hozzáférési-lista-száma {in | out}+

%%----------------------------------------------------------------------------
\subsection{Bevezetés a Cisco eszközök programozásába 2}
%----------------------------------------------------------------------------
\subsubsection{A forgalomirányítási táblázatok felépítése, statikus és dinamikus routing összehasonlítása.}

\printindex\addcontentsline{toc}{section}{Tárgymutató}
\end{document}